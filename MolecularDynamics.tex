\documentclass[slidestop,mathserif,compress,xcolor=svgnames]{beamer} 
\mode<presentation>
{  
  \setbeamertemplate{background canvas}[vertical shading][bottom=blue!10,top=blue!10]
  \setbeamertemplate{navigation symbols}{}%{\insertsectionnavigationsymbol}
   \usetheme{BatonRouge}
}

\usepackage{pgf,pgfarrows,pgfnodes,pgfautomata,pgfheaps,pgfshade}
\usepackage{amsmath,amssymb,amsfonts,subfigure}
\usepackage{tabularx}
\usepackage{booktabs}
\usepackage{colortbl}
\usepackage{tikz}
\usetikzlibrary{shapes,arrows}
\usetikzlibrary{calc}
\pgfdeclarelayer{background}
\pgfdeclarelayer{foreground}
\pgfsetlayers{background,main,foreground}
\usepackage[latin1]{inputenc}
\usepackage{colortbl}
\usepackage[english]{babel}
\usepackage{hyperref}
\usepackage{movie15}
\hypersetup{
  pdftitle={Molecular Dynamics},
  pdfauthor={Alexander B. Pacheco, User Services Consultant, Louisiana State University}
}
\usepackage{times}

\setbeamercovered{dynamic}
\beamersetaveragebackground{DarkBlue!2}
\beamertemplateballitem

\usepackage[english]{babel}
\usepackage[latin1]{inputenc}
\usepackage{times}
\usepackage{amsmath}
\usepackage[T1]{fontenc}
\usepackage{graphicx}
\definecolor{DarkGreen}{rgb}{0.0,0.3,0.0}
\definecolor{Blue}{rgb}{0.0,0.0,0.8} 
\definecolor{dodgerblue}{rgb}{0.1,0.1,1.0}
\definecolor{indigo}{rgb}{0.41,0.1,0.0}
\definecolor{seagreen}{rgb}{0.1,1.0,0.1}
\definecolor{seablue}{rgb}{0.72,0.72,0.82}
\DeclareSymbolFont{extraup}{U}{zavm}{m}{n}
\DeclareMathSymbol{\vardiamond}{\mathalpha}{extraup}{87}
\newcommand*\up{\textcolor{green}{%
  \ensuremath{\blacktriangle}}}
\newcommand*\down{\textcolor{red}{%
  \ensuremath{\blacktriangledown}}}
\newcommand*\const{\textcolor{darkgray}%
  {\textbf{--}}}

\setbeamercolor{uppercol}{fg=white,bg=red!30!black}%
\setbeamercolor{lowercol}{fg=black,bg=red!15!white}%
\newenvironment{colorblock}[4]
{
\setbeamercolor{upperblock}{fg=#1,bg=#2}
\setbeamercolor{lowerblock}{fg=#3,bg=#4}
\begin{beamerboxesrounded}[upper=upperblock,lower=lowerblock,shadow=true]}
{\end{beamerboxesrounded}}

\newenvironment{colorblock2}[2]
{
\begin{beamerboxesrounded}[upper=#1,lower=#2,shadow=true]}
{\end{beamerboxesrounded}}

\title[Comp. Chem.]{Molecular Dynamics}


\author[Alex Pacheco]{Alexander~B.~Pacheco}
       
\institute[High Performance Computing @ Louisiana State University - http://www.hpc.lsu.edu] {\inst{}User Services Consultant\\LSU HPC \& LONI\\sys-help@loni.org}

\date[July 6, 2011\hspace{2cm}\insertframenumber/\inserttotalframenumber]{\tiny{LSU HPC Training Series\\Louisiana State University\\July 6, 2011}}
     
\subject{Talks}
% This is only inserted into the PDF information catalog. Can be left
% out. 




% If you have a file called "university-logo-filename.xxx", where xxx
% is a graphic format that can be processed by latex or pdflatex,
% resp., then you can add a logo as follows:

\pgfdeclareimage[height=1.5cm]{loni-logo}{LONI}
\pgfdeclareimage[height=0.5cm]{gold-logo}{GOLD_LSU}
\pgfdeclareimage[height=0.8cm]{purple-logo}{PURPLELSU}
\pgfdeclareimage[height=1.5cm]{lsuccttower-logo}{FULLCOLORTOWER_VERT}%{qtp}
\pgfdeclareimage[height=0.5cm]{lsutower-logo}{ProcessHorizontal}
% \logo{\pgfuseimage{purple-logo}}

% put MSRI logo in bottom left
\setbeamertemplate{sidebar left}{%
   \vfill%
   \rlap{\hskip0.02cm%
         %
         {\pgfuseimage{loni-logo}}}%
   \vskip-7pt%
   \llap{\usebeamertemplate***{navigation symbols}\hskip0.1cm}%
   \vskip2pt%
}
\setbeamertemplate{sidebar right}{%
   \vfill%
   \rlap{\hskip-2.1cm%
         %
         {\pgfuseimage{purple-logo}}}%
   \vskip-5pt%
   \llap{\usebeamertemplate***{navigation symbols}\hskip0.1cm}%
   \vskip2pt%
}

% Delete this, if you do not want the table of contents to pop up at
% the beginning of each subsection:
% \AtBeginSection[]
% {
%   \begin{frame}<beamer>
%     \frametitle{\small{Outline}}
%     \small
%     \tableofcontents[currentsection,currentsubsection]
%   \end{frame}
% }

\begin{document}

\frame{\titlepage}

\normalsize
\begin{frame}[label=toc,squeeze]
  \footnotesize
  \frametitle{\small{Outline}}
  \tableofcontents
\end{frame}

\footnotesize

%\section{Tutorial Goals}
%\begin{frame}
%\frametitle{\small Tutorial Goals}
%\begin{itemize}
%\item Cover the fundamentals of Molecular Dynamics Simulation: Ab-Initio and %Classical. 
%\item Expose researchers to the theory and computational packages used for MD simulations. 
%\item Worked out examples for various computational packages such as CPMD, Gaussian, GAMESS and NWCHEM. 
%\item[] {\color{green!30!black} Linux machines, LONI and LSU HPC at /home/apacheco/CompChem.} Go ahead with the examples if you want {\color{red!70!black}but hold off all questions until tutorial is complete}.
%(Time permitting or if requested, overview of using/writing scripts to analyze results will be provided.) .
%\item My Background: {\it Ab-Initio} Molecular Dynamics.
%\item Questions about examples/tutorials and/or using Electronic Structure codes for AIMD, email me at sys-help@loni.org or apacheco@cct.lsu.edu
%\end{itemize}
%\end{frame}
% \part{Introduction}
\section{Introduction}
\begin{frame}
\frametitle{\small What is Computational Chemistry}
\begin{itemize}
\item {\bf Computational Chemistry} is a branch of chemistry that uses principles of computer science to assist in solving chemical problems.
\item Uses the results of theoretical chemistry, incorporated into efficient computer programs.
\item Application to single molecule, groups of molecules, liquids or solids.
\item Calculates the structure and properties such as relative energies, charge distributions, dipole and multipole moments, spectroscopy, reactivity, etc.
\item Computational Chemistry Methods range from
 \begin{enumerate}
 {\footnotesize
  \item Highly accurate ({\it Ab-initio},DFT) feasible for small systems
  \item Less accurate (semi-empirical)
  \item Very Approximate (Molecular Mechanics, Classical Mechanics) large systems
}
 \end{enumerate}
\end{itemize}
\end{frame}

\begin{frame}
\frametitle{ \small }
\begin{columns}
\column{11cm}
\begin{colorblock}{white}{green!30!black}{black}{green!15!white}{\bf Theoretical Chemistry: broadly can be divided into two main categories}
\begin{enumerate}
\item Static Methods {\Large$\Rightarrow$} {\color{blue}Time-Independent Schr\"{o}dinger Equation}
\begin{enumerate}
 {\footnotesize
\item[$\vardiamond$] Quantum Chemical/\emph{Ab Initio} /Electronic Structure Methods
\item[$\vardiamond$] Molecular Mechanics
}
\end{enumerate}
\item Dynamical Methods {\Large$\Rightarrow$} {\color{blue}Time-Dependent Schr\"{o}dinger Equation}
\begin{enumerate}
 {\footnotesize
\item[$\vardiamond$] Classical Molecular Dynamics
\item[$\vardiamond$] Semi-classical and \textit{Ab-Initio} Molecular Dynamics
}
\end{enumerate}
\end{enumerate}
\end{colorblock}
\end{columns}
\end{frame}

%\section{Molecular Dynamics}
\section{Fundamentals of Molecular Dynamics}
\begin{frame}
\frametitle{\small Molecular Dynamics}
\begin{columns}
\column{12cm}
\begin{beamerboxesrounded}[upper=uppercol,lower=lowercol,shadow=true]{\bf Why Molecular Dynamics?}
\begin{itemize}
\item Electronic Structure Methods are applicable to systems in gas phase under low pressure (vaccum).
\item Majority of chemical reactions take place in solution at some temperature with biological reactions usually at specific pH's.
\item Calculating molecular properties taking into account such environmental effects which can be dynamical in nature are not adequately described by electronic structure methods.
\end{itemize}
\end{beamerboxesrounded}
%\vspace{0.15cm}
{
\begin{colorblock}{white}{blue!30!black}{black}{blue!15!white}{\bf Molecular Dynamics}
\begin{itemize}
\item Generate a series of time-correlated points in phase-space (a trajectory).
\item Propagate the initial conditions, position and velocities in accordance with Newtonian Mechanics. ${\bf F}=m{\bf a}=-{\boldsymbol\nabla}V$
\item Fundamental Basis is the \textbf{Ergodic Hypothesis}: the average obtained by following a small number of particles over a
long time is equivalent to averaging over a large number of particles for a short time.
\end{itemize}
\end{colorblock}
}
\end{columns}
\end{frame}

\begin{frame}
\frametitle{}
\begin{colorblock}{white}{blue!30!black}{black}{blue!15!white}{\bf Applications of Molecular Dynamics}
\begin{itemize}
\item Liquids, transport phenomena like viscosity and heat flow
\item Crystal structure, defects 
\item Solids: Fracture, Friction between two solids
\item Surface Phenomena, adsorption, diffusion
\item Clusters are a bridge between molecular systems and solids, catalysis
\item Biomolecules
\item Electronic Properties and Dynamics
\end{itemize}
\end{colorblock}
\begin{colorblock}{white}{blue!30!black}{black}{blue!15!white}{\bf Limitations of Molecular Dynamics}
\begin{itemize}
\item Use of classical forces: Nuclear quantum effects become important for lighter nuclei (e.g. H, He), low temperatures.
\item How realistic are the forces?
\item How long should the simulation run?
\item Size of system? 
\end{itemize}
\end{colorblock}
\end{frame}

\begin{frame}
\frametitle{\small General Schematic for Molecular Dynamics Program}
\vspace{-0.5cm}
\begin{columns}
\column{4.5cm}
\begin{colorblock}{white}{blue!30!black}{black}{blue!15!white}{}
\begin{enumerate}
\item Setup: Read input parameters
\item Initialize: Obtain initial positions and velocities
\item Evaluate: Potential Energy and Forces on nuclei
\item Propagate nuclei using an appropriate time integration algorithm
\item Check if Dynamics is complete. 
\item If incomplete update variables and goto Step 3.
\item If complete end dynamics or carry out all required analysis.
\end{enumerate}
\end{colorblock}
\column{7cm}
\begin{colorblock}{white}{blue!30!black}{black}{blue!15!white}{}
%\begin{center}
%\includegraphics[width=0.8\textwidth,clip=true]{md-scheme}
%\end{center}
\scriptsize{
\tikzstyle{decision} = [ellipse, draw, fill=red!50!yellow, 
    text width=6em, text badly centered, node distance=1.5cm, inner sep=0pt, minimum height=4.5em]
\tikzstyle{block} = [rectangle, draw, fill=green!50!black, 
    text width=10em, text centered, rounded corners, minimum height=3em]
\tikzstyle{line} = [draw, -latex']
\tikzstyle{cloud} = [draw, ellipse, fill=green!20,node distance=1.5cm, minimum height=2em]
\begin{tikzpicture}[node distance = 1.35cm, auto]
    \node [block,fill=gray](setup){{\bf 1. Setup}\\Read Input Parameters};
    \node [block, below of=setup, fill=gray](initial){{\bf 2. Initialize}\\Positions \& velocities};
    \node [block, below of=initial](solve){{\bf 3. Evaluate}\\Potential \& Forces};
    \node [block, below of=solve](integrate){{\bf 4. Propagate}\\Find new \\Potential \& Forces};
    \node [decision ,left of=integrate,node distance=2.65cm, fill=red!50!yellow!60!green,text width=4.75em, minimum height=2.5em](update){\bf 6. Update};
    \node [decision, below of=integrate,node distance=1.5cm](decide){{\bf 5. Dynamics Complete?}};
    \node [decision, right of=decide, text width=4em, node distance=2.65cm,minimum height=2em, fill=red!80!black] (stop) {\bf 7. END};

    \path [line](setup) -- (initial);
    \path [line](initial) -- (solve);
    \path [line](solve) -- (integrate);
    \path [line](integrate) -- (decide);
    \path [line](decide) -- node[near end]{yes}(stop);
    \path [line](decide) -| node[near start]{no}(update);
    \path [line](update) |- (solve);
\end{tikzpicture}
}
\end{colorblock}
\end{columns}
\end{frame}

\begin{frame}
\frametitle{\small Models for Physical System}
\begin{colorblock}{white}{blue!30!black}{black}{blue!15!white}{}
\begin{itemize}
\item Model for the physical system being simulated
\item Choose the potential which is a function $\mathcal{V}({\bf r}_1\cdots{\bf r}_N)$ of the positions of the nuclei, representing the potential energy of the system when the atoms are arranged in that specific configuration.
\item Forces are obtained as gradients of the potential
\begin{align*}
{\bf F}_i = -\nabla_{{\bf r}_i}\mathcal{V}({\bf r}_1\cdots{\bf r}_i\cdots{\bf r}_N)
\end{align*}
\item Simplest choice for potential is a sum of pairwise interactions
\begin{align*}
\mathcal{V}({\bf r}_1\cdots{\bf r}_N) = \sum_i\sum_{j>i} V(|{\bf r}_i - {\bf r}_j|)
\end{align*}
\end{itemize}
\end{colorblock}
\end{frame}\begin{frame}
\begin{colorblock}{white}{blue!30!black}{black}{blue!15!white}{\bf Potential Energy Functions}
\begin{itemize}
\item Pair Potentials:
\begin{enumerate}
{\footnotesize
\item Attraction
\begin{itemize}
{\footnotesize
\item Long range
\item Dispersive forces, instantaneous dipole interactions
}
\end{itemize}
\item Repulsion
\begin{itemize}
{\footnotesize
\item Short range
\item Exchange forces, overlap of electron cloud.
}
\end{itemize}}
\end{enumerate}
\item Lennard-Jones (LJ) Potential
\begin{align*}
U(r) = 4\epsilon\left[\left(\dfrac{\sigma}{r}\right)^{12} - \left(\dfrac{\sigma}{r}\right)^6\right]
\end{align*}
\begin{enumerate}
{\footnotesize
\item The LJ potential decays rapidly, significant computation time can be saved by
neglecting pair interactions beyond a cut-off. [$\mathrm{for}\, r_c > 2.5\sigma,\,U(r)=0$]
}
\end{enumerate}
\item Potential from Molecular Mechanics Force fields like AMBER, CHARMM, Drieding etc or from electronic structure calculations.
\end{itemize}
\end{colorblock}
\end{frame}
\begin{frame}
%\frametitle{\small Molecular Mechanics}
\footnotesize{
\begin{columns}
\column{6.5cm}
\begin{colorblock}{white}{blue!30!black}{black}{blue!8!white}{General form of Molecular Mechanics equations}
\begin{align*}
E&={\color{red}E_{\mathrm{bond}}}+{\color{blue}E_{\mathrm{angle}}} + {\color{tigerspurple}E_{\mathrm{torsion}}} + {\color{DarkGreen}E_{\mathrm{vdW}}} + {\color{indigo}E_{\mathrm{elec}}}\\
&={\color{red}\frac{1}{2}\sum_{\mathrm{bonds}}K_b(b-b_0)^2 \hspace{1.6cm}\mathrm{Bond}}\\
&+{\color{Blue}\frac{1}{2}\sum_{\mathrm{angles}}K_\theta(\theta-\theta_0)^2 \hspace{1.6cm}\mathrm{Angle}}\\
&+{\color{tigerspurple}\frac{1}{2}\sum_{\mathrm{dihedrals}}K_\phi\left[1+\cos(n\phi)\right]^2 \hspace{0.55cm}\mathrm{Torsion}}\\
&+\sum_{\mathrm{non bonds}}\left\{ \begin{array}{l}
{\color{DarkGreen}\left[\left(\dfrac{\sigma}{r}\right)^{12}-\left(\dfrac{\sigma}{r}\right)^{6}\right] \mathrm{van\,der\,Waals}}\\
+ {\color{indigo}\dfrac{q_1q_2}{Dr} \hspace{1.6cm}\mathrm{Electrostatics}}
       \end{array} \right.
\end{align*}
\end{colorblock}
\column{5.2cm}
\vspace{0.6cm}
\begin{colorblock}{white}{blue!30!black}{black}{blue!8!white}{}
\includegraphics[width=5.2cm,keepaspectratio=true,clip=true]{MM_PEF}
\let\thefootnote\relax\footnotetext{\tiny Picture taken from }
\let\thefootnote\relax\footnotetext{\tiny http://en.wikipedia.org/wiki/Molecular\_mechanics}
\end{colorblock}
\end{columns}
}
\end{frame}

\begin{frame}
\frametitle{}
{\scriptsize
\begin{colorblock}{white}{blue!30!black}{black}{blue!15!white}{\bf Initial Positions}
\begin{enumerate}
\item From experimental results
\item Assign to lattice positions
\end{enumerate}
\end{colorblock}
\begin{colorblock}{white}{blue!30!black}{black}{blue!15!white}{\bf Initial Velocity}
\begin{itemize}
\item Trajectories from previous simulation
\item Set to desired temperature of simulation
\begin{enumerate}
{\scriptsize
\item Assign a Maxwell -Boltzmann distribution
\begin{align*}
f_v(v_x) =\sqrt{\dfrac{m}{2\pi kT}}\exp\left[\dfrac{-mv^2_x}{2kT}\right]
\end{align*}
\item Random Uniform distribution: Random velocities scaled to desired temperature
\begin{align*}
v_{desired} = \sqrt{\dfrac{T_{desired}}{T_{current}}}v_{current}
\end{align*}
}
\end{enumerate}
\item Total linear momentum is zero
\begin{align*}
\sum_i m_iv_i = 0\hspace{0.5cm}{\mathrm or}\hspace{0.5cm}
v^{desired}_j = v^{current}_j - \dfrac{\sum_i m_iv^{current}_i }{m_j}
\end{align*}
\end{itemize}
\end{colorblock}
}
\end{frame}

\begin{frame}
\begin{colorblock}{white}{blue!30!black}{black}{blue!15!white}{Boundary Conditions}
\begin{itemize}
\item What do we do at the boundaries of the simulated system?
\begin{enumerate}
{\footnotesize
\item Nothing: Not particularly useful for infinite systems but good for single isolated or finite systems.
\item Use Periodic Boundary Conditions (PBC) for simulating liquids, solids and clusters.
}
\end{enumerate}
\end{itemize}
\end{colorblock}
\end{frame}

\begin{frame}
\begin{colorblock}{white}{blue!30!black}{black}{blue!15!white}{\bf Periodic Boundary Conditions}
\begin{itemize}
\item Consider a small subset or representative size of the infinite or large system and replicate to infinity (or your desired system size).
\begin{center}
\includegraphics[width=0.2\textwidth,clip=true]{pbc}
\end{center}
\item For each particle in our simulation box, there are infinite particles located at positions
\begin{align*}
{\bf r} + l{\bf a} + m{\bf b} + n{\bf c}\, , \, (l,m,n=-\infty\cdots\infty)
\end{align*}
where {\bf a}, {\bf b} and {\bf c} are the lattice dimensions of the box.
\item Thus each particle interacts not only with the other particles in the simulation box but also with their images in every other box.
\item The simulated box and its images have the same size, shape and number of atoms with the same position, velocity and acceleration.
\item How do calculate these infinite interactions which will undoubtedly increase the simulation cost?
\end{itemize}
\end{colorblock}
\end{frame}

\begin{frame}
\begin{colorblock}{white}{blue!30!black}{black}{blue!15!white}{\bf Minimum Image Criterion}
\begin{itemize}
\item If the potential has a finite range, say $R_c$, then two particles at a distance greater than $R_c$ do not interact with each other.
\item If the size of the simulation box is large than $2R_c$, then each particle $i$ will at most interact with one particle $j$ in the simulation box or the neighboring box.
\item Thus interaction between particle $i$ with particle $j$ and its images in the replicated boxes will only consist of one interaction between the closest particle.
\begin{center}
\includegraphics[width=0.6\textwidth,clip=true]{minimagecrit}
\end{center}
\end{itemize}
\end{colorblock}
\end{frame}

\begin{frame}
\begin{colorblock}{white}{blue!30!black}{black}{blue!15!white}{}
\begin{itemize}
\item Periodic Boundary Conditions are well suited for modeling infinite systems such as liquids and solids.
\item What about surfaces? How do you treat surface effects?
\item For surface simulation, a model of a slab is used i.e. a thick slice of the material delimited with two free surfaces.
\item This is done by removing the PBC from one direction, say $z$ direction while maintaining PBC in the $xy$ plane.
\item  Thus the system is infinite in the $xy$ plane and finite in the $z$ direction.
\item Removing PBC in two directions gives rise to a wire model.
\item Removing PBC's completely gives rise to a cluster model.
\end{itemize}
\end{colorblock}
\end{frame}

\begin{frame}
\begin{colorblock}{white}{blue!30!black}{black}{blue!15!white}{\bf Time Integration}
\begin{itemize}
\item The main ingredient of Molecular Dynamics is the time integration algorithm: integrate equation of motions of particles to follow trajectories.
\item Based on finite difference methods: time is discretized onto a finite grid with the time step $\Delta t$ being the distance between consecutive points on the grid.
\begin{center}
\includegraphics[width=0.6\textwidth,clip=true]{trajec}
\end{center}
\item From position and time derivative at time $t$, obtain positions and time derivatives at time $t+\Delta t$
\end{itemize}
\end{colorblock}
\end{frame}

\begin{frame}[allowframebreaks]
\begin{colorblock}{white}{blue!30!black}{black}{blue!15!white}{\bf Verlet Algorithm}
\begin{itemize}
\item Taylor expansion around ${\bf r}(t)$
\begin{align*}
{\bf r}(t+\Delta t) &= 2{\bf r}(t) + {\bf r}(t-\Delta t) + {\bf a}(t)({\Delta t})^2 + \mathcal{O}(\Delta t)^4\\
{\bf a}(t) &= (-1/m)\nabla\mathcal{V}({\bf r}(t))\\
{\bf v}(t) &= \dfrac{{\bf r}(t+\Delta t) - {\bf r}(t-\Delta t) }{2\Delta t}
\end{align*}
\end{itemize}
\end{colorblock}

\begin{colorblock}{white}{blue!30!black}{black}{blue!15!white}{\bf Velocity Verlet Algorithm}
\begin{itemize}
\item Obtain velocity at half step and position at full step.
\begin{align*}
{\bf v}(t+\Delta t/2) &=  {\bf v}(t)  + 1/2{\bf a}(t){\Delta t}\\
{\bf r}(t+\Delta t) &= {\bf r}(t) + {\bf v}(t+\Delta t/2)\Delta t\\
{\bf a}(t+\Delta t) &= (-1/m)\nabla\mathcal{V}({\bf r}(t+\Delta t))\\
{\bf v}(t+\Delta t) &= {\bf v}(t+\Delta t/2) + 1/2{\bf a}(t+\Delta t)\Delta t
\end{align*}
\end{itemize}
\end{colorblock}

\begin{colorblock}{white}{blue!30!black}{black}{blue!15!white}{\bf Predictor Corrector Algorithm}
\begin{itemize}
\item Predictor Step: Predict position and time derivatives  time $t+\Delta t$ by Taylor expansion at time $t$.
\item Compute Forces as gradient of potentials at the predicted positions. The difference between the predicted acceleration and the calculated acceleration is the "error signal"
\begin{align*}
\Delta a(t+\Delta t) = a^C(t+\Delta t) - a^P(t+\Delta t)
\end{align*}
\item Corrector Step: Use the "error signal" to correct the positions and its time derivatives. 
\begin{align*}
r^C(t+\Delta t)  &= r^P(t+\Delta t) + c_0\Delta a(t+\Delta t)\\
v^C(t+\Delta t)  &= v^P(t+\Delta t) + c_1\Delta a(t+\Delta t)\\
a^C(t+\Delta t)  &= a^P(t+\Delta t) + c_2\Delta a(t+\Delta t)\\
b^C(t+\Delta t)  &= b^P(t+\Delta t) + c_3\Delta a(t+\Delta t)
\end{align*}
\item The coefficients maximizes stability and are dependent on specific algorithm
\item {\color{blue}{\bf Gear Predictor Corrector Algorithm:}\\ $c_0=1/6$, $c_1=5/6$, $c_2=1$ and $c_3=1/3$.}
\end{itemize}
\end{colorblock}

%\begin{colorblock}{white}{blue!30!black}{black}{blue!15!white}{\bf Leap Frog Algorithm}
%\begin{itemize}
%\item Similar to the Velocity Verlet algorithm.
%\begin{align*}
%{\bf r}(t+\Delta t) &= {\bf r}(t) + {\bf v}(t+\Delta t/2)\, \Delta t\\
%{\bf v}{(t+\Delta t/2)} &= {\bf v}{(t-\Delta t/2)} + {\bf a}(t)\, \Delta t
%\end{align*}
%\end{itemize}
%\end{colorblock}

%\begin{colorblock}{white}{blue!30!black}{black}{blue!15!white}{\bf The Runge-Katta method}
%For an initial value problem  specified as follows.
%\begin{align*}
%y' = f(t, y) \hspace{0.1cm};\hspace{0.1cm}
%y(t_0) = y_0
%\end{align*}
%the RK equations are
%\begin{align*}
%y_{n+1} = y_n + \tfrac{1}{6} h \left(k_1 + 2k_2 + 2k_3 + k_4 \right)\hspace{0.1cm} ; \hspace{0.1cm}
%t_{n+1} = t_n + h
%\end{align*}
%where $y_{n+1}$ is the RK approximation of $y(t_{n+1})$, and
%\begin{align*} 
%k_1 = f(t_n, y_n)\hspace{0.1cm} &; \hspace{0.1cm}
%k_2 = f(t_n + \tfrac{1}{2}h, y_n +  \tfrac{1}{2} h k_1)\\
%k_3 = f(t_n + \tfrac{1}{2}h, y_n +   \tfrac{1}{2} h k_2)\hspace{0.1cm} &; \hspace{0.1cm}
%k_4 = f(t_n + h, y_n + h k_3)
%\end{align*}
%\end{colorblock}
\end{frame}

\begin{frame}[allowframebreaks]
\frametitle{\small Analysis of Trajectory}
\begin{itemize}
\item Kinetic Energy
\begin{align*}
K = \dfrac{1}{2}\sum_im_iv^2_i
\end{align*}
\item Temperature: from average kinetic energy using equipartition theorem
\begin{align*}
T = \dfrac{2K}{3k_B}
\end{align*}
\item Pressure: from virial theorem
\begin{align*}
P=\dfrac{Nk_BT}{V} -\dfrac{1}{N_{dim}}\left\langle\sum^N_i{\bf r}_i\cdot{\bf F}_i\right\rangle
\end{align*}
where $N$ is number of particles and $N_{dim}$ is dimensionality of the system.
\item Diffusion Coefficient: related to mean square displacement
\begin{align*}
D = \dfrac{1}{2N_{dim}}\lim_{t\rightarrow\infty}\dfrac{\left\langle|{\bf r}(t) - {\bf r}(0)|^2\right\rangle}{t}
\end{align*}
\item Spectral analysis
\begin{enumerate}
{\footnotesize
\item FT-VAC: Fourier Transform of Velocity Auto-Correlation function
\begin{align*}
V(\omega) = \dfrac{1}{{2\pi}}\int\,\exp(-\imath\omega t)\left\langle{\bf v}(t)\cdot{\bf v}(0)\right\rangle
\end{align*}
if mass weighted velocities $\boldsymbol{\cal V}_i=\sqrt{m_i}{\bf v}_i$ are used, then $V(\omega)$ is the kinetic energy spectra.
\item FT-DAC: Fourier Transform of Dipole Auto-Correlation function, related to IR spectra
\begin{align*}
S(\omega) = \dfrac{1}{{2\pi}}\int\,\exp(-\imath\omega t)\left\langle{\boldsymbol \mu}(t)\cdot{\boldsymbol \mu}(0)\right\rangle
\end{align*}
\item Short-Time Fourier Transform (STFT): 2D time-frequency spectra to simulate pump-probe experiments 
\begin{align*}
{\cal K}(t,\omega) = \frac{1}{2\pi}\int^{\infty}_{-\infty}dt^\prime
\left\langle{\boldsymbol{\cal V}}(t)\cdot{\boldsymbol{\cal V}}(t^\prime)\right\rangle
% f(t,t^\prime)
H(t,t^\prime)\exp(-\imath\omega t^\prime)\label{eq:stft-vac}
\end{align*}
where %$f(t,t^\prime)$ is a time correlation function and 
$H(t,t^\prime)$ is a window function
}
\end{enumerate}
\end{itemize}
\end{frame}

\beamertemplateshadingbackground{seablue}{seablue} % New background
\begin{frame}
\frametitle{\small Example code for MD}
\begin{itemize}
\item On LONI and LSU HPC Linux systems: /home/apacheco/CompChem/MD\_Prog2Prod
\item Equilibration of liquid Hydrogen.
\item Courtesy: Matt McKenzie, formerly LSU HPC now at NICS.
\item Input File: fort.40
\item Output File: fort.44 (energy data) and fort.77 (xyz file of dynamics)
\item In directory crystal, crystal.f90 to generate lattice structure, courtesy Furio Ercolessi {\color{Blue}\url{http://www.fisica.uniud.it/~ercolessi/md}}
\end{itemize}
\begin{center}
\includemovie[autoplay,text={\includegraphics[height=4cm,keepaspectratio=true]{liquid-H-md}},repeat=10]{}{}
{liquid-H-md.mpg}
\end{center}
\end{frame}
\beamertemplateshadingbackground{blue!5}{blue!5} % back to original background

\section{{\it Ab Initio} Molecular Dynamics Theory} 
\begin{frame}
\frametitle{\small {\it Ab Initio} Molecular Dynamics: Theory}
\begin{itemize}
\item Solve the time-dependent Schr\"{o}dinger equation
\begin{align*}
\imath\hbar\dfrac{\partial}{\partial t}\Psi({\bf R},{\bf r},t) = \hat{H}\Psi({\bf R},{\bf r},t)
\end{align*}
with 
\begin{align*}
\Psi({\bf R},{\bf r},t) = \chi({\bf R},t)\Phi({\bf r},t)
\end{align*}
and 
\begin{align*}
\hat{H} = -\sum_I\dfrac{\hbar^2}{2M_I}\nabla^2_I +\underbrace{\dfrac{-\hbar^2}{2m_e}\nabla^2_i + V_{n-e}(\bf r,R)}_{H_e(\bf r,R)}
\end{align*}
\item Obtain coupled equations of motion for electrons and nuclei: Time-Dependent Self-Consistent Field (TD-SCF) approach.
\begin{align*}
\imath\hbar\dfrac{\partial\Phi}{\partial t} &= \left[-\sum_i\dfrac{\hbar^2}{2m_e}\nabla^2_i + \langle\chi|V_{n-e}|\chi\rangle\right]\Phi\\
\imath\hbar\dfrac{\partial\chi}{\partial t} &= \left[-\sum_I\dfrac{\hbar^2}{2M_I}\nabla^2_I + \langle\Phi|H_e|\Phi\rangle\right]\chi
\end{align*}
\end{itemize}
\end{frame}

\begin{frame}
%\frametitle{\small Molecular Dynamics Theory}
\begin{itemize}
%\item Alternative approach: Decouple electronic and nuclei motions.
%\item Solve for the clamped nuclei:
%\begin{align*}
%\hat{H}_e\Phi_k({\bf r};{\bf R}) = E_k({\bf R})\Phi_k({\bf r};{\bf R})
%\end{align*}
%\item Equation of motion for nuclear wavefunction is
%\begin{align*}
%\imath\hbar\dfrac{\partial|\chi\rangle}{\partial t} &= \left[-\sum_I\dfrac{\hbar^2}{2M_I}\nabla^2_I + E({\bf R})\right]|%\chi\rangle
%\end{align*}
\item Define nuclear wavefunction as
\begin{align*}
\chi({\bf R},t) = A({\bf R},t)\exp\left[iS({\bf R},t)/\hbar\right]
\end{align*}
where $A$ and $S$ are real.
\item Solve the time-dependent equation for nuclear wavefunction and take classical limit ($\hbar\rightarrow0$) to obtain 
\begin{align*}
\dfrac{\partial S}{\partial t} + \sum_I\dfrac{\hbar^2}{2M_I}(\nabla_IS)^2 + \langle\Phi|H_e|\Phi\rangle = 0
\end{align*}
an equation that is isomorphic with the Hamilton-Jacobi equation with the classical Hamilton function given by
\begin{align*}
\mathcal{H}(\{{\bf R}_I\},\{{\bf P}_I\}) = \sum_I \dfrac{\hbar^2}{2M_I}{\bf P}^2_I + V(\{{\bf R}_I\})
\end{align*}
where
\begin{align*}
{\bf P}_I \equiv \nabla_IS\hspace{1cm}\mathrm{and}\hspace{1cm}
V(\{{\bf R}_I\}) = \langle\Phi|H_e|\Phi\rangle
\end{align*}
\item Obtain equations of nuclear motion from Hamilton's equation
\begin{align*}
\dfrac{d{\bf P}_I}{dt} = -\dfrac{d\mathcal{H}}{d{\bf R}_I}&\Rightarrow M\ddot{\bf R}_I = -\nabla _IV\\
\dfrac{d{\bf R}_I}{dt} &= \dfrac{d\mathcal{H}}{d{\bf P}_I}
\end{align*}
\end{itemize}
\end{frame}

\begin{frame}
\begin{itemize}
\item Replace nuclear wavefunction by delta functions centered on nuclear position to obtain
\begin{align*}
i\hbar\dfrac{\partial\Phi}{\partial t} = H_e({\bf r},\{{\bf R}_I\})\Phi({\bf r};\{{\bf R}_I\},t)
\end{align*}
\item This approach of simultaneously solving the electronic and nuclear degrees of freedom by incorporating feedback in both directions is known as \textbf{Ehrenfest Molecular Dynamics}.
\item Expand $\Phi$ in terms of many electron wavefunctions or  determinants
\begin{align*}
\Phi({\bf r};\{{\bf R}_I\},t) = \sum_i c_i(t)\Phi_i({\bf r};\{{\bf R}_I\})
\end{align*}
with matrix elements
\begin{align*}
H_{ij} = \langle\Phi_i|H_e|\Phi_j\rangle
\end{align*} 
\item Inserting $\Phi$ in the TDSE above, we get
\begin{align*}
\imath\hbar\dot{c}_i(t) = c_i(t)H_{ii} -\imath\hbar\sum_{I,i}\dot{\bf R}_I{\bf d}^{ij}_I
\end{align*}
with non-adiabatic coupling elements given by
\begin{align*}
{\bf d}^{ij}_I({\bf R}_I) = \langle\Phi_i|\nabla_I|\Phi_j\rangle
\end{align*}
\end{itemize}
\end{frame}

\begin{frame}[allowframebreaks]
\begin{itemize}
\item Up to this point, no restriction on the nature of $\Phi_i$ i.e. adiabatic or diabatic basis has been made. 
\item Ehrenfest method rigorously includes non-adiabatic transitions between electronic states within the framework of classical nuclear motion and mean field (TD-SCF) approximation to the electronic structure.
\item Now suppose, we define $\{\Phi_i\}$ to be the adiabatic basis obtained from solving the time-independent Schr\"{o}dinger equation,
\begin{align*}
H_e({\bf r},\{{\bf R}_I\})\Phi_i({\bf r};\{{\bf R}_I\}) = E_i(\{{\bf R}_I\})\Phi_i({\bf r};\{{\bf R}_I\})
\end{align*}
\item The classical nuclei now move along the adiabatic or Born-Oppenheimer potential surface. Such dynamics are commonly known as \textbf{Born-Oppenheimer Molecular Dynamics} or BOMD.
\item If we restrict the dynamics to only the ground electronic state, then we obtain ground state BOMD.
\item If the Ehrenfest potential $V(\{{\bf R}_I\})$ is approximated to a global potential surface in terms of many-body contributions $\{v_n\}$.
\begin{align*}
V(\{{\bf R}_I\}) \approx V^{approx}_e({\bf R}) = \sum^{N}_{I=1}v_1({\bf R}_I) + \sum^N_{I>J}v_2({\bf R}_I,{\bf R}_J) +\sum^N_{I>J>K}v_3({\bf R}_I,{\bf R}_J,{\bf R}_K) + \cdots
\end{align*}
\item[] 
\item Thus the problem is reduced to purely classical mechanics once the $\{v_n\}$ are determined usually Molecular Mechanics Force Fields. This class of dynamics is most commonly known as {\bf Classical Molecular Dynamics}.
\item Another approach to obtain equations of motion for \text{ab-initio} molecular dynamics is to apply the Born-Oppenheimer approximation to the full wavefunction $\Psi({\bf r},{\bf R},t)$
\begin{align*}
\Psi({\bf r},{\bf R},t) = \sum_k \chi_k({\bf R},t)\Phi_k({\bf r};{\bf R}(t))
\end{align*}
where
\begin{align*}
H_e\Phi_k({\bf r};{\bf R}(t)) = E_k({\bf R}(t))\Phi_k({\bf r};{\bf R}(t))
\end{align*}
\item Assuming that the nuclear dynamics does not change the electronic state, we arrive at the equation of motion for nuclear wavefunction 
\begin{align*}
\imath\hbar\dfrac{\partial}{\partial t}\chi_k({\bf R},t) = \left[\sum_I-\dfrac{\hbar^2}{2M_I}\nabla^2_I + E_k({\bf R})\right]\chi_k({\bf R},t)
\end{align*}
\item The Lagrangian for this system is given by.
\begin{align*}
\mathcal{L} = \hat{T} - \hat{V}
\end{align*}
\item Corresponding Newton's equation of motion are then obtained from the associated Euler-Lagrange equations,
\begin{align*}
\dfrac{d}{dt}\dfrac{\partial\mathcal{L}}{\partial\dot{\bf R}_I} = \dfrac{\partial\mathcal{L}}{\partial{\bf R}_I}
\end{align*}
\item The Lagrangian for ground state BOMD is
\begin{align*}
\mathcal{L}_\mathrm{BOMD} = \sum_I\dfrac{1}{2}M_I\dot{\bf R}^2_I - \min_{\Phi_0}\langle\Phi|H_e|\Phi\rangle
\end{align*}
and equations of motions
\begin{align*}
{\color{blue}M_I\ddot{\bf R}_I} = \dfrac{d}{dt}\dfrac{\partial\mathcal{L}_\mathrm{BOMD}}{\partial\dot{\bf R}_I} = \dfrac{\partial\mathcal{L}_\mathrm{BOMD}}{\partial{\bf R}_I} = {\color{blue}-\nabla_I\min_{\Phi_0}\langle\Phi|H_e|\Phi\rangle}
\end{align*}
\end{itemize}
\begin{colorblock}{white}{blue!30!black}{black}{blue!15!white}{\textbf{Extended Lagrangian Molecular Dynamics} (ELMD)}
Extend the Lagrangian by adding kinetic energy of fictitious particles and obtain their equation of motions from Euler-Lagrange equations.
\end{colorblock}
\vspace{-0.3cm}
\begin{columns}
\column{6cm}
\centering{
\begin{colorblock}{white}{blue!30!black}{black}{green!25!white}{}
\centering{
Molecular Orbitals: $\{\phi_i\}$ \\
Density Matrix: $\displaystyle{{P}_{\mu\nu}=\sum_{i}c^\ast_{\mu i}c_{\nu i}}$}
\end{colorblock}
}
\end{columns}
\end{frame}

\begin{frame}
\begin{colorblock}{white}{green!30!black}{black}{green!15!white}{\textbf{Car-Parrinello Molecular Dynamics} (CPMD)}
{\color{DarkBlue}CPMD and NWCHEM}\let\thefootnote\relax\footnotetext{\tiny R. Car and M. Parrinello, Phys. Rev. Lett. 55 (22), 2471 (1985)}
\vspace{-0.1cm}
\begin{align*}
\mathcal{L}_\mathrm{CPMD} = \sum_I\dfrac{1}{2}M_I\dot{\bf R}^2_I + \sum_i\dfrac{1}{2}\mu_i\langle\dot{\phi_i}|\dot{\phi_i}\rangle - \langle\Phi_0|H_e|\Phi_0\rangle + \mathrm{constraints}
\end{align*}
\end{colorblock}
 \begin{colorblock}{white}{green!30!black}{black}{green!15!white}{\textbf{Atom centered Density Matrix Propagation} (ADMP)}
{\color{DarkBlue}Gaussian 03/09}\let\thefootnote\relax\footnotetext{\tiny H. B. Schlegel, J. M. Millam, S. S. Iyengar, G. A. Voth, A. D. Daniels, G. E. Scuseria, M. J. Frisch, J. Chem. Phys. 114, 9758 (2001)}
\vspace{-0.1cm}
\begin{align*}
\mathcal{L}_\mathrm{ADMP} = \dfrac{1}{2}\mathrm{Tr}({\bf V}^T{\bf M}{\bf V}) + \dfrac{1}{2}\mu\mathrm{Tr}(\dot{\bf P}\dot{\bf P}) - E({\bf R},{\bf P}) - \mathrm{Tr}[{\boldsymbol\Lambda}({\bf PP}-{\bf P})]
\end{align*}
\end{colorblock}
\begin{colorblock}{white}{red!30!black}{black}{red!15!white}{\textbf{curvy-steps ELMD} (csELMD)}
{\color{purple}Q-Chem}\let\thefootnote\relax\footnotetext{\tiny J.M. Herbert and M. Head-Gordon, J. Chem. Phys. 121, 11542 (2004)}
\vspace{-0.1cm}
\begin{align*}
\mathcal{L}_\mathrm{csELMD} = \sum_I\dfrac{1}{2}M_I\dot{\bf R}^2_I + \dfrac{1}{2}\mu\sum_{i<j}\dot{\Delta}_{ij} - E({\bf R},{\bf P}) ;\hspace{0.2cm} {\bf P}(\lambda) = e^{\lambda\Delta}{\bf P}(0)e^{-\lambda\Delta}
\end{align*}
\end{colorblock}
\end{frame}


\beamertemplateshadingbackground{green!10!white}{green!10!white} % New background
\begin{frame}
\frametitle{\small Molecular Dynamics: Methods and Programs}
\begin{columns}
\column{12cm}
\begin{itemize}
\item Electronic energy obtained from 
\begin{itemize}
\item {Molecular Mechanics $\Rightarrow$ Classical Molecular Dynamics}
\begin{enumerate}
{\footnotesize
\item LAMMPS
\item NAMD
\item Amber
\item Gromacs
}
\end{enumerate}
\item {Ab-Initio Methods $\Rightarrow$ Quantum or Ab-Initio Molecular Dynamics}
\begin{enumerate}
{\footnotesize
\item Born-Oppenheimer Molecular Dynamics: Gaussian, GAMESS
\item Extended Lagrangian Molecular Dynamics: CPMD, Gaussian (ADMP), NWCHEM(CPMD), {\color{red}VASP}, {\color{red}QChem (curvy-steps ELMD)}
\item Time Dependent Hartree-Fock and Time Dependent Density Functional Theory: Gaussian, GAMESS, NWCHEM, {\color{red}QChem}
\item {\color{red}Multiconfiguration Time Dependent Hartree(-Fock), MCTDH(F)}
\item {\color{red}Non-Adiabatic and  Ehrenfest Molecular Dynamics, Multiple Spawning, Trajectory Surface Hopping}
\item {\color{red}Quantum Nuclei: QWAIMD(Gaussian), NEO(GAMESS)}
}
\end{enumerate}
\end{itemize}
\end{itemize}
\end{columns}
\end{frame}
\beamertemplateshadingbackground{blue!5}{blue!5} % back to original background
\begin{frame}
%\frametitle{\small Classical Molecular Dynamics}
\begin{colorblock}{white}{tigerspurple!90!white}{black}{tigersgold!85!black}{\bf Classical Molecular Dynamics}
\begin{itemize}
\item Advantages
\begin{enumerate}
{\footnotesize
\item Large Biological Systems
\item Long time dynamics
}
\end{enumerate}
\item Disadvantages
\begin{enumerate}
{\footnotesize
\item Cannot describe Quantum Nuclear Effects
}
\end{enumerate}
\end{itemize}
\end{colorblock}
\begin{colorblock}{white}{tigerspurple!90!white}{black}{tigersgold!85!black}{\bf \textit{Ab Initio} and Quantum Dynamics}
\begin{itemize}
\item Advantages
\begin{enumerate}
{\footnotesize
\item Quantum Nuclear Effects
}
\end{enumerate}
\item Disadvantages
\begin{enumerate}
{\footnotesize
\item $\sim$ 100 atoms
\item Full Quantum Dynamics ie treating nuclei quantum mechanically: less than 10 atoms
\item Picosecond dynamics at best
}
\end{enumerate}

\end{itemize}
\end{colorblock}
%{\footnotesize
%\begin{enumerate}
%\end{enumerate}
%}
\end{frame}


\section{Computational Chemistry Programs}
\begin{frame}
\frametitle{\small Software Installed on LONI \& LSU HPC Systems}
\scriptsize{
\begin{columns}
\column{12cm}
\vspace{-1.25cm}
\begin{center}
\begin{tikzpicture}
\node (tbl) {
\begin{tabularx}{\textwidth}{ccccccccc}
\arrayrulecolor{tigersgold}
\textcolor{white}{\textbf{Software} }& \textcolor{white}{\textbf{QB}} &\textcolor{white}{\textbf {Eric}} & \textcolor{white}{\textbf{Louie}} & \textcolor{white}{\textbf{Oliver}} & \textcolor{white}{\textbf{Painter}} & \textcolor{white}{\textbf{Poseidon}} & \textcolor{orange}{\textbf{Philip}} & \textcolor{orange}{\textbf{Tezpur}}\\
Amber\rule{0pt}{3.5ex} & \checkmark & \checkmark & \checkmark & \checkmark & \checkmark & \checkmark & \checkmark & \checkmark \\ 
Desmond & \checkmark &  &  &  &  &  &  &  \\
DL\_Poly & \checkmark & \checkmark & \checkmark & \checkmark & \checkmark & \checkmark & \checkmark & \checkmark \\
Gromacs & \checkmark & \checkmark & \checkmark & \checkmark & \checkmark & \checkmark & \checkmark & \checkmark \\
LAMMPS & \checkmark & \checkmark & \checkmark & \checkmark & \checkmark & \checkmark & \checkmark & \checkmark \\
NAMD & \checkmark & \checkmark & \checkmark & \checkmark & \checkmark & \checkmark &  \checkmark & \checkmark \\
%OpenEye & \checkmark & \checkmark & \checkmark & \checkmark & \checkmark & \checkmark & \checkmark & \checkmark \\
CPMD & \checkmark & \checkmark & \checkmark & \checkmark & \checkmark & \checkmark &  & \checkmark \\
GAMESS & \checkmark & \checkmark & \checkmark & \checkmark & \checkmark & \checkmark & \checkmark & \checkmark \\
Gaussian &  & \checkmark & \checkmark & \checkmark & \checkmark &  & \checkmark & \checkmark \\
NWCHEM & \checkmark & \checkmark & \checkmark & \checkmark & \checkmark & \checkmark &  & \checkmark \\
%Piny\_MD & \checkmark & \checkmark & \checkmark & \checkmark & \checkmark & \checkmark & \checkmark & \checkmark  \\
%ACES & \checkmark &  &  &  &  &  & {\color{red}\checkmark}  &  \\
[1.0ex]
\end{tabularx}};
\begin{pgfonlayer}{background}
\draw[rounded corners,top color=blue!30!black,bottom color=blue!10!white,
    draw=tigerspurple!30] ($(tbl.north west)+(0.14,0)$)
    rectangle ($(tbl.north east)-(0.13,0.9)$);
\draw[rounded corners,top color=gray!50,bottom color=gray!50,draw=black!15]
    ($(tbl.north east)-(0.13,0.6)$)
    rectangle ($(tbl.south west)+(0.13,0.2)$);
\end{pgfonlayer}
\end{tikzpicture}
\end{center}
\vspace{-1.2cm}
\begin{center}
\begin{tikzpicture}
\node (tbl) {
\begin{tabularx}{\textwidth}{cccccccc}
\arrayrulecolor{tigersgold}
\textcolor{white}{\textbf{Software}} & \textcolor{white}{\textbf{Bluedawg}} & \textcolor{white}{\textbf {Ducky}} & \textcolor{white}{\textbf{Lacumba}} & \textcolor{white}{\textbf{Neptune}} & \textcolor{white}{\textbf{Zeke}} & \textcolor{orange}{\textbf{Pelican}} & \textcolor{orange}{\textbf{Pandora}}\\
Amber\rule{0pt}{3.5ex} &  & \checkmark & \checkmark &  &  & \checkmark & \checkmark \\ 
Gromacs & \checkmark & \checkmark & \checkmark & \checkmark & \checkmark & \checkmark & \checkmark\\
LAMMPS & \checkmark & \checkmark & \checkmark & \checkmark & \checkmark & \checkmark & \checkmark \\
NAMD & \checkmark & \checkmark & \checkmark & \checkmark & \checkmark &  & \checkmark\\
CPMD & \checkmark & \checkmark & \checkmark & \checkmark & \checkmark &  & \checkmark\\
GAMESS & & & & & & & \checkmark \\
Gaussian & \checkmark & \checkmark & \checkmark &  & \checkmark & \checkmark & \checkmark\\
NWCHEM & \checkmark & \checkmark & \checkmark & \checkmark & \checkmark & \checkmark & \checkmark \\
%Piny\_MD & \checkmark & \checkmark & \checkmark & \checkmark & \checkmark & \checkmark  & \checkmark\\
[1.0ex]
\end{tabularx}};
\begin{pgfonlayer}{background}
\draw[rounded corners,top color=blue!40!black,bottom color=blue!10!white,
    draw=tigerspurple!30] ($(tbl.north west)+(0.14,0)$)
    rectangle ($(tbl.north east)-(0.13,0.9)$);
\draw[rounded corners,top color=gray!50,bottom color=gray!50,draw=red!15]
    ($(tbl.north east)-(0.13,0.6)$)
    rectangle ($(tbl.south west)+(0.13,0.2)$);
\end{pgfonlayer}
\end{tikzpicture}
\end{center}
\end{columns}
}
\end{frame}

\begin{frame}
\frametitle{\small Computational Chemistry Programs}
\begin{itemize}
 \item {\color{black}Other Software: Q-Chem, SIESTA ,CHARMM, VASP, Quantum Expresso, Octopus}
 \item {\color{tigersblue}{\bf \url{http://en.wikipedia.org/wiki/Quantum\_chemistry\_computer\_programs}}}
 \item {\color{tigersblue}{\bf \url{http://www.ccl.net/chemistry/links/software/index.shtml}}}
 \item {\color{tigersblue}{\bf \url{http://www.redbrick.dcu.ie/~noel/linux4chemistry/}}}
\end{itemize}
\end{frame}

\begin{frame}
%\footnotesize{
\vspace{0.5cm}
\begin{colorblock}{white}{green!30!black}{black}{green!15!white}{\bf Molecular Dynamics Calculations}
\begin{itemize}
\item CPMD:
 \begin{itemize}
  \item CP: Car-Parrinello Molecular Dynamics
  \item BO: Born-Oppenheimer Molecular Dynamics
 \end{itemize}
\item Gaussian:
 \begin{itemize}
% {\scriptsize
 \item BOMD: Born-Oppenheimer Molecular Dynamics
 \item ADMP: Atom centered Density Matrix Propagation and ground state BOMD
% }
 \end{itemize}
\item GAMESS:
  \begin{itemize}
  \item DRC: Direct Dynamics, a classical trajectory method based on "on-the-fly" ab-initio or semi-empirical potential energy surfaces
  \end{itemize}
\item NWCHEM:
  \begin{itemize}
  \item Car-Parrinello: Car Parrinello Molecular Dynamics (CPMD)
%   \item dynamics: Perform classical molecular dynamics
  \item DIRDYVTST: Direct Dynamics Calculations using POLYRATE with electronic structure from NWCHEM
  \end{itemize}
\end{itemize}
\end{colorblock}
%}
\end{frame}


%\section{Example Jobs}
\begin{frame}
%\frametitle{\small Worked out Examples On LONI Linux Systems}
\begin{columns}
\column{1.16\textwidth}
\vspace{-0.9cm}
\begin{alertblock}{On LONI/LSU HPC Linux Systems}
\begin{center}
\includegraphics[width=0.74\textwidth,clip=true]{workshop}
\end{center}
\end{alertblock}
\end{columns}
\end{frame}

\section{LAMMPS on LONI Systems}
\begin{frame}
\frametitle{\small Using LAMMPS on LONI systems}
\begin{itemize}
\item LAMMPS stands for Large-scale Atomic/Molecular Parallel Simulator.
\item LAMMPS is a classical molecular dynamics code that models an ensemble of particles in a liquid, solid, or gaseous state designed to run efficiently on parallel computers.
\item It can model atomic, polymeric, biological, metallic, granular, and coarse-grained systems using a variety of force fields and boundary conditions.
\item LAMMPS can model systems with only a few particles up to millions or billions.
\item LAMMPS is designed to be easy to modify or extend with new capabilities, such as new force fields, atom types, boundary conditions, or diagnostics.
\item LAMMPS runs efficiently on single-processor desktop or laptop machines, but is designed for parallel computers.
\item It is an open-source code, distributed freely under the terms of the GNU Public License (GPL).
\end{itemize}
\end{frame}

\begin{frame}[allowframebreaks]
\begin{itemize}
{\footnotesize
\item LAMMPS doesn't
\begin{enumerate}
{\footnotesize
\item Build molecular systems
\item Assign force-dield coefficients auto-magically
\item Compute lots of diagnostics on-the-fly
\item Visualize your output
}
\end{enumerate}
\item LAMMPS version "4 May 2011" is installed on all LONI Dell Linux Clusters and LSU HPC machines, Tezpur (Linux) and Pandora (AIX).
\item Add the appropriate soft keys to your .soft file
\begin{enumerate}
{\footnotesize
\item Linux: +lammps-4May11-intel-11.1-mvapich-1.1
\item Pandora: +lammps-4May11
}
\end{enumerate}
\item Running LAMMPS
\begin{enumerate}
\begin{columns}
\column{11cm}
\vspace{-0.5cm}
{\footnotesize
\item[$\vardiamond$] Linux: mpirun -np \$\{no of procs\} \{LAMMPS directory\}/bin/lmp\_linux < inputfile
\item[$\vardiamond$] Pandora: poe \{LAMMPS directory\}/bin/lmp\_power7 < inputfile
}
\end{columns}
\end{enumerate}
\item Command line options
\begin{itemize}
{\footnotesize
\item[$\vardiamond$] -in inputfile: specify input file
\item[$\vardiamond$] -log logfile: specify log file
\item[$\vardiamond$] -partition MxN L: Run on (MxN)+L processors with M partitions on N processors each and 1 partition with L processors
%run LAMMPS in multi partition mode.\\
%\item[]Usage: -partition MxN L : 
\item[$\vardiamond$] -screen file: Specify a file to write screen information
}
\end{itemize}
}
\end{itemize}
\end{frame}

\begin{frame}
\frametitle{\footnotesize LAMMPS Input}
\begin{itemize}
{\scriptsize
\item Reads an input script in ASCII format one line at a time.
\item Input script consists of 4 parts
\begin{enumerate}
{\scriptsize
\item Initialization: Set parameters that need to be defined before atoms are created or read-in from a file.
\item[] \texttt{units, dimension, newton, processors, boundary, atom\_style, atom\_modify}
\item Atom definition: \texttt{read\_data, read\_restart, lattice, region, create\_box, create\_atoms, replicate}
\item Settings: Once atoms and molecular topology are defined, a variety of settings can be specified: force field coefficients, simulation parameters, output options, etc. 
\item[] \texttt{pair\_coeff, bond\_coeff, angle\_coeff, dihedral\_coeff, improper\_coeff, kspace\_style, dielectric, special\_bonds, neighbor, neigh\_modify, group, timestep, reset\_timestep, run\_style, min\_style, min\_modify, fix, compute, compute\_modify, variable}
\item Run a simulation: A MD is run using the run command. Energy minimization (molecular statics) is performed using the minimize command. A parallel tempering (replica-exchange) simulation can be run using the temper command.
}
\end{enumerate}
\item \url{http://lammps.sandia.gov/doc/Section_commands.html}
}
\end{itemize}
\end{frame}

\begin{frame}
\begin{itemize}
{\scriptsize
\only<1>{
\item "dump" command outputs snapshots of atom properties
\begin{columns}
\column{11cm}
\vspace{-0.5cm}
\begin{align*}
\begin{array}{cr}
\mathrm{default\,format\,is\,simple:} & {\rm id,type,x,y,z}\\
{\rm other\,supported\,formats:} &{\rm XYZ,DCD,XTC}
\end{array}
\end{align*}
\column{0.05cm}
\end{columns}
\item Visualization using VMD
\item[]vmd -lammpstrj dumpfile
}
\only<2->{
\item The LAMMPS distribution includes an examples sub-directory with several sample problems. 
}
\only<2>{
\item crack: crack propagation in a 2d solid
\begin{center}
\includemovie[text={\includegraphics[height=4cm,keepaspectratio=true]{crack-init}}]{}{}{crack.mpg}
\end{center}
}
\only<3>{
\item friction: frictional contact of spherical asperities between 2d surfaces
\begin{center}
\includemovie[text={\includegraphics[height=4cm,keepaspectratio=true]{friction-init}}]{}{}{friction.mpg}
\end{center}
}
\only<4>{
\item micelle: self-assembly of small lipid-like molecules into 2d bilayers
\begin{center}
\includemovie[text={\includegraphics[height=4cm,keepaspectratio=true]{micelle-init}}]{}{}{micelle.mpg}
\end{center}
}
}
\end{itemize}
\end{frame}

%\section{Benchmarks}
\begin{frame}
\frametitle{Benchmarks}
\begin{itemize}
\only<1>{
\item LAMMPS on LONI \& LSU HPC machines
\begin{enumerate}
{\footnotesize
\item[$\vardiamond$] Atomic fluid with Lennard-Jones Potential
\item[$\vardiamond$] 32,000 atoms for 100,000 steps
\item[$\vardiamond$] force cutoff = 2.5$\sigma$, neighbor skin = 0.3$\sigma$, neighbor/atom = 55
\item[$\vardiamond$] NVE time integration
}
\end{enumerate}
\begin{center}
\includegraphics[width=0.75\textwidth,clip=true]{lammps}
\end{center}
}
\only<2>{
\item NAMD on LONI \& LSU HPC machines
\begin{enumerate}
{\footnotesize
\item[$\vardiamond$] Apoa1 Benchmark
\item[$\vardiamond$] 92,224 atoms, 12\AA cutoff + PME every 4 steps, periodic
}
\end{enumerate}
\begin{center}
\includegraphics[width=0.8\textwidth,clip=true]{namd}
\end{center}
}
\only<3>{
\item Gromacs on LONI \& LSU HPC machines
\begin{enumerate}
{\footnotesize
\item[$\vardiamond$] 159 residue protein dihydrofolate reductase (dhfr) with implicit solvent
\item[$\vardiamond$] 2489 atoms, 1nm cutoff
\begin{center}
\includegraphics[width=0.8\textwidth,clip=true]{gromacs}
\end{center}
}
\end{enumerate}
}
\end{itemize}
\end{frame}

\section{Molecular Dynamics on GPU}
\begin{frame}
\frametitle{\small Molecular Dynamics on GPU}
\begin{itemize}
\item Using a graphics processing unit (GPU) for molecular simulations.
\item MD programs capable of GPU computing
\begin{enumerate}
{\footnotesize
\item LAMMPS
\item NAMD
\item GROMACS
\item Amber/PMEMD
}
\end{enumerate}
\item Not all features of the MD programs are capable of GPU computing.
\end{itemize}
\end{frame}

\begin{frame}
\frametitle{\small Molecular Dynamics on GPU}
\begin{itemize}
\item Coming Soon: 2  Philip nodes with 3  GPU's each.
\begin{itemize}
{\footnotesize
\item Intel Xeon X5650
\begin{enumerate}
{\footnotesize
\item[$\vardiamond$] 2.66GHz Dual Hexa-core with hyperthreading: 12 cores, 24 threads
\item[$\vardiamond$] Memory : 48GB
\item[$\vardiamond$] L3 Cache: 12MB
}
\end{enumerate}
\item Tesla M2070
\begin{enumerate}
{\footnotesize
\item[$\vardiamond$] 448 CUDA cores (14 Multiprocessor $\times$ 32 CUDA Cores/MP )
\item[$\vardiamond$] GPU Clock Speed: 1.15GHz
\item[$\vardiamond$] Total Memory: 5.25GB
}
\end{enumerate}
}
\end{itemize}
\item Compilers: CUDA 4.0, PGI Accelerator
\item Infiniband Interconnect between the two nodes.
\item For optimum scaling, ratio for CPU core:GPU should be 1. 
\item LAMMPS and NAMD can bind  GPU to multiple cores but performance degrades.
\end{itemize}
\end{frame}

\begin{frame}
\frametitle{\small LAMMPS on GPU}
\begin{itemize}
\item Need to use \texttt{newton off}
\item \texttt{fix\_gpu} should be used in order to initialize and configure the GPUs for use
\item[] \texttt{fix ID group-ID gpu mode first last split}
\item[] \texttt{mode} can be force or force/neigh
\item currently limited to a few \texttt{pair\_styles} and PPPM
\begin{center}
\includegraphics[width=0.7\textwidth,clip=true]{lammps-gpu}
\end{center}
\end{itemize}
\end{frame}

\begin{frame}
\frametitle{\small NAMD on GPU}
\begin{itemize}
\item command line options to add
\item[] \texttt{+idlepoll +devices 0,1}
\item[] If \texttt{+devices} options is not added then all available GPU's will be used
\item[] Each CPU will be assigned a GPU cyclically
\item[] i.e. \texttt{mpirun -np 4 namd2 +idlepoll +devices 0,1} will bind CPU 0 and 2 to GPU device 0 and CPU 1 to GPU device 1
\begin{center}
\includegraphics[width=0.7\textwidth,clip=true]{namd-gpu}
\end{center}
\end{itemize}
\end{frame}

\begin{frame}
\frametitle{\small Gromacs on GPU}
\begin{itemize}
\item Use mdrun compiled with cuda: \texttt{mdrun-gpu}
\item \texttt{mdrun-gpu} can only run on 1 GPU, multiple GPUs functionality not available yet
\item Usage: same as serial or parallel mdrun
\begin{center}
\includegraphics[width=0.8\textwidth,clip=true]{gromacs-gpu}
\end{center}
\end{itemize}
\end{frame}


\begin{frame}<0>
\frametitle{\small Related HPC Tutorials}
%\begin{itemize}
%\item Fall Semester
\begin{colorblock}{white}{green!30!black}{black}{green!15!white}{HPC Training in Fall}
\begin{itemize}
\item{ Introduction to Electronic Structure Calculation in Quantum Chemistry using Gaussian, GAMESS and NWCHEM.}
\end{itemize}
\end{colorblock}
%\item Spring Semester
%\begin{alertblock}{MD: Programming to Production}
%April 6$^{\rm th}$
%\end{alertblock}
%\begin{alertblock}{\small Introduction to Computational Chemistry: Molecular Dynamics}
%April 27$^{\rm th}$
%\end{alertblock}
%\item LONI HPC workshops
\begin{colorblock}{white}{green!30!black}{black}{green!15!white}{LONI HPC Workshops}
\begin{itemize}
\item  {Introduction to Computational Chemistry}:
Condensed tutorial on Electronic Structure and Molecular Dynamics.
\end{itemize}
\end{colorblock}
%\item LONI/HPC Moodle
\begin{colorblock}{white}{green!30!black}{black}{green!15!white}{LONI HPC Moodle}
\begin{enumerate}
{\footnotesize
\item HPC 108: How to use CPMD
\item HPC 109: Intro to Gaussian
}
\end{enumerate}
\end{colorblock}
%\end{itemize}
\end{frame}

\begin{frame}[allowframebreaks]
%\frametitle{\small Useful Links/Further Reading}
\begin{block}{Useful Links}
\begin{itemize}
{\footnotesize\color{Blue}
\item{\color{tigerspurple}Amber:}\url{http://ambermd.org}
\item{\color{tigerspurple}Desmond:}\url{http://www.deshawresearch.com/resources_desmond.html}
\item{\color{tigerspurple}DL\_POLY:}\url{http://www.cse.scitech.ac.uk/ccg/software/DL_POLY}
\item{\color{tigerspurple}Gromacs:}\url{http://www.gromacs.org}
\item{\color{tigerspurple}LAMMPS:}\url{http://lammps.sandia.gov}
\item{\color{tigerspurple}NAMD:}\url{http://www.ks.uiuc.edu/Research/namd}
\item {\color{tigerspurple}CPMD:} \url{http://www.cpmd.org}
\item {\color{tigerspurple}GAMESS:} \url{http://www.msg.chem.iastate.edu/gamess}
\item {\color{tigerspurple}Gaussian:} \url{http://www.gaussian.com}
\item {\color{tigerspurple}NWCHEM:} \url{http://www.nwchem-sw.org}
\item{\color{tigerspurple}PINY\_MD:}\url{http://homepages.nyu.edu/~mt33/PINY_MD/PINY.html}
\item {\color{tigerspurple}Basis Set:} \url{https://bse.pnl.gov/bse/portal}
}
\end{itemize}
\end{block}
\begin{block}{\small Further Reading}
\begin{itemize}
{\scriptsize
\item A Molecular Dynamics Primer by Furio Ercolessi {\color{Blue}\url{http://www.fisica.uniud.it/~ercolessi/md}}
\item Molecular Modeling - Principles and Applications, A. R. Leach.
\item Computer Simulation of Liquids, M. P. Allen and D. J. Tildesley.
\item Mark Tuckerman's Notes at NYU: {\color{Blue}\url{http://www.nyu.edu/classes/tuckerman/quant.mech/index.html}}
\item Ab Initio Molecular Dynamics: Theory and Implementation, D. Marx and J. Hutter{\color{Blue}\url{http://www.theochem.ruhr-uni-bochum.de/research/marx/marx.pdf}}
\item Quantum Dynamics with Trajectories: Introduction to Quantum Hydrodynamics, R. E. Wyatt.
\item Quantum Dynamics of Complex Molecular Systems, Editors: D. A. Micha and I. Burghardt
\item Energy Transfer Dynamics in Biomaterial Systems. Editors: I. Burghardt, V. May, D. A. Micha and E. R. Bittner .
%\item David Sherill's Notes at Ga Tech: {\color{Blue}\url{http://vergil.chemistry.gatech.edu/notes/index.html}}
%\item Modern Quantum Chemistry: Introduction to Advanced Electronic Structure Theory, A. Szabo and N. Ostlund.
%\item Introduction to Computational Chemistry, F. Jensen.
%\item Essentials of Computational Chemistry - Theories and Models, C. J. Cramer.
%\item Exploring Chemistry with Electronic Structure Methods, J. B. Foresman and A. Frisch
%\item[$\vardiamond$]Modern Electronic Structure Theory, T. Helgaker, P. Jorgensen and J. Olsen (Highly advanced text, second quantization approach to electronic structure theory)
}
\end{itemize}
\end{block}
\end{frame}

\end{document}

