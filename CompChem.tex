\documentclass[slidestop,mathserif,compress,xcolor=svgnames]{beamer} 
\mode<presentation>
{  
  \setbeamertemplate{background canvas}[vertical shading][bottom=blue!5,top=blue!5]
  \setbeamertemplate{navigation symbols}{}%{\insertsectionnavigationsymbol}
   \usecolortheme{tigers}
   \useinnertheme{rounded}
  \usefonttheme[onlysmall]{structureitalicserif}
\usefoottemplate{\vbox{%
\tinycolouredline{structure}%
{\color{tigersgold}\textbf{\insertshortinstitute\hfill\insertshortdate}}%
}}
}

\usepackage{pgf,pgfarrows,pgfnodes,pgfautomata,pgfheaps,pgfshade}
\usepackage{amsmath,amssymb,amsfonts,subfigure}
\usepackage{tabularx}
\usepackage{booktabs}
\usepackage{colortbl}
\usepackage{tikz}
\usetikzlibrary{calc}
\pgfdeclarelayer{background}
\pgfdeclarelayer{foreground}
\pgfsetlayers{background,main,foreground}
\usepackage[latin1]{inputenc}
\usepackage{colortbl}
\usepackage[english]{babel}
\usepackage{hyperref}
\usepackage{movie15}
\hypersetup{
  pdftitle={Introduction to Computational Chemistry},
  pdfauthor={Alexander B. Pacheco, User Services Consultant, Louisiana State University}
}
%\usepackage{movie15}
\usepackage{times}

\setbeamercovered{dynamic}
\beamersetaveragebackground{DarkBlue!2}
\beamertemplateballitem

\usepackage[english]{babel}
\usepackage[latin1]{inputenc}
\usepackage{times}
\usepackage{amsmath}
\usepackage[T1]{fontenc}
\usepackage{graphicx}
\definecolor{DarkGreen}{rgb}{0.0,0.3,0.0}
\definecolor{Blue}{rgb}{0.0,0.0,0.8} 
\definecolor{dodgerblue}{rgb}{0.1,0.1,1.0}
\definecolor{indigo}{rgb}{0.41,0.1,0.0}
\definecolor{seagreen}{rgb}{0.1,1.0,0.1}
\DeclareSymbolFont{extraup}{U}{zavm}{m}{n}
\DeclareMathSymbol{\vardiamond}{\mathalpha}{extraup}{87}
\newcommand*\up{\textcolor{green}{%
  \ensuremath{\blacktriangle}}}
\newcommand*\down{\textcolor{red}{%
  \ensuremath{\blacktriangledown}}}
\newcommand*\const{\textcolor{darkgray}%
  {\textbf{--}}}

\setbeamercolor{uppercol}{fg=white,bg=red!30!black}%
\setbeamercolor{lowercol}{fg=black,bg=red!15!white}%
\newenvironment{colorblock}[4]
{
\setbeamercolor{upperblock}{fg=#1,bg=#2}
\setbeamercolor{lowerblock}{fg=#3,bg=#4}
\begin{beamerboxesrounded}[upper=upperblock,lower=lowerblock,shadow=true]}
{\end{beamerboxesrounded}}

\newenvironment{colorblock2}[2]
{
\begin{beamerboxesrounded}[upper=#1,lower=#2,shadow=true]}
{\end{beamerboxesrounded}}

\title[Comp. Chem.]{Introduction to Computational Chemistry}


\author[Alex Pacheco]{\large{Alexander~B.~Pacheco}}
       
\institute[High Performance Computing @ Louisiana State University - http://www.hpc.lsu.edu] {\inst{}\footnotesize{User Services Consultant\\LSU HPC \& LONI\\sys-help@loni.org}}

\date[\hfill{March 25, 2011\hspace{2cm}\insertframenumber/\inserttotalframenumber}]{\scriptsize{LONI Worshop Series\\Tulane University, New Orleans\\March 25, 2011}}
     
\subject{Talks}
% This is only inserted into the PDF information catalog. Can be left
% out. 




% If you have a file called "university-logo-filename.xxx", where xxx
% is a graphic format that can be processed by latex or pdflatex,
% resp., then you can add a logo as follows:

\pgfdeclareimage[height=1.0cm]{loni-logo}{LONI}
\pgfdeclareimage[height=0.5cm]{gold-logo}{GOLD_LSU}
\pgfdeclareimage[height=0.8cm]{purple-logo}{PURPLELSU}
\pgfdeclareimage[height=1.5cm]{lsuccttower-logo}{FULLCOLORTOWER_VERT}%{qtp}
\pgfdeclareimage[height=0.5cm]{lsutower-logo}{ProcessHorizontal}
% \logo{\pgfuseimage{purple-logo}}

% put MSRI logo in bottom left
\setbeamertemplate{sidebar left}{%
   \vfill%
   \rlap{\hskip0.02cm%
         %
         {\pgfuseimage{loni-logo}}}%
   \vskip-7pt%
   \llap{\usebeamertemplate***{navigation symbols}\hskip0.1cm}%
   \vskip2pt%
}
\setbeamertemplate{sidebar right}{%
   \vfill%
   \rlap{\hskip-2.1cm%
         %
         {\pgfuseimage{purple-logo}}}%
   \vskip-6pt%
   \llap{\usebeamertemplate***{navigation symbols}\hskip0.1cm}%
   \vskip2pt%
}

% Delete this, if you do not want the table of contents to pop up at
% the beginning of each subsection:
% \AtBeginSection[]
% {
%   \begin{frame}<beamer>
%    \frametitle{\small{Outline}}
%     \small
%     \tableofcontents[currentsection,currentsubsection]
%   \end{frame}
% }

\begin{document}

\frame{\titlepage}

\normalsize
\begin{frame}[label=toc,squeeze]
  \footnotesize
  \frametitle{\small{Outline}}
  \tableofcontents
  \tableofcontents[part=1]
  \tableofcontents[part=2]
  \tableofcontents[part=3]
  \tableofcontents[part=4]
  \tableofcontents[part=5]
\end{frame}

\normalsize

%\part{Introduction}
\section{Introduction}
\begin{frame}
\frametitle{\small What is Computational Chemistry}
\begin{itemize}
\item {\bf Computational Chemistry} is a branch of chemistry that uses computer science to assist in solving chemical problems.
\item Incorporates the results of theoretical chemistry into efficient computer programs.
\item Application to single molecule, groups of molecules, liquids or solids.
\item Calculates the structure and properties of interest.
\item Computational Chemistry Methods range from
 \begin{enumerate}
  \item Highly accurate ({\it Ab-initio},DFT) feasible for small systems
  \item Less accurate (semi-empirical)
  \item Very Approximate (Molecular Mechanics) large systems
 \end{enumerate}
\end{itemize}
\end{frame}

\begin{frame}
%\frametitle{\small What is Computational Chemistry}
\begin{columns}
\column{11cm}
\begin{block}{Theoretical Chemistry: broadly can be divided into two main categories}
\begin{enumerate}
\item Static Methods {\Large$\Rightarrow$} {\color{blue}Time-Independent Schr\"{o}dinger Equation}
\begin{enumerate}
\item[$\vardiamond$] Quantum Chemical/\emph{Ab Initio} /Electronic Structure Methods
\item[$\vardiamond$] Molecular Mechanics
\end{enumerate}
\item Dynamical Methods {\Large$\Rightarrow$} {\color{blue}Time-Dependent Schr\"{o}dinger Equation}
\begin{enumerate}
\item[$\vardiamond$] Classical Molecular Dynamics
\item[$\vardiamond$] Semi-classical and \textit{Ab-Initio} Molecular Dynamics
\end{enumerate}
\end{enumerate}
\end{block}
\end{columns}

\end{frame}


%\part{Ab initio Methods}
\section{Ab Initio Methods}
\begin{frame}
\frametitle{\small Ab Initio Methods}
\footnotesize{
\begin{block}{}
\begin{itemize}
\item \emph{Ab Initio} meaning "from first principles" methods solve the Schr\"{o}dinger equation and does not rely on empirical or experimental data. 
\item Begining with fundamental and physical properties, calculate how electrons and nuclei interact.
\item The Schr\"{o}dinger equation can be solved exactly only for a few systems
\begin{itemize}
\footnotesize{
 \item[$\vardiamond$] Particle in a Box
 \item[$\vardiamond$] Rigid Rotor
 \item[$\vardiamond$] Harmonic Oscillator
 \item[$\vardiamond$] Hydrogen Atom
}
\end{itemize}
\item For complex systems, \emph{Ab Initio} methods make assumptions to obtain approximate solutions to the  Schr\"{o}dinger equations and solve it numerically.
\item "Computational Cost" of calculations increases with the accuracy of the calculation and size of the system.
\end{itemize}
\end{block}
}
\end{frame}

\begin{frame}
%\frametitle{\small Ab Initio Methods}
\footnotesize{
\vspace{-0.1cm}
\begin{block}{What can we predict with \emph{Ab Initio} methods?}
\begin{itemize}
\footnotesize{
 \item Molecular Geometry: Equilibrium and Transition State
 \item Dipole and Quadrupole Moments and polarizabilities
 \item Thermochemical data like Free Energy, Energy of reaction.
 \item Potential Energy surfaces, Barrier heights
 \item Reaction Rates and cross sections
 \item Ionization potentials (photoelectron and X-ray spectra) and Electron affinities
 \item Frank-Condon factors (transition probabilities, vibronic intensities)
 \item Vibrational Frequencies, IR and Raman Spectra and Intensities
 \item Rotational spectra
 \item NMR Spectra
 \item Electronic excitations and UV-VIS spectra
 \item Electron density maps and population analyses
 \item Thermodynamic quantities like partition function
}
\end{itemize}
\end{block}
}
\end{frame}


\begin{frame}
%\frametitle{\small Ab Initio Methods}
\footnotesize{
\begin{block}{Ab Initio Theory}
\begin{itemize}
\item {\color{blue}Born-Oppenheimer Approximation}: Nuclei are heavier than electrons and can be considered stationary with respect to electrons. Also know as "clamped nuclei" approximations and leads to idea of potential surface
\item {\color{blue}Slater Determinants}: Expand the many electron wave function in terms of Slater determinants.
\item {\color{blue}Basis Sets}: Represent Slater determinants by molecular orbitals, which are linear combination of atomic-like-orbital functions i.e. basis sets
\end{itemize}
\end{block}
}
\end{frame}

\begin{frame}
\frametitle{\small Born-Oppenheimer Approximation}
\footnotesize{
\begin{itemize}
\item Solve time-independent Schr\"{o}dinger equation
\begin{align*}
\hat{H}\Psi = E\Psi
\end{align*}
\item For many electron system:
\begin{align*}
\hat{H} = \underbrace{-\frac{\hbar^2}{2}\sum_\alpha\frac{\nabla^2_\alpha}{M_\alpha}}_{\hat{T}_n} - \underbrace{\frac{\hbar^2}{2m_e}\sum_i\nabla^2_i}_{\hat{T}_e} + \underbrace{\underbrace{\sum_{\alpha>\beta}\frac{e^2Z_{\alpha}Z_{\beta}}{4\pi\epsilon_0R_{\alpha\beta}}}_{\hat{V}_{nn}} - \underbrace{\sum_{\alpha,i}\frac{e^2Z_{\alpha}}{4\pi\epsilon_0R_{\alpha i}}}_{\hat{V}_{en}} + \underbrace{\sum_{i>j}\frac{e^2}{4\pi\epsilon_0r_{ij}}}_{\hat{V}_{ee}}}_{\hat{V}}
\end{align*}
\item The wave function $\Psi(R,r)$ of the many electron molecule is a function of nuclear ($R$) and electronic ($r$) coordinates.
\item Motion of nuclei and electrons are coupled.
\item However, since nuclei are much heavier than electrons, the nuclei appear fixed or stationary.
\end{itemize}
}
\end{frame}

\begin{frame}
%\frametitle{\small Born-Oppenheimer Approximation}
\footnotesize{
\begin{itemize}
\item Born-Oppenheimer Approximation: Separate electronic and nuclear motion:
\begin{align*}
\Psi(R,r) = \psi_e(r;R)\psi_n(R)
\end{align*}
\item Solve electronic part of Schr\"{o}dinger equation
\begin{align*}
\hat{H}_e\psi_e(r;R) = E_e\psi_e(r;R)
\end{align*}
\end{itemize}
\begin{columns}
\column{5cm}
\vspace{-0.5cm}
\begin{itemize}
\item 
BO approximation leads to the concept of potential energy surface
\begin{align*}
V(R) = E_e + V_{nn}
\end{align*}
\end{itemize}
\column{5cm}
\vspace{-0.5cm}
%\begin{block}{}
\includegraphics[width=6cm,keepaspectratio=true,clip=true]{HOH}
%\end{block}
\end{columns}
}
\end{frame}
\begin{frame}
\frametitle{\small Potential Energy Surfaces}
\footnotesize{
%\begin{columns}
%\column{0.5\textwidth}
\begin{itemize}
\item The potential energy surface (PES) is multi-dimensional ($3N-6$ for non-linear molecule and $3N-5$ for linear molecule)
\item The PES contains multiple minima and maxima.\let\thefootnote\relax\footnotetext{\tiny Picture taken from Bernard Schlegel's course slide at http://www.chem.wayne.edu/$\sim$hbs/chm6440/}
\item Geometry optimization search aims to find the global minimum of the potential surface.
\item Transition state or saddle point search aims to find the maximum of this potential surface, usually along the reaction coordinate of interest.
\end{itemize}
%\column{0.5\textwidth}
\begin{center}
\includegraphics[width=6cm,keepaspectratio=true,clip=true]{PES}
\end{center}
%\end{columns}
}
\end{frame}


\begin{frame}
\frametitle{\small Wavefunction Methods}
\footnotesize{
\begin{itemize}
\item The electronic Hamiltonian (in atomic units, $\hbar,m_e,4\pi\epsilon_0,e=1$) to be solved is 
\begin{align*}
\hat{H}_e = -\frac{1}{2}\sum_i\nabla^2_i - \sum_{\alpha,i}\frac{Z_\alpha}{R_{i\alpha}} + \sum_{i>j}\frac{1}{r_{ij}} + \sum_{\alpha>\beta}\frac{Z_\alpha Z_\beta}{R_{\alpha\beta}}
\end{align*}
\item Calculate electronic wave function and energy
\begin{align*}
E_e = \frac{\langle\psi_e\mid\hat{H}_e\mid\psi_e\rangle}{\langle\psi_e\mid\psi_e\rangle}
\end{align*}
\item The total electronic wave function is written as a Slater Determinant of the one electron functions, i.e. molecular orbitals, MO's
\begin{align*}
\psi_e = \frac{1}{\sqrt{N!}}\left| \begin{array}{cccc}
\phi_1(1) & \phi_2(1) & \cdots & \phi_N(1)\\
\phi_1(2) & \phi_2(2) & \cdots & \phi_N(2)\\
\cdots & \cdots & \cdots & \cdots \\
\phi_1(N) & \phi_2(N) & \cdots & \phi_N(N)\\
\end{array} \right|
\end{align*}
\end{itemize}
}
\end{frame}

\begin{frame}
%\frametitle{\small Wavefunction Methods}
\footnotesize{
\begin{itemize}
\item MO's are written as a linear combination of one electron atomic functions or atomic orbitals (AO's)
\begin{align*}
\phi_i = \sum^N_{\mu=1}c_{\mu i}\chi_\mu
\end{align*}
$c_{\mu i} \Rightarrow$ MO coefficients\\
$\chi_\mu\Rightarrow$ atomic basis functions.
\item Obtain coefficients by minimizing the energy via Variational Theorem.
\item Variational Theorem: Expectation value of the energy of a trial wavefunction is always greater than or equal to the true energy
\begin{align*}
E_e = \langle\psi_e\mid\hat{H}_e\mid\psi_e\rangle \ge \varepsilon_0
\end{align*}
\item Increasing $N \Rightarrow$ Higher quality of wavefunction $\Rightarrow$ Higher computational cost
\end{itemize}
}
\end{frame}

\begin{frame}
\frametitle{\small Ab Initio Methods}
\footnotesize{
\vspace{-0.2cm}
\begin{block}{\small The most popular classes of ab initio electronic structure methods:}
  \begin{itemize}
\item Hartree-Fock methods
  \begin{itemize}
\footnotesize{
  \item[$\vardiamond$] Hartree-Fock (HF)
  \begin{itemize}
  \item[$\blacksquare$] Restricted Hartree-Fock (RHF): singlets
  \item[$\blacksquare$] Unrestricted Hartree-Fock (UHF): higher multiplicities
  \item[$\blacksquare$] Restricted open-shell Hartree-Fock (ROHF)
  \end{itemize}
}
  \end{itemize}
\item Post Hartree-Fock methods
  \begin{itemize}
\footnotesize{
  \item[$\vardiamond$] M{\o}ller-Plesset perturbation theory (MPn)
  \item[$\vardiamond$] Configuration interaction (CI)
  \item[$\vardiamond$] Coupled cluster (CC)
}
  \end{itemize}
\item Multi-reference methods
  \begin{itemize}
\footnotesize{
  \item[$\vardiamond$] Multi-configurational self-consistent field (MCSCF)
  \item[$\vardiamond$] Multi-reference configuration interaction (MRCI)
  \item[$\vardiamond$] n-electron valence state perturbation theory (NEVPT)
  \item[$\vardiamond$] Complete active space perturbation theory (CASPTn)
}
  \end{itemize}
\end{itemize}
\end{block}
}
\end{frame}

\begin{frame}
\frametitle{\small Hartree-Fock}
\footnotesize{
\begin{enumerate}
\item Wavefunction is written as a single determinant
\begin{align*}
\Psi = det(\phi_1,\phi_2,\cdots\phi_N)
\end{align*}
\item The electronic Hamiltonian can be written as
\begin{align*}
\hat{H} = \sum_ih(i) + \sum_{i>j}v(i,j)
\end{align*}
where $\displaystyle{h(i) = -\dfrac{1}{2}\nabla^2_i - \sum_{i,\alpha}\dfrac{Z_\alpha}{r_{i\alpha}}}$ and $v(i,j) =\dfrac{1}{r_{ij}}$
\item The electronic energy of the system is given by:
\begin{align*}
E = \langle\Psi|\hat{H}|\Psi\rangle
\end{align*}
\item The resulting HF equations from minimization of energy by applying of variational theorem:
\begin{align*}
\hat{f}(x_1)\chi_i(x_1)= \varepsilon_i\chi_i(x_1)
\end{align*}
\item[] where $\varepsilon_i$ is the energy of orbital $\chi_i$ and the Fock operator $f$, is defined as
\begin{align*}
\hat{f}(x_1) = \hat{h}(x_1) + \sum_j\left[\hat{J}_j(x_1)-\hat{K}_j(x_1)\right]
\end{align*}
\end{enumerate}
}
\end{frame}

\begin{frame}
%\frametitle{\small Hartree-Fock}
\footnotesize{
\begin{enumerate}
\item $\hat{J}_j\Rightarrow$ Coulomb operator $\Rightarrow$ average potential at $x$ due to charge distribution from electron in orbital $\chi_i$ defined as
\begin{align*}
\hat{J}_j(x_1)\chi_i(x_1) = \left[\int\dfrac{\chi^\ast_j(x_2)\chi_j(x_2)}{r_{12}}dx_2\right]\chi_i(x_1)
\end{align*}
\item $\hat{K}_j\Rightarrow$ Exchange operator $\Rightarrow$ Energy associated with exchange of electrons $\Rightarrow$ No classical interpretation for this term.
\begin{align*}
\hat{K}_j(x_1)\chi_i(x_1) = \left[\int\dfrac{\chi^\ast_j(x_2)\chi_i(x_2)}{r_{12}}dx_2\right]\chi_j(x_1)
\end{align*}
\item The Hartree-Fock equation are solved numerically or in a space spanned by a set of basis functions (Hartree-Fock-Roothan equations)
\begin{columns}
\column{4cm}
\vspace{-0.5cm}
\begin{align*}
\chi_i &= \sum^K_{\mu=1}C_{\mu i}\tilde{\chi}_\mu\\
\sum_\nu F_{\mu\nu}C_{\nu i} &= \varepsilon_i\sum_\nu S_{\mu\nu}C_{\nu i}\\
{\bf FC}&={\bf SC}{\boldsymbol\varepsilon}
\end{align*}
\column{4cm}
\vspace{-0.35cm}
\begin{align*}
S_{\mu\nu} &= \int dx_1\tilde{\chi}^\ast_\mu(x_1)\tilde{\chi}_\nu(x_1)\\
F_{\mu\nu} &= \int dx_1\tilde{\chi}^\ast_\mu(x_1)\hat{f}(x_1)\tilde{\chi}_\nu(x_1)
\end{align*}
\end{columns}
\end{enumerate}
}
\end{frame}

\begin{frame}
%\frametitle{\small Hartree-Fock}
\footnotesize{
\begin{columns}
\column{12cm}
\begin{enumerate}
\item The Hartree-Fock-Roothan equation is a pseudo-eigenvalue equation
\item ${\bf C}$'s are the expansion coefficients for each orbital expressed as a linear combination of the basis function.
\item Note: ${\bf C}$ depends on ${\bf F}$ which depends on ${\bf C}\Rightarrow$ need to solve self-consistently.
\item Starting with an initial guess orbitals, the HF equations are solved iteratively or self consistently (Hence HF procedure is also known as self-consistent field or SCF approach) obtaining the best possible orbitals that minimize the energy.
\end{enumerate}
\vspace{-0.3cm}
\begin{block}{\footnotesize SCF procedure}
\begin{enumerate}
\item Specify molecule, basis functions and electronic state of interest
\item Form overlap matrix ${\bf S}$
\item Guess initial MO coefficients ${\bf C}$
\item Form Fock Matrix ${\bf F}$
\item Solve ${\bf FC}={\bf SC}{\boldsymbol\varepsilon}$
\item Use new MO coefficients ${\bf C}$ to build new Fock Matrix ${\bf F}$
\item Repeat steps 5 and 6 until ${\bf C}$ no longer changes from one iteration to the next.
\end{enumerate}
\end{block}
\end{columns}
}
\end{frame}

\begin{frame}
\frametitle{\small What are Post Hartree-Fock Methods}
\footnotesize{
\begin{columns}
\column{12cm}
\begin{enumerate}
\item In Hartree-Fock theory, electron motions of independent of each other i.e. uncorrelated.
\item However, this is not true. For two electrons with same spin $\mid\Psi_1(r_1)\alpha(\omega_1)\Psi_2(r_2)\alpha(\omega_2)\rangle$, the probability of finding electron 1 at $r_1$ and electron 2 at $r_2$
\begin{align*}
P(r_1,r_2)dr_1dr_2 &= \dfrac{1}{2}\left(|\Psi_1(r_1)|^2|\Psi_2(r_2)|^2 + |\Psi_1(r_2)|^2|\Psi_2(r_1)|^2 \right.\\ 
&- \left[\Psi^\ast_1(r_1)\Psi_2(r_1)\Psi^\ast_2(r_2)\Psi_1(r_2) \right.\\
& +\left.\left.\Psi^\ast_2(r_1)\Psi_1(r_1)\Psi^\ast_1(r_2)\Psi_2(r_2)\right]\right)dr_1dr_2
\end{align*}
\item[] Now $P(r_1,r_1)=0 \Rightarrow$ No two electrons with same spins can be at the same place $\Rightarrow$ "Fermi hole"
\item Same-spin electrons are correlated while different spin electrons are not.
\item Energy difference between HF energy and the true energy is the correlation energy
\begin{align*}
E_{corr} = E_0 - E_{HF}
\end{align*}
\end{enumerate}
\end{columns}
}
\end{frame}

\begin{frame}
\scriptsize{
\begin{columns}
\column{12cm}
\vspace{-0.5cm}
\begin{block}{}
\begin{itemize}
\item[$\vardiamond$]Methods that improve the Hartree-Fock results by accounting for the correlation energy are known as {\bf Post Hartree-Fock methods}
\item[$\vardiamond$]The starting point for most Post HF methods is the Slater Determinant obtain from Hartree-Fock Methods.
\item[$\vardiamond$]{\bf Configuration Interaction (CI) methods}: Express the wavefunction as a linear combination of Slater Determinants with the coeffcients obtained variationally
\begin{align*}
|\Psi\rangle = \sum_Ic_I|\Psi_I\rangle
\end{align*}
\item[$\vardiamond$]{\bf Many Body Perturbation Theory}: Treat the HF determinant as the zeroth order solution with the correlation energy as a pertubation to the HF equation.
\begin{align*}
\hat{H} &= \hat{H}_0 + \lambda\hat{H}^\prime\vspace{0.5cm};\vspace{0.5cm}
\varepsilon_i = E^{(0)}_i + \lambda E^{(1)}_i + \lambda^2E^{(2)}_i + \cdots\\
|\Psi_i\rangle &= |\Psi^{(0)}_i\rangle + \lambda|\Psi^{(1)}_i\rangle + \lambda^2|\Psi^{(2)}_i\rangle\cdots
\end{align*}
\item[$\vardiamond$]{\bf Coupled Cluster Theory}:The wavefunction is written as an exponential ansatz
\begin{align*}
|\Psi\rangle = e^{\hat{T}}|\Psi_0\rangle
\end{align*}
where $|\Psi_0\rangle$ is a Slater determinant obtained from HF calculations and $\hat{T}$ is an excitation operator which when acting on $|\Psi_0\rangle$ produces a linear combination of excited Slater determinants.
\end{itemize}
\end{block}
\end{columns}
}
\end{frame}

\begin{frame}
\frametitle{\small Scaling}
%\begin{block}{}
\begin{center}
\vspace{-1cm}
\includegraphics[width=\textwidth,keepaspectratio=true,clip=true]{Scaling}%Elec-Struct/Slide2}
\end{center}
\vspace{-1cm}
\begin{itemize}
\item N = Number of Basis Functions
\end{itemize}
%\end{block}
\end{frame}

%\part{Density Functional Theory}
\section{Density Functional Theory}
\begin{frame}
\frametitle{\small Density Functional Theory}
\footnotesize{
\begin{columns}
\column{12cm}
\begin{itemize}
\item Density Functional Theory (DFT) is an alternative to wavefunction based electronic structure methods of many-body systems such as Hartree-Fock and Post Hartree-Fock.
\item In DFT, the ground state energy is expressed in terms of the total electron density.
\begin{align*}
\rho_0(r) = \langle\Psi_0|\hat{\rho}|\Psi_0\rangle
\end{align*}
\item We again start with Born-Oppenheimer approximation and write the electronic Hamiltonian as
\begin{align*}
\hat{H} = \hat{F} + \hat{V}_{ext}
\end{align*}
where $\hat{F}$ is the sum of the kinetic energy of electrons and the electron-electron interaction and $\hat{V}_{ext}$ is some external potential.
\end{itemize}
\end{columns}
}
\end{frame}


\begin{frame}
%\frametitle{\small Density Functional Theory II}
\footnotesize{
\vspace{-0.5cm}
\begin{columns}
\column{12cm}
\begin{itemize}
\item Modern DFT methods result from the Hohenberg-Kohn theorem
\begin{enumerate}
{\footnotesize
\item The external potential $V_{ext}$, and hence total energy is a unique functional of the electron density $\rho(r)$
\begin{align*}
{\mathrm {Energy}} = \dfrac{\langle\Psi\mid\hat{H}\mid\Psi\rangle}{\langle\Psi\mid\Psi\rangle} \equiv E[\rho]
\end{align*}
\item The ground state energy can be obtained variationally, the density that minimizes the total energy is the exact ground state density
\begin{align*}
E[\rho] > E[\rho_0], {\mathrm if }\rho \ne \rho_0
\end{align*}
}
\end{enumerate}
\item If density is known, then the total energy is:
\begin{align*}
E[\rho]=T[\rho]+V_{ne}[\rho] + J[\rho] + E_{nn} + E_{xc}[\rho] 
\end{align*}
where 
\begin{align*}
E_{nn}[\rho] = \sum_{A>B}\dfrac{Z_AZ_B}{R_{AB}}  \hspace{1cm}&\hspace{1cm}
V_{ne}[\rho] = \int\rho(r)V_{ext}(r)dr \\
J[\rho] = \dfrac{1}{2}&\int\dfrac{\rho(r_1)\rho(r_2)}{r_{12}}dr_1dr_2
\end{align*}
\end{itemize}
\end{columns}
}
\end{frame}


\begin{frame}
%\frametitle{\small Density Functional Theory III}
\footnotesize{
\begin{columns}
\column{12cm}
\begin{itemize}
\item If the density is known, the two unknowns in the energy expression are the kinetic energy functional $T[\rho]$ and the exchange-correlation functional $E_{xc}[\rho]$
\item To calculate $T[\rho]$, Kohn and Sham introduced the concept of Kohn-Sham orbitals which are eigenvectors of the Kohn-Sham equation 
\begin{align*}
\left(-\dfrac{1}{2}\nabla^2+v_{\rm eff}(r)\right)\phi_{i}(r)=\varepsilon_{i}\phi_{i}(r)
\end{align*}
Here, $\varepsilon_i$ is the orbital energy of the corresponding Kohn-Sham orbital, $\phi_i$, and the density for an ''N''-particle system is
\begin{align*}
\rho(r)=\sum_i^N |\phi_{i}(r)|^2
\end{align*}
\item The total energy of a system is 
\begin{align*}
E[\rho]  = T_s[\rho] + \int dr\ v_{\rm ext}(r)\rho(r) + V_{H}[\rho] + E_{\rm xc}[\rho]
\end{align*}
\end{itemize}
\end{columns}
}
\end{frame}


\begin{frame}
%\frametitle{\small Density Functional Theory IV}
\footnotesize{
%\vspace{-0.5cm}
\begin{columns}
\column{12cm}
\begin{itemize}
\item $T_s$ is the Kohn-Sham kinetic energy which is expressed in terms of the Kohn-Sham orbitals as
\begin{align*}
T_s[\rho]=\sum_{i=1}^N\int dr\ \phi_i^*(r)\left(-\frac{1}{2}\nabla^2\right)\phi_i(r)
\end{align*}
$v_{\rm ext}$ is the external potential acting on the interacting system (at minimum, for a molecular system, the electron-nuclei interaction), $V_H$ is the Hartree (or Coulomb) energy,
\begin{align*}
 V_{H}=\dfrac{1}{2}\int drdr^\prime\  \dfrac{\rho(r)\rho(r^\prime)}{|r-r^\prime|}
\end{align*}
and $E_{xc}$ is the exchange-correlation energy. 
\item The Kohn-Sham equations are found by varying the total energy expression with respect to a set of orbitals to yield the Kohn-Sham potential as
\begin{align*}
v_{\rm eff}(r) = v_{\rm ext}(r) + \int \dfrac{\rho(r^\prime)}{|r-r^\prime|}dr^\prime + \dfrac{\delta E_{\rm xc}[\rho]}{\delta\rho(r)}
\end{align*}
where the last term $\displaystyle{v_{\rm xc}(r)\equiv\dfrac{\delta E_{\rm xc}[\rho]}{\delta\rho(r)}}$ is the exchange-correlation potential. 
\end{itemize}
\end{columns}
}
\end{frame}


\begin{frame}
%\frametitle{\small Density Functional Theory V}
\footnotesize{
\begin{columns}
\column{12cm}
%where the last term
%\begin{align*}
%v_{\rm xc}(r)\equiv\dfrac{\delta E_{\rm xc}[\rho]}{\delta\rho(r)}
%\end{align*}
%s the exchange-correlation potential. 
\begin{itemize}
\item The exchange-correlation potential, and the corresponding energy expression, are the only unknowns in the Kohn-Sham approach to density functional theory. 
%
%\item The exchange and correlation energy is expressed as functional of the electron density.
%\begin{align*}
%E[\rho] = T[\rho] + E_{ne}[\rho] + E_{xc}[\rho] + E_{ee}[\rho] + E_{nn}
%\end{align*}
\item There are many ways to approximate this functional $E_{\rm xc}$, generally divided into two separate terms
\begin{align*}
E_{\rm xc}[\rho] = E_{\rm x}[\rho] + E_{\rm c}[\rho]
\end{align*}
where the first term is the exchange functional while the second term is the correlation functional.
\item Quite a few research groups have developed the exchange and correlation functionals which are fit to empirical data or data from explicity correlated methods.
%\item The fitting of the functional is often done with empirical data (Hence DFT is not an "\emph{Ab Initio}" Method)
%\item Some density functionals can be considered \emph{Ab Initio} because they do not fit to empirical data.
\item Popular DFT functionals (according to a recent poll)
\begin{enumerate}
{\footnotesize
\item[$\vardiamond$] { PBE0 (PBEPBE), B3LYP, PBE, BP86, M06-2X, B2PLYP, B3PW91, B97-D, M06-L, CAM-B3LYP}
\item[$\blacksquare$]  {\scriptsize \url{http://www.marcelswart.eu/dft-poll/index.html}}
\item[$\blacksquare$]  {\scriptsize \url{http://www.ccl.net/cgi-bin/ccl/message-new?2011+02+16+009}}
}
\end{enumerate}
\end{itemize}
\end{columns}
}
\end{frame}


%\part{Semi-empirical Methods}
\section{Semi-empirical Methods}
\begin{frame}
\frametitle{\small Semi-empirical Methods}
\footnotesize{
\begin{itemize}
\item Semi-empirical quantum methods: 
\begin{enumerate}
\footnotesize{
\item[$\vardiamond$]Represents a middle road between the mostly qualitative results from molecular mechanics and the highly computationally demanding quantitative results from \emph{ab initio} methods.
\item[$\vardiamond$] Address limitations of the Hartree-Fock claculations, such as speed and low accuracy, by omitting or parametrizing certain integrals
}
\end{enumerate}
\item integrals are either determined directly from experimental data or calculated from analytical formula with \emph{ab initio} methods or from suitable parametric expressions.
\item Integral approximations:
\begin{enumerate}
\footnotesize{
\item[$\vardiamond$] Complete Neglect of Differential Overlap (CNDO)
\item[$\vardiamond$] Intermediate Neglect of Differential Overlap (INDO)
\item[$\vardiamond$] Neglect of Diatomic Differential Overlap (NDDO) ( Used by PM3, AM1, ...)
}
\end{enumerate} 
\end{itemize}
\begin{block}{}
Semi-empirical methods are fast, very accurate when applied to molecules that are similar to those used for parametrization and are applicable to very large molecular systems.
\end{block}
}
\end{frame}

\begin{frame}
\frametitle{\small Heirarchy of Methods}
%\begin{block}{}
\begin{center}
\vspace{-1cm}
\includegraphics[width=\textwidth,keepaspectratio=true, clip=true]{Elec-Corr}%Elec-Struct/Slide1}
\end{center}
%\end{block}
\end{frame}

\section{Basis Sets}
\begin{frame}
\frametitle{\small Basis Sets}
\footnotesize{
\begin{itemize}
\item %Most electronic structure codes use either a 
Slater type orbital (STO) or Gaussian type orbital (GTO) to describe the AO's
\begin{align*}
\chi^{\mathrm{STO}}(r) &= x^ly^mz^ne^{-\zeta r}\\
\chi^{\mathrm{GTO}}(r) &= x^ly^mz^ne^{-\xi r^2}
\end{align*}
where $L=l+m+n$ is the total angular momentun and $\zeta,\xi$ are orbital exponents. 
\end{itemize}
\begin{center}
\includegraphics[width=5cm,keepaspectratio=true,clip=true]{STO-GTO}
\end{center}
}
\end{frame}

\begin{frame}
\footnotesize{
\begin{columns}
\column{5.5cm}
\vspace{-0.2cm}
\begin{block}{Why STO}
\begin{itemize}
\item Correct cups at $r\rightarrow0$
\item Desired decay at $r\rightarrow\infty$ 
\item Correctly mimics H orbitals
\item Natural Choice for orbitals
\item Computationally expensive to compute integrals and derivatives.
\end{itemize}
\end{block}
\column{5.5cm}
\vspace{-0.2cm}
\begin{block}{Why GTO}
\begin{itemize}
\item Wrong behavior at $r\rightarrow0$ and $r\rightarrow\infty$
\item Gaussian $\times$ Gaussian = Gaussian
\item Analytical solutions for most integrals and derivatives.
\item Computationally less expensive than STO's
\end{itemize}
\end{block}
\end{columns}
}
\end{frame}

\begin{frame}
\footnotesize{
%\item Two main families fo basis sets used in modern computational chemistry calculations:
\begin{block}{Pople family basis set}
\begin{enumerate}
\item Minimal Basis: STO-nG
\begin{enumerate}
\item[$\vardiamond$]Each atom optimized STO is fit with n GTO's
\item[$\vardiamond$]Minimum number of AO's needed
\end{enumerate}
\item Split Valence Basis: 3-21G,4-31G, 6-31G
\begin{enumerate}
\item[$\vardiamond$]Contracted GTO's optimized per atom.
\item[$\vardiamond$]Valence AO's represented by 2 contracted GTO's
\end{enumerate}
\item Polarization: Add AO's with higher angular momentum (L)
\begin{enumerate}
\item[$\vardiamond$]3-21G* or 3-21G(d),6-31G* or 6-31G(d),6-31G** or 6-31G(d,p)
\end{enumerate}
\item Diffuse function: Add AO with very small exponents for systems with diffuse electron densities
\begin{enumerate}
\item[$\vardiamond$]6-31+G*, 6-311++G(d,p)
\end{enumerate}
\end{enumerate}
\end{block}
}
\end{frame}

\begin{frame}
\footnotesize{
\begin{block}{Correlation consistent basis set}
\begin{enumerate}
\item[$\vardiamond$]Family of basis sets of increasing sizes.
\item[$\vardiamond$]Can be used to extrapolate basis set limit.
\item[$\vardiamond$]cc-pVDZ: Double Zeta(DZ) with d's on heavy atoms, p's on H
\item[$\vardiamond$]cc-pVTZ: triple split valence with 2 sets of d's and 1 set of f's on heavy atom, 2 sets of p's and 1 set of d's on H
\item[$\vardiamond$]cc-pVQZ, cc-pV5Z, cc-pV6Z
\item[$\vardiamond$]can be augmented with diffuse functions: aug-cc-pVXZ (X=D,T,Q,5,6)
\end{enumerate}
\end{block}
}
\end{frame}

\begin{frame}
\footnotesize{
\begin{block}{Pseudopotentials or Effective Core Potentials}
\begin{enumerate}
\item[$\vardiamond$]All Electron calculations are prohibitively expensive.
\item[$\vardiamond$]Only valence electrons take part in bonding interaction leaving core electrons unaffected.\item[$\vardiamond$]Effective Core Potentials (ECP) a.k.a Pseudopotentials describe interactions between the core and valence electrons.
\item[$\vardiamond$]Only valence electrons explicitly described using basis sets.
\item[$\vardiamond$] Pseudopotentials commonly used
\begin{enumerate}
\footnotesize{
\item[$\blacksquare$]Los Alamos National Laboratory: LanL1MB and LanL2DZ
\item[$\blacksquare$]Stuttgard Dresden Pseudopotentials: SDDAll can be used.
\item[$\blacksquare$]Stevens/Basch/Krauss ECP's: CEP-4G,CEP-31G,CEP-121G
}
\end{enumerate}
\item[$\vardiamond$]Pseudopotential basis are "ALWAYS" read in pairs
\begin{enumerate}
\footnotesize{
\item[$\blacksquare$]Basis set for valence electrons
\item[$\blacksquare$]Parameters for core electrons
}
\end{enumerate}
\end{enumerate}
\end{block}
}
\end{frame}


%\part{Molecular Mechanics}
\section{Molecular Mechanics}
\begin{frame}
\frametitle{\small Molecular Mechanics}
\footnotesize{
\begin{itemize}
%\item Molecular mechanics uses Newtonian mechanics to model molecular systems. 
\item The potential energy of all systems in molecular mechanics is calculated using force fields. 
\item Molecular mechanics can be used to study small molecules as well as large biological systems or material assemblies with many thousands to millions of atoms.
\item All-atomistic molecular mechanics methods have the following properties:
\begin{itemize}
\footnotesize{
\item[$\vardiamond$] Each atom is simulated as a single particle
\item[$\vardiamond$] Each particle is assigned a radius (typically the van der Waals radius), polarizability, and a constant net charge (generally derived from quantum calculations and/or experiment)
\item[$\vardiamond$] Bonded interactions are treated as "springs" with an equilibrium distance equal to the experimental or calculated bond length
}
\end{itemize}
\item The exact functional form of the potential function, or force field, depends on the particular simulation program being used. 
\end{itemize}
}
\end{frame}

\begin{frame}
%\frametitle{\small Molecular Mechanics}
\footnotesize{
\begin{columns}
\column{6.5cm}
\begin{colorblock}{white}{blue!30!black}{black}{blue!8!white}{General form of Molecular Mechanics equations}
\begin{align*}
E&={\color{red}E_{\mathrm{bond}}}+{\color{blue}E_{\mathrm{angle}}} + {\color{tigerspurple}E_{\mathrm{torsion}}} + {\color{DarkGreen}E_{\mathrm{vdW}}} + {\color{indigo}E_{\mathrm{elec}}}\\
&={\color{red}\frac{1}{2}\sum_{\mathrm{bonds}}K_b(b-b_0)^2 \hspace{1.6cm}\mathrm{Bond}}\\
&+{\color{Blue}\frac{1}{2}\sum_{\mathrm{angles}}K_\theta(\theta-\theta_0)^2 \hspace{1.6cm}\mathrm{Angle}}\\
&+{\color{tigerspurple}\frac{1}{2}\sum_{\mathrm{dihedrals}}K_\phi\left[1+\cos(n\phi)\right]^2 \hspace{0.55cm}\mathrm{Torsion}}\\
&+\sum_{\mathrm{non bonds}}\left\{ \begin{array}{l}
{\color{DarkGreen}\left[\left(\dfrac{\sigma}{r}\right)^{12}-\left(\dfrac{\sigma}{r}\right)^{6}\right] \mathrm{van\,der\,Waals}}\\
+ {\color{indigo}\dfrac{q_1q_2}{Dr} \hspace{1.6cm}\mathrm{Electrostatics}}
       \end{array} \right.
\end{align*}
\end{colorblock}
\column{5.2cm}
\vspace{0.6cm}
\begin{colorblock}{white}{blue!30!black}{black}{white!95!black}{}
\includegraphics[width=5.2cm,keepaspectratio=true,clip=true]{MM_PEF}
\let\thefootnote\relax\footnotetext{\tiny Picture taken from }
\let\thefootnote\relax\footnotetext{\tiny http://en.wikipedia.org/wiki/Molecular\_mechanics}
\end{colorblock}
\end{columns}
}
\end{frame}

\section{Quantum Mechanics/Molecular Mechanics (QM/MM)}
\begin{frame}
\frametitle{\small QM/MM}
\begin{columns}
\column{12cm}
\footnotesize{
\vspace{-0.5cm}
\begin{colorblock}{white}{blue!30!black}{black}{white!70!black}{}
\begin{itemize}
\item What do we do if we want simulate chemical reaction in large systems?
\item Quantum Mechanics(QM): Accurate, expensive ($\mathcal{O}(N^4)$), suitable for small systems.
\item Molecular Mechanics(MM): Approximate, does not treat electrons explicitly, suitable for large systems such as enzymes and proteins, cannot simulate bon breaking/forming
\end{itemize}
\end{colorblock}
{
\begin{colorblock}{white}{blue!30!black}{black}{white!80!black}{}
\begin{itemize}
\item Methods that combine QM and MM are the solution.
\item Such methods are called Hybrid QM/MM methods.
\item The basic idea is to partition the system into two (or more) parts
\begin{enumerate}
\footnotesize{
\item The region of chemical interest is treated using accurate QM methods eg. active site of an enzyme.
\item The rest of the system is treated using MM or less accurate QM methods such as semi-empirical methods or  a combination of the two.
}
\end{enumerate}
\vspace{-0.3cm}
\begin{align*}
\hat{H}_{\rm Total} = \hat{H}_{\rm QM} + \hat{H}_{\rm MM} + \hat{H}^{\rm int}_{\rm QM-MM}
\end{align*}
\end{itemize}
\end{colorblock}}
}
\end{columns}
\end{frame}

\begin{frame}
\frametitle{\small QM/MM}
\footnotesize{
\begin{itemize}
\vspace{0.2cm}
\begin{colorblock}{white}{blue!30!black}{black}{black!25!white}{}
\item[] \textbf{ONIOM:} Divide the system into a real (full) system and the model system. Treat the model system at high and low level. The total energy of the system is given by
\begin{align*}
E = E(low,real) + E(high,model) - E(low,model)
\end{align*}
\end{colorblock}
\vspace{0.2cm}
\begin{colorblock}{white}{blue!30!black}{black}{black!20!white}{}
\item[] \textbf{Empirical Valence Bond:} Treat any point on a reaction surface as a combination of two or more valence bond structures
\begin{align*}
H({\bf R},{\bf r}) = \left|
\begin{array}{cc}
H_{11}({\bf R},{\bf r}) & H_{12}({\bf R},{\bf r})\\
H_{21}({\bf R},{\bf r}) & H_{22}({\bf R},{\bf r})
\end{array}
\right|
\end{align*}
\end{colorblock}
\vspace{0.2cm}
\begin{colorblock}{white}{blue!30!black}{black}{black!15!white}{}
\item[] \textbf{Effective Fragment Potential:} Divide a large system into fragments and perform \textit{ab initio} or DFT calculations of fragments and their dimers and including the Coulomb field from the whole system.
\end{colorblock}
\end{itemize}
}
\end{frame}

\section{Molecular Dynamics}
\begin{frame}
\frametitle{\small Molecular Dynamics}
\footnotesize{
\begin{columns}
\column{12cm}
\begin{beamerboxesrounded}[upper=uppercol,lower=lowercol,shadow=true]{\bf Why Molecular Dynamics?}
\begin{itemize}
\item Electronic Structure Methods are applicable to systems in gas phase under low pressure (vaccum).
\item Majority of chemical reactions take place in solution at some temperature with biological reactions usually at specific pH's.
\item Calculating molecular properties taking into account such environmental effects which can be dynamical in nature are not adequately described by electronic structure methods.
\end{itemize}
\end{beamerboxesrounded}
%\vspace{0.15cm}
{
\begin{colorblock}{white}{blue!30!black}{black}{blue!15!white}{\bf Molecular Dynamics}
\begin{itemize}
\item Generate a series of time-correlated points in phase-space (a trajectory).
\item Propagate the initial conditions, position and velocities in accordance with Newtonian Mechanics. ${\bf F}=m{\bf a}=-{\boldsymbol\nabla}V$
\item Fundamental Basis is the \textbf{Ergodic Hypothesis}: the average obtained by following a small number of particles over a
long time is equivalent to averaging over a large number of particles for a short time.
\end{itemize}
\end{colorblock}
}
\end{columns}
}
\end{frame}

\begin{frame}
\frametitle{\small Molecular Dynamics Theory}
\footnotesize{
\begin{itemize}
\item Solve the time-dependent Schr\"{o}dinger equation
\begin{align*}
\imath\hbar\dfrac{\partial}{\partial t}\Psi({\bf R},{\bf r},t) = \hat{H}\Psi({\bf R},{\bf r},t)
\end{align*}
with 
\begin{align*}
\Psi({\bf R},{\bf r},t) = \chi({\bf R},t)\Phi({\bf r},t)
\end{align*}
and 
\begin{align*}
\hat{H} = -\sum_I\dfrac{\hbar^2}{2M_I}\nabla^2_I +\underbrace{\dfrac{-\hbar^2}{2m_e}\nabla^2_i + V_{n-e}(\bf r,R)}_{H_e(\bf r,R)}
\end{align*}
\item Obtain coupled equations of motion for electrons and nuclei: Time-Dependent Self-Consistent Field (TD-SCF) approach.
\begin{align*}
\imath\hbar\dfrac{\partial\Phi}{\partial t} &= \left[-\sum_i\dfrac{\hbar^2}{2m_e}\nabla^2_i + \langle\chi|V_{n-e}|\chi\rangle\right]\Phi\\
\imath\hbar\dfrac{\partial\chi}{\partial t} &= \left[-\sum_I\dfrac{\hbar^2}{2M_I}\nabla^2_I + \langle\Phi|H_e|\Phi\rangle\right]\chi
\end{align*}
\end{itemize}
}
\end{frame}

\begin{frame}
%\frametitle{\small Molecular Dynamics Theory}
\footnotesize{
\begin{itemize}
%\item Alternative approach: Decouple electronic and nuclei motions.
%\item Solve for the clamped nuclei:
%\begin{align*}
%\hat{H}_e\Phi_k({\bf r};{\bf R}) = E_k({\bf R})\Phi_k({\bf r};{\bf R})
%\end{align*}
%\item Equation of motion for nuclear wavefunction is
%\begin{align*}
%\imath\hbar\dfrac{\partial|\chi\rangle}{\partial t} &= \left[-\sum_I\dfrac{\hbar^2}{2M_I}\nabla^2_I + E({\bf R})\right]|%\chi\rangle
%\end{align*}
\item Define nuclear wavefunction as
\begin{align*}
\chi({\bf R},t) = A({\bf R},t)\exp\left[iS({\bf R},t)/\hbar\right]
\end{align*}
where $A$ and $S$ are real.
\item Solve the time-dependent equation for nuclear wavefunction and take classical limit ($\hbar\rightarrow0$) to obtain 
\begin{align*}
\dfrac{\partial S}{\partial t} + \sum_I\dfrac{\hbar^2}{2M_I}(\nabla_IS)^2 + \langle\Phi|H_e|\Phi\rangle = 0
\end{align*}
an equation that is isomorphic with the Hamilton-Jacobi equation with the classical Hamilton function given by
\begin{align*}
\mathcal{H}(\{{\bf R}_I\},\{{\bf P}_I\}) = \sum_I \dfrac{\hbar^2}{2M_I}{\bf P}^2_I + V(\{{\bf R}_I\})
\end{align*}
where
\begin{align*}
{\bf P}_I \equiv \nabla_IS\hspace{1cm}\mathrm{and}\hspace{1cm}
V(\{{\bf R}_I\}) = \langle\Phi|H_e|\Phi\rangle
\end{align*}
\item Obtain equations of nuclear motion from Hamilton's equation
\begin{align*}
\dfrac{d{\bf P}_I}{dt} = -\dfrac{d\mathcal{H}}{d{\bf R}_I}&\Rightarrow M\ddot{\bf R}_I = -\nabla _IV\\
\dfrac{d{\bf R}_I}{dt} &= \dfrac{d\mathcal{H}}{d{\bf P}_I}
\end{align*}
\end{itemize}
}
\end{frame}

\begin{frame}
\footnotesize{
\begin{itemize}
\item Replace nuclear wavefunction by delta functions centered on nuclear position to obtain
\begin{align*}
i\hbar\dfrac{\partial\Phi}{\partial t} = H_e({\bf r},\{{\bf R}_I\})\Phi({\bf r};\{{\bf R}_I\},t)
\end{align*}
\item This approach of simultaneously solving the electronic and nuclear degrees of freedom by incorporating feedback in both directions is known as \textbf{Ehrenfest Molecular Dynamics}.
\item Expand $\Phi$ in terms of many electron wavefunctions or  determinants
\begin{align*}
\Phi({\bf r};\{{\bf R}_I\},t) = \sum_i c_i(t)\Phi_i({\bf r};\{{\bf R}_I\})
\end{align*}
with matrix elements
\begin{align*}
H_{ij} = \langle\Phi_i|H_e|\Phi_j\rangle
\end{align*} 
\item Inserting $\Phi$ in the TDSE above, we get
\begin{align*}
\imath\hbar\dot{c}_i(t) = c_i(t)H_{ii} -\imath\hbar\sum_{I,i}\dot{\bf R}_I{\bf d}^{ij}_I
\end{align*}
with non-adiabtic coupling elements given by
\begin{align*}
{\bf d}^{ij}_I({\bf R}_I) = \langle\Phi_i|\nabla_I|\Phi_j\rangle
\end{align*}
\end{itemize}
}
\end{frame}

\begin{frame}[allowframebreaks]
\footnotesize{
\begin{itemize}
\item Upto this point, no restriction on the nature of $\Phi_i$ i.e. adiabatic or diabatic basis has been made. 
\item Ehrenfest method rigorously includes non-adiabtic transitions between electronic states within the framework of classical nuclear motion and mean field (TD-SCF) approximation to the electronic structure.
\item Now suppose, we define $\{\Phi_i\}$ to be the adiabatic basis obtained from solving the time-independent Schr\"{o}dinger equation,
\begin{align*}
H_e({\bf r},\{{\bf R}_I\})\Phi_i({\bf r};\{{\bf R}_I\}) = E_i(\{{\bf R}_I\})\Phi_i({\bf r};\{{\bf R}_I\})
\end{align*}
\item The classical nuclei now move along the adiabatic or Born-Oppenheimer potential surface. Such dynamics are commonly known as \textbf{Born-Oppenheimer Molecular Dynamics} or BOMD.
\item If we restrict the dynamics to only the ground electronic state, then we obtain ground state BOMD.
\item If the Ehrenfest potential $V(\{{\bf R}_I\})$ is approximated to a global potential surface in terms of many-body contributions $\{v_n\}$.
\begin{align*}
V(\{{\bf R}_I\}) \approx V^{approx}_e({\bf R}) = \sum^{N}_{I=1}v_1({\bf R}_I) + \sum^N_{I>J}v_2({\bf R}_I,{\bf R}_J) +\sum^N_{I>J>K}v_3({\bf R}_I,{\bf R}_J,{\bf R}_K) + \cdots
\end{align*}
\item[] 
\item Thus the problem is reduced to purely classical mechanics once the $\{v_n\}$ are determined usually Molecular Mechanics Force Fields. This class of dynamics is most commonly known as {\bf Classical Molecular Dynamics}.
\item Another approach to obtain equations of motion for \text{ab-initio} molecular dynamics is to apply the Born-Oppenheimer approximation to the full wavefunction $\Psi({\bf r},{\bf R},t)$
\begin{align*}
\Psi({\bf r},{\bf R},t) = \sum_k \chi({\bf R},t)\Phi_k({\bf r};{\bf R}(t))
\end{align*}
where
\begin{align*}
H_e\Phi_k({\bf r};{\bf R}(t)) = E_k({\bf R}(t))\Phi_k({\bf r};{\bf R}(t))
\end{align*}
\item Assuming that the nuclear dynamics doesn not change the electronic state, we arrive at the equation of motion for nuclear wavefunction 
\begin{align*}
\imath\hbar\dfrac{\partial}{\partial t}\chi({\bf R},t) = \left[\sum_I-\dfrac{\hbar^2}{2M_I}\nabla^2_I + E_k({\bf R})\right]\chi({\bf R},t)
\end{align*}
\item The Lagrangian for this system is given by.
\begin{align*}
\mathcal{L} = \hat{T} - \hat{V}
\end{align*}
\item Corresponding Newton's equation of motion are then obtained from the associated Euler-Lagrange equations,
\begin{align*}
\dfrac{d}{dt}\dfrac{\partial\mathcal{L}}{\partial\dot{\bf R}_I} = \dfrac{\partial\mathcal{L}}{\partial{\bf R}_I}
\end{align*}
\item The Lagrangian for ground state BOMD is
\begin{align*}
\mathcal{L}_\mathrm{BOMD} = \sum_I\dfrac{1}{2}M_I\dot{\bf R}^2_I - \min_{\Phi_0}\langle\Phi|H_e|\Phi\rangle
\end{align*}
and equations of motions
\begin{align*}
{\color{blue}M_I\ddot{\bf R}_I} = \dfrac{d}{dt}\dfrac{\partial\mathcal{L}_\mathrm{BOMD}}{\partial\dot{\bf R}_I} = \dfrac{\partial\mathcal{L}_\mathrm{BOMD}}{\partial{\bf R}_I} = {\color{blue}-\nabla_I\min_{\Phi_0}\langle\Phi|H_e|\Phi\rangle}
\end{align*}
\end{itemize}
\begin{colorblock}{white}{blue!30!black}{black}{blue!15!white}{\textbf{Extended Lagrangian Molecular Dynamics} (ELMD)}
Extend the Lagrangian by adding kinetic energy of fictitious particles and obtain their equation of motions from Euler-Lagrange equations.
\end{colorblock}
\vspace{-0.3cm}
\begin{columns}
\column{6cm}
\centering{
\begin{colorblock}{white}{blue!30!black}{black}{green!25!white}{}
\centering{
Molecular Orbitals: $\{\phi_i\}$ \\
Density Matrix: $\displaystyle{{P}_{\mu\nu}=\sum_{i}c^\ast_{\mu i}c_{\nu i}}$}
\end{colorblock}
}
\end{columns}
}
\end{frame}

\begin{frame}
\footnotesize{
\begin{colorblock}{white}{green!30!black}{black}{green!15!white}{\textbf{Car-Parrinello Molecular Dynamics} (CPMD)}
{\color{DarkBlue}CPMD and NWCHEM}\let\thefootnote\relax\footnotetext{\tiny R. Car and M. Parrinello, Phys. Rev. Lett. 55 (22), 2471 (1985)}
\vspace{-0.1cm}
\begin{align*}
\mathcal{L}_\mathrm{CPMD} = \sum_I\dfrac{1}{2}M_I\dot{\bf R}^2_I + \sum_i\dfrac{1}{2}\mu_i\langle\dot{\phi_i}|\dot{\phi_i}\rangle - \langle\Phi_0|H_e|\Phi_0\rangle + \mathrm{constraints}
\end{align*}
\end{colorblock}
 \begin{colorblock}{white}{green!30!black}{black}{green!15!white}{\textbf{Atom centered Density Matrix Propagation} (ADMP)}
{\color{DarkBlue}Gaussian 03/09}\let\thefootnote\relax\footnotetext{\tiny H. B. Schlegel, J. M. Millam, S. S. Iyengar, G. A. Voth, A. D. Daniels, G. E. Scuseria, M. J. Frisch, J. Chem. Phys. 114, 9758 (2001)}
\vspace{-0.1cm}
\begin{align*}
\mathcal{L}_\mathrm{ADMP} = \dfrac{1}{2}\mathrm{Tr}({\bf V}^T{\bf M}{\bf V}) + \dfrac{1}{2}\mu\mathrm{Tr}(\dot{\bf P}\dot{\bf P}) - E({\bf R},{\bf P}) - \mathrm{Tr}[{\boldsymbol\Lambda}({\bf PP}-{\bf P})]
\end{align*}
\end{colorblock}
\begin{colorblock}{white}{red!30!black}{black}{red!15!white}{\textbf{curvy-steps ELMD} (csELMD)}
{\color{purple}Q-Chem}\let\thefootnote\relax\footnotetext{\tiny J.M. Herbert and M. Head-Gordon, J. Chem. Phys. 121, 11542 (2004)}
\vspace{-0.1cm}
\begin{align*}
\mathcal{L}_\mathrm{csELMD} = \sum_I\dfrac{1}{2}M_I\dot{\bf R}^2_I + \dfrac{1}{2}\mu\sum_{i<j}\dot{\Delta}_{ij} - E({\bf R},{\bf P}) ;\hspace{0.2cm} {\bf P}(\lambda) = e^{\lambda\Delta}{\bf P}(0)e^{-\lambda\Delta}
\end{align*}
\end{colorblock}
}
\end{frame}

\beamertemplateshadingbackground{green!10!white}{green!10!white} % New background
\begin{frame}
\frametitle{\small Molecular Dynamics: Methods and Programs}
\begin{columns}
\column{12cm}
\begin{itemize}
\item Electronic energy obtained from 
\begin{itemize}
\item {Molecular Mechanics $\Rightarrow$ Classical Molecular Dynamics}
\begin{enumerate}
{\footnotesize
\item LAMMPS
\item NAMD
\item Amber
\item Gromacs
}
\end{enumerate}
\item {Ab-Initio Methods $\Rightarrow$ Quantum or Ab-Initio Molecular Dynamics}
\begin{enumerate}
{\footnotesize
\item Born-Oppenheimer Molecular Dynamics: Gaussian, GAMESS
\item Extended Lagrangian Molecular Dynamics: VASP, CPMD, Gaussian (ADMP), NWCHEM(CPMD), {\color{red}QChem (curvy-steps ELMD)}
\item Time Dependent Hartree-Fock and Time Dependent Density Functional Theory: Gaussian, GAMESS, NWCHEM, {\color{red}QChem}
\item {\color{red}Multiconfiguration Time Dependent Hartree(-Fock), MCTDH(F)}
\item {\color{red}Non-Adiabatic and  Ehrenfest Molecular Dynamics, Multiple Spawning, Trajectory Surface Hopping}
\item {\color{red}Quantum Nuclei: QWAIMD(Gaussian), NEO(GAMESS)}
}
\end{enumerate}
\end{itemize}
\end{itemize}
\end{columns}
\end{frame}
\beamertemplateshadingbackground{blue!5}{blue!5} % back to original background
\begin{frame}
%\frametitle{\small Classical Molecular Dynamics}
\begin{block}{Classical Molecular Dynamics}
\begin{itemize}
\item Advantages
{\scriptsize
\begin{enumerate}
\item Large Biological Systems
\item Long time dynamics
\end{enumerate}
}
\item Disadvantages
{\footnotesize
\begin{enumerate}
\item Cannot describe Quantum Nuclear Effects
\end{enumerate}
}
\end{itemize}
\end{block}
\begin{block}{\textit{Ab Initio} and Quantum Dynamics}
\begin{itemize}
\item Advantages
{\footnotesize
\begin{enumerate}
\item Quantum Nuclear Effects
\end{enumerate}
}
\item Disadvantages
{\footnotesize
\begin{enumerate}
\item $\sim$ 100 atoms
\item Full Quantum Dynamics ie treating nuclei quantum mechanically: less than 10 atoms
\item Picosecond dynamics at best
\end{enumerate}
}
\end{itemize}
\end{block}
%{\footnotesize
%\begin{enumerate}
%\end{enumerate}
%}
\end{frame}


\section{Computational Chemistry Programs}

\begin{frame}
%\frametitle{\small Software Installed on Linux Systems}
\scriptsize{
\begin{columns}
\column{12cm}
\vspace{-1cm}
\begin{center}
\begin{tikzpicture}
\node (tbl) {
\begin{tabularx}{\textwidth}{ccccccccc}
\arrayrulecolor{tigersgold}
\textcolor{white}{\textbf{Software} }& \textcolor{white}{\textbf{QB}} &\textcolor{white}{\textbf {Eric}} & \textcolor{white}{\textbf{Louie}} & \textcolor{white}{\textbf{Oliver}} & \textcolor{white}{\textbf{Painter}} & \textcolor{white}{\textbf{Poseidon}} & \textcolor{orange}{\textbf{Philip}} & \textcolor{orange}{\textbf{Tezpur}}\\
Amber\rule{0pt}{3.5ex} & \checkmark & \checkmark & \checkmark & \checkmark & \checkmark & \checkmark & \checkmark & \checkmark \\ 
Desmond & \checkmark &  &  &  &  &  &  &  \\
DL\_Poly & \checkmark & \checkmark & \checkmark & \checkmark & \checkmark & \checkmark & \checkmark & \checkmark \\
Gromacs & \checkmark & \checkmark & \checkmark & \checkmark & \checkmark & \checkmark & \checkmark & \checkmark \\
LAMMPS & \checkmark & \checkmark & \checkmark & \checkmark & \checkmark & \checkmark & \checkmark & \checkmark \\
NAMD & \checkmark & \checkmark & \checkmark & \checkmark & \checkmark & \checkmark &  & \checkmark \\
OpenEye & \checkmark & \checkmark & \checkmark & \checkmark & \checkmark & \checkmark & \checkmark & \checkmark \\
CPMD & \checkmark & \checkmark & \checkmark & \checkmark & \checkmark & \checkmark &  & \checkmark \\
GAMESS & \checkmark & \checkmark & \checkmark & \checkmark & \checkmark & \checkmark & \checkmark & \checkmark \\
Gaussian &  & \checkmark & \checkmark & \checkmark & \checkmark &  & \checkmark & \checkmark \\
NWCHEM & \checkmark & \checkmark & \checkmark & \checkmark & \checkmark & \checkmark &  & \checkmark \\
Piny\_MD & \checkmark & \checkmark & \checkmark & \checkmark & \checkmark & \checkmark & \checkmark & \checkmark  \\
[1.0ex]
\end{tabularx}};
\begin{pgfonlayer}{background}
\draw[rounded corners,top color=blue!30!black,bottom color=blue!10!white,
    draw=tigerspurple!30] ($(tbl.north west)+(0.14,0)$)
    rectangle ($(tbl.north east)-(0.13,0.9)$);
%\draw[rounded corners,top color=tigersgold,bottom color=tigerspurple!20,
 %   middle color=tigerspurple!20,draw=tigerspurple!20] ($(tbl.south west)
 %   +(0.12,0.5)$) rectangle ($(tbl.south east)-(0.12,0)$);
\draw[rounded corners,top color=green!5,bottom color=green!5,draw=green!5]
    ($(tbl.north east)-(0.13,0.6)$)
    rectangle ($(tbl.south west)+(0.13,0.2)$);
\end{pgfonlayer}
\end{tikzpicture}
\end{center}
\vspace{-1cm}
\begin{center}
\begin{tikzpicture}
\node (tbl) {
\begin{tabularx}{\textwidth}{cccccccc}
\arrayrulecolor{tigersgold}
\textcolor{white}{\textbf{Software}} & \textcolor{white}{\textbf{Bluedawg}} & \textcolor{white}{\textbf {Ducky}} & \textcolor{white}{\textbf{Lacumba}} & \textcolor{white}{\textbf{Neptune}} & \textcolor{white}{\textbf{Zeke}} & \textcolor{orange}{\textbf{Pelican}} & \textcolor{orange}{\textbf{Pandora}}\\
Amber\rule{0pt}{3.5ex} &  & \checkmark & \checkmark &  &  & \checkmark & \\ 
Gromacs & \checkmark & \checkmark & \checkmark & \checkmark & \checkmark & \checkmark & \\
LAMMPS & \checkmark & \checkmark & \checkmark & \checkmark & \checkmark & \checkmark & \\
NAMD & \checkmark & \checkmark & \checkmark & \checkmark & \checkmark &  &\\
CPMD & \checkmark & \checkmark & \checkmark & \checkmark & \checkmark &  & \\
Gaussian & \checkmark & \checkmark & \checkmark &  & \checkmark & \checkmark & \\
NWCHEM & \checkmark & \checkmark & \checkmark & \checkmark & \checkmark & \checkmark & \\
Piny\_MD & \checkmark & \checkmark & \checkmark & \checkmark & \checkmark & \checkmark  & \\
[1.0ex]
\end{tabularx}};
\begin{pgfonlayer}{background}
\draw[rounded corners,top color=green!40!black,bottom color=green!10!white,
    draw=green!5!white] ($(tbl.north west)+(0.14,0)$)
    rectangle ($(tbl.north east)-(0.13,0.9)$);
%\draw[rounded corners,top color=tigersgold,bottom color=tigerspurple!20,
 %   middle color=tigerspurple!20,draw=tigerspurple!20] ($(tbl.south west)
 %   +(0.12,0.5)$) rectangle ($(tbl.south east)-(0.12,0)$);
\draw[rounded corners,top color=purple!5,bottom color=purple!5,draw=purple!5]
    ($(tbl.north east)-(0.13,0.6)$)
    rectangle ($(tbl.south west)+(0.13,0.2)$);
\end{pgfonlayer}
\end{tikzpicture}
\end{center}
\end{columns}
}
\end{frame}

\begin{frame}
\frametitle{\small Computational Chemistry Programs}
\begin{itemize}
 \item {\color{black}Commercial Software: Q-Chem, Jaguar,CHARMM}
 \item {\color{tigerspurple}GPL/Free Software: ACES, ABINIT, Octopus}
 \item {\color{tigersblue}{\bf \url{http://en.wikipedia.org/wiki/Quantum\_chemistry\_computer\_programs}}}
 \item {\color{tigersblue}{\bf \url{http://www.ccl.net/chemistry/links/software/index.shtml}}}
 \item {\color{tigersblue}{\bf \url{http://www.redbrick.dcu.ie/~noel/linux4chemistry/}}}
\end{itemize}
\end{frame}

\begin{frame}
\frametitle{\small Job Types and Keywords}
\footnotesize{
\vspace{-0.5cm}
\begin{columns}
\column{12cm}
\begin{exampleblock}{}
\begin{tabular}{|c|c|c|c|}
\hline
Job Type & Gaussian & GAMESS & NWCHEM\\
\cline{2-4}
  & \# keyword & runtyp= & task \\       
\hline
Energy & sp & energy & energy \\
Force & force & gradient & gradient \\
Geometry optimization & opt & optimize & optimize \\
Transition State & opt=ts & sadpoint & saddle \\
Frequency & freq & hessian & frequencies, freq \\
Potential Energy Scan & scan & surface & \checkmark \\
Excited State & \checkmark & \checkmark & \checkmark \\
Reaction path following & irc & irc & \checkmark \\
Molecular Dynamics & admp, bomd & drc & dynamics, Car-Parrinello \\
Population Analysis & pop & pop & \checkmark \\
Electrostatic Properties & prop & \checkmark & \checkmark \\
Molecular Mechanics & \checkmark & \checkmark & \checkmark \\
Solvation Models & \checkmark & \checkmark & \checkmark \\
QM/MM & oniom & \checkmark & qmmm \\
\hline
\end{tabular}
\end{exampleblock}
\end{columns}
}
\end{frame}

\begin{frame}
%\footnotesize{
\vspace{0.5cm}
\begin{colorblock}{white}{green!30!black}{black}{green!15!white}{Molecular Dynamics Calculations}
\begin{itemize}
\item Gaussian:
 \begin{itemize}
% {\scriptsize
 \item BOMD: Born-Oppenheimer Molecular Dynamics
 \item ADMP: Atom centered Density Matrix Propagation (an extended Lagrangian Molecular Dynamics similar to CPMD) and ground state BOMD
% }
 \end{itemize}
\item GAMESS:
  \begin{itemize}
  \item DRC: Direct Dynamics, a classical trajectory method based on "on-the-fly" ab-initio or semi-empirical potential energy surfaces
  \end{itemize}
\item NWCHEM:
  \begin{itemize}
  \item Car-Parrinello: Car Parrinello Molecular Dynamics (CPMD)
%   \item dynamics: Perform classical molecular dynamics
  \item DIRDYVTST: Direct Dynamics Calculations using POLYRATE with electronic structure from NWCHEM
  \end{itemize}
\end{itemize}
\end{colorblock}
%}
\end{frame}

\begin{frame}
\frametitle{\small Related HPC Tutorials}
\footnotesize{
\begin{itemize}
\item Fall Semester
\begin{block}{\small Introduction to Gaussian/Electronic Structure Methods}
\end{block}
\item Spring Semester
%\begin{alertblock}{MD: Programming to Production}
%April 6$^{\rm th}$
%\end{alertblock}
\begin{alertblock}{\small Introduction to Computational Chemistry: Molecular Dynamics}
April 27$^{\rm th}$
\end{alertblock}
\end{itemize}
}
\end{frame}

\begin{frame}[allowframebreaks]
%\frametitle{\small Useful Links/Further Reading}
\begin{block}{Useful Links}
\begin{itemize}
{\footnotesize\color{Blue}
\item{\color{tigerspurple}Amber:}\url{http://ambermd.org}
\item{\color{tigerspurple}Desmond:}\url{http://www.deshawresearch.com/resources_desmond.html}
\item{\color{tigerspurple}DL\_POLY:}\url{http://www.cse.scitech.ac.uk/ccg/software/DL_POLY}
\item{\color{tigerspurple}Gromacs:}\url{http://www.gromacs.org}
\item{\color{tigerspurple}LAMMPS:}\url{http://lammps.sandia.gov}
\item{\color{tigerspurple}NAMD:}\url{http://www.ks.uiuc.edu/Research/namd}
\item {\color{tigerspurple}CPMD:} \url{http://www.cpmd.org}
\item {\color{tigerspurple}GAMESS:} \url{http://www.msg.chem.iastate.edu/gamess}
\item {\color{tigerspurple}Gaussian:} \url{http://www.gaussian.com}
\item {\color{tigerspurple}NWCHEM:} \url{http://www.nwchem-sw.org}
\item{\color{tigerspurple}PINY\_MD:}\url{http://homepages.nyu.edu/~mt33/PINY_MD/PINY.html}
\item {\color{tigerspurple}Basis Set:} \url{https://bse.pnl.gov/bse/portal}
}
\end{itemize}
\end{block}
\begin{block}{\small Further Reading}
\begin{itemize}
{\scriptsize
\item David Sherill's Notes at Ga Tech: {\color{Blue}\url{http://vergil.chemistry.gatech.edu/notes/index.html}}
\item Mark Tuckerman's Notes at NYU: {\color{Blue}\url{http://www.nyu.edu/classes/tuckerman/quant.mech/index.html}}
\item Modern Quantum Chemistry: Introduction to Advanced Electronic Structure Theory, A. Szabo and N. Ostlund
\item Introduction to Computational Chemistry, F. Jensen
\item Essentials of Cmputational Chemistry - Theories and Models, C. J. Cramer
\item Exploring Chemistry with Electronic Structure Methods, J. B. Foresman and A. Frisch
\item Ab Initio Molecular Dynamics: Theory and Implementation, D. Marx and J. Hutter{\color{Blue}\url{http://www.theochem.ruhr-uni-bochum.de/research/marx/marx.pdf}}
\item Molecular Modeling - Principles and Applications, A. R. Leach
\item Computer Simulation of Liquids, M. P. Allen and D. J. Tildesley
\item[$\vardiamond$]Modern Electronic Structure Theory, T. Helgaker, P. Jorgensen and J. Olsen (Highly advanced text, second quantization approach to electronic structure theory)
}
\end{itemize}
\end{block}
\end{frame}

\section{Example Jobs}
\begin{frame}
%\frametitle{\small Worked out Examples On LONI Linux Systems}
\begin{columns}
\column{1.16\textwidth}
\vspace{-0.95cm}
\begin{alertblock}{On LONI Linux Systems}
\begin{center}
\includegraphics[width=0.74\textwidth,clip=true]{workshop}
\end{center}
\end{alertblock}
\end{columns}
\end{frame}

\begin{frame}
\frametitle{\small Using Gaussian on LONI Systems}
\begin{itemize}
\item Site specific license% ({\color{tigerspurple}Gaussian 03})
\begin{enumerate}
 \item {\color{tigersblue}Gaussian 03 and 09}
\begin{itemize}
 \item {\color{tigersblue}LSU Users}: Eric
 \item {\color{tigersblue}Latech Users}: Painter, Bluedawg
\end{itemize}
 \item {\color{tigerspurple}Gaussian 03}
\begin{itemize}
 \item {\color{tigerspurple}ULL Users}: Oliver, Zeke
 \item {\color{tigerspurple}Tulane Users}: Louie, Ducky
 \item {\color{tigerspurple}Southern Users}: Lacumba
\end{itemize}
 \item UNO Users: No License
\end{enumerate}
\item Add {\texttt +gaussian-03/+gaussian-09} to your .soft file and resoft
\item \alert{If your institution has license to both G03 and G09, have only one active at a given time.}
\end{itemize}
\end{frame}

\begin{frame}[fragile]
\tiny{
\begin{exampleblock}{Example Job submission script on Intel x86}
{\color{white}
\begin{verbatim}
 #!/bin/tcsh
 #PBS -A your_allocation
 # specify the allocation. Change it to your allocation
 #PBS -q checkpt
 # the queue to be used. 
 #PBS -l nodes=1:ppn=4
 # Number of nodes and processors
 #PBS -l walltime=1:00:00
 # requested Wall-clock time.
 #PBS -o g03_output 
 # name of the standard out file to be "output-file".
 #PBS -j oe 
 # standard error output merge to the standard output file.
 #PBS -N g03test 
 # name of the job (that will appear on executing the qstat command).
 
 # setup g03 variables
 source $g03root/g03/bsd/g03.login
 set NPROCS=`wc -l $PBS_NODEFILE |gawk '//{print $1}'`
 setenv GAUSS_SCRDIR /scratch/$USER
 # cd to the directory with Your input file
 cd ~apacheco/g03test 
 # Change this line to reflect your input file and output file
 g03 < g03job.inp  > g03job.out
\end{verbatim}
}
\end{exampleblock}
\begin{alertblock}{Linda Access}
% If your instition have TCP Linda license, then you can run parallel jobs by adding the following to your job submission script
\begin{verbatim}
  set NODELIST = ( -vv -nodelist '"' `cat $PBS_NODEFILE` '"' -mp 4)
  setenv GAUSS_LFLAGS " $NODELIST "
  g03l < g03job.inp  > g03job.out
\end{verbatim}
\end{alertblock}
}
\end{frame}

\begin{frame}[fragile]
\tiny{
\begin{exampleblock}{Example Job submission script on P5}
{\color{white}
\begin{verbatim}
 #!/bin/tcsh
 # @ account_no = your_allocation
 # @ requirements = (Arch == "Power5")
 # @ environment = LL_JOB=TRUE ;  MP_PULSE=1200
 # @ job_type = parallel
 # @ node_usage = shared
 # @ wall_clock_limit = 12:00:00
 # @ initialdir = /home/apacheco/g03test
 # @ class = checkpt
 # @ error = g03_$(jobid).err
 # @ queue

 # setup g03 variables
 source $g03root/g03/bsd/g03.login
 # setup and create Gaussian scratch directory
 setenv GAUSS_SCRDIR /scratch/default/$USER
 mkdir -p $GAUSS_SCRDIR
 # cd to the directory with Your input file
 cd ~apacheco/g03test
 # Change this line to reflect your input file and output file
 g03 < g03job.inp  > g03job.out
\end{verbatim}
}
\end{exampleblock}
}
\end{frame}

\begin{frame}[fragile]
\footnotesize{
% \begin{itemize}
% \item Geometry Optimzation and Freqeuncy calculation for water molecule
% \end{itemize}
\begin{columns}
\column{6.5cm}
\begin{block}{Sample Input}
\begin{verbatim}
%chk=h2o-opt-freq.chk
%mem=512mb
%NProcShared=4

#p b3lyp/6-31G opt freq

H2O OPT FREQ B3LYP

0 1
O
H 1 r1
H 1 r1 2 a1

r1 1.05 
a1 104.5

\end{verbatim}
\end{block}
\column{5cm}
{\color{tigersblue}
\begin{alertblock}{Input Description}
\begin{verbatim}
checkpoint file
amount of memory
number of smp processors
blank line
Job description
blank line
Job Title
blank line
Charge Multiplicity
Molecule Description
 Z-matrix format 
 with variables
blank line
 variable value

blank line
\end{verbatim}
\end{alertblock}
}
\end{columns}
}
\end{frame}

\begin{frame}[fragile]
\frametitle{\small Using GAMESS on LONI Systems}
\begin{itemize}
\item Add +gamess-12Jan2009R1-intel-11.1 (on Queenbee) to your .soft and resoft
\end{itemize}
{\tiny
\begin{exampleblock}{Job submission script}
{\color{white}
\begin{verbatim}
#!/bin/bash
#PBS -A your_allocation
#PBS -q checkpt
#PBS -l nodes=1:ppn=4
#PBS -l walltime=00:10:00
#PBS -j oe
#PBS -N gamess-exam1

export WORKDIR=$PBS_O_WORKDIR
export NPROCS=`wc -l $PBS_NODEFILE | gawk '//{print $1}'`
export SCRDIR=/work/$USER/scr
if [ ! -e $SCRDIR ]; then mkdir -p $SCRDIR; fi
rm -f $SCRDIR/*

cd $WORKDIR
rungms h2o-opt-freq 01 $NPROCS h2o-opt-freq.out  $SCRDIR
cp -p $SCRDIR/$OUTPUT $WORKDIR/
\end{verbatim}
}
\end{exampleblock}
}
\end{frame}

\begin{frame}[fragile]
\scriptsize{
% \begin{itemize}
% \item Geometry Optimzation and Freqeuncy calculation for water molecule
% \end{itemize}
\begin{columns}
\column{6.5cm}
\begin{block}{Sample Input}
\begin{verbatim}
 $CONTRL SCFTYP=RHF RUNTYP=OPTIMIZE 
   COORD=ZMT NZVAR=0 $END
 $STATPT OPTTOL=1.0E-5 HSSEND=.T. $END
 $BASIS GBASIS=N31 NGAUSS=6 
   NDFUNC=1 NPFUNC=1 $END
 $DATA
H2O OPT 
Cnv 2

O 
H 1 rOH 
H 1 rOH 2 aHOH

rOH=1.05
aHOH=104.5
 $END
\end{verbatim}
\end{block}
\column{5cm}
{\color{tigersblue}
\begin{alertblock}{Input Description}
\begin{verbatim}
 Job control data
   
 geometry search control
 6-31G** basis set  
   
 molecular data control
 Title
 Symmetry group and axis

 molecule description in 
    z-matrix


 variables

 end molecular data control
\end{verbatim}
\end{alertblock}
}
\end{columns}
}
\end{frame}

\begin{frame}[fragile]
\frametitle{\small Using NWCHEM on LONI Systems}
\begin{itemize}
\item Add +nwchem-5.1.1-intel-11.1-mvapich-1.1 (on Queenbee) to your .soft and resoft
\end{itemize}
{\tiny
\begin{exampleblock}{Job submission script}
{\color{white}
\begin{verbatim}
#!/bin/sh
#
#PBS -q checkpt
#PBS -M apacheco@cct.lsu.edu
#PBS -l nodes=1:ppn=4
#PBS -l walltime=0:30:00
#PBS -V
#PBS -o nwchem_h2o.out
#PBS -e nwchem_h2o.err
#PBS -N nwchem_h2o 

export EXEC=nwchem
export EXEC_DIR=/usr/local/packages/nwchem-5.1-mvapich-1.0-intel-10.1/bin/LINUX64/
export WORK_DIR=$PBS_O_WORKDIR
export NPROCS=`wc -l $PBS_NODEFILE |gawk '//{print $1}'` 

cd $WORK_DIR
mpirun_rsh -machinefile $PBS_NODEFILE -np $NPROCS $EXEC_DIR/$EXEC \
  $WORK_DIR/h2o-opt-freq.nw >& $WORK_DIR/h2o-opt-freq.nwo
\end{verbatim}
}
\end{exampleblock}
}
\end{frame}

\begin{frame}[fragile]
\tiny{
% \begin{itemize}
% \item Geometry Optimzation and Freqeuncy calculation for water molecule
% \end{itemize}
\begin{columns}
\column{6.5cm}
\begin{block}{Sample Input}
\begin{verbatim}
title "H2O"
echo
charge 0
geometry
zmatrix
O
H 1 r1
H 1 r1 2 a1
variables
r1 1.05 
a1 104.5
end
end
basis noprint
 * library 6-31G
end
dft
 XC b3lyp
 mult 1
end
task dft optimize
task dft energy
task dft freq
\end{verbatim}
\end{block}
\column{5cm}
{\color{tigersblue}
\begin{alertblock}{Input Description}
\begin{verbatim}
Job title
echo contents of input file
charge of molecule
geometry description in
z-matrix format



variables used with values
 

end z-matrix block
end geometry block
basis description
 
 
dft calculation options
 
 

job type


\end{verbatim}
\end{alertblock}
}
\end{columns}
}
\end{frame}

\end{document}

