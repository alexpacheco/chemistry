\documentclass[slidestop,mathserif,compress,xcolor=svgnames]{beamer}

\mode<presentation>
{  
  \setbeamertemplate{background canvas}[vertical shading][bottom=blue!10,top=blue!10]
  \setbeamertemplate{navigation symbols}{}%{\insertsectionnavigationsymbol}
  \usetheme{LSU}
}

\usepackage{pgf,pgfarrows,pgfnodes,pgfautomata,pgfheaps,pgfshade}
\usepackage{amsmath,amssymb,amsfonts,subfigure}
\usepackage{multirow}
\usepackage{tabularx}
\usepackage{booktabs}
\usepackage{colortbl}
\usepackage[normalem]{ulem}
\usepackage{tikz}
\usetikzlibrary{shapes,arrows}
\usetikzlibrary{calc}
\pgfdeclarelayer{background}
\pgfdeclarelayer{foreground}
\pgfsetlayers{background,main,foreground}
\usepackage[latin1]{inputenc}
\usepackage{colortbl}
\usepackage[english]{babel}
\usepackage{hyperref}
% \usepackage{movie15}
\hypersetup{
  pdftitle={Computational Chemistry packages},
  pdfauthor={Alexander B. Pacheco, User Services Consultant, Louisiana State University}
}
\usepackage{times}

\setbeamercovered{dynamic}
\beamersetaveragebackground{DarkBlue!2}
\beamertemplateballitem

\usepackage[english]{babel}
\usepackage[latin1]{inputenc}
\usepackage{times}
\usepackage{amsmath}
\usepackage[T1]{fontenc}
\usepackage{graphicx}
\definecolor{DarkGreen}{rgb}{0.0,0.3,0.0}
\definecolor{Blue}{rgb}{0.0,0.0,0.8} 
\definecolor{dodgerblue}{rgb}{0.1,0.1,1.0}
\definecolor{indigo}{rgb}{0.41,0.1,0.0}
\definecolor{seagreen}{rgb}{0.1,1.0,0.1}
\DeclareSymbolFont{extraup}{U}{zavm}{m}{n}
\DeclareMathSymbol{\vardiamond}{\mathalpha}{extraup}{87}
\newcommand*\up{\textcolor{green}{%
  \ensuremath{\blacktriangle}}}
\newcommand*\down{\textcolor{red}{%
  \ensuremath{\blacktriangledown}}}
\newcommand*\const{\textcolor{darkgray}%
  {\textbf{--}}}

\setbeamercolor{uppercol}{fg=white,bg=red!30!black}%
\setbeamercolor{lowercol}{fg=black,bg=red!15!white}%
\newenvironment{colorblock}[4]
{
\setbeamercolor{upperblock}{fg=#1,bg=#2}
\setbeamercolor{lowerblock}{fg=#3,bg=#4}
\begin{beamerboxesrounded}[upper=upperblock,lower=lowerblock,shadow=true]}
{\end{beamerboxesrounded}}

\newenvironment{colorblock2}[2]
{
\begin{beamerboxesrounded}[upper=#1,lower=#2,shadow=true]}
{\end{beamerboxesrounded}}
\newenvironment{ablock}[0]
{
\begin{beamerboxesrounded}[upper=uppercol,lower=lowercol,shadow=true]}
{\end{beamerboxesrounded}}
\newenvironment{bblock}[0]
{
\begin{beamerboxesrounded}[upper=uppercol1,lower=lowercol1,shadow=true]}
{\end{beamerboxesrounded}}
\newenvironment{eblock}[0]
{
\begin{beamerboxesrounded}[upper=uppercol2,lower=lowercol2,shadow=true]}
{\end{beamerboxesrounded}}

\title[Computational Chemistry packages]{Introduction to Computational Chemistry Packages}


\author[Alex Pacheco]{\large{Alexander~B.~Pacheco}}
       
\institute[LONI HPC Workshop, University of Louisiana - Lafayette, 11/09/2011] {\inst{}\footnotesize{User Services Consultant\\LSU HPC \& LONI\\sys-help@loni.org}}

\date[\hfill{Nov 9, 2011}]{\scriptsize{LONI HPC Workshop\\University of Louisiana - Lafayette\\Lafayette\\Nov 9, 2011}}
     
\subject{Talks}
% This is only inserted into the PDF information catalog. Can be left
% out. 

% If you have a file called "university-logo-filename.xxx", where xxx
% is a graphic format that can be processed by latex or pdflatex,
% resp., then you can add a logo as follows:

% Main Logo on bottom left
\pgfdeclareimage[height=0.55cm]{its-logo}{PUR_BLK_HOR}
\logo{\pgfuseimage{its-logo}}
% University Logo on top left
\pgfdeclareimage[height=0.55cm]{university-logo}{LSUGeauxPurp}
\tllogo{\pgfuseimage{university-logo}}
% Logo at top right
\pgfdeclareimage[height=0.6cm]{institute-logo}{LONI}
\trlogo{\pgfuseimage{institute-logo}}
% Logo at bottom right
\pgfdeclareimage[height=0.55cm]{hpc-logo}{LONI-2}
\brlogo{\pgfuseimage{hpc-logo}}

% Delete this, if you do not want the table of contents to pop up at
% the beginning of each subsection:
 \AtBeginSection[]
 {
   \begin{frame}<beamer>
    \frametitle{\small{Outline}}
     \footnotesize
     \tableofcontents[currentsection,currentsubsection]
   \end{frame}
 }

\begin{document}
\footnotesize

\frame{\titlepage}

\begin{frame}[label=toc,squeeze]
  \footnotesize
  \frametitle{\small{Outline}}
  \tableofcontents
  \tableofcontents[part=1]
  \tableofcontents[part=2]
  \tableofcontents[part=3]
  \tableofcontents[part=4]
  \tableofcontents[part=5]
\end{frame}

\section{Introduction}
\begin{frame}
 \frametitle{\small Introduction}
  \begin{block}{{\bf What is Computational Chemistry?}}
    \begin{itemize}
      \item {\bf Computational Chemistry} is a branch of chemistry that uses computer science to assist in solving chemical problems.
      \item Incorporates the results of theoretical chemistry into efficient computer programs.
      \item Application to single molecule, groups of molecules, liquids or solids.
      \item Calculates the structure and properties of interest.
      \item Computational Chemistry Methods range from
      \begin{enumerate}
	\footnotesize{
	\item Highly accurate ({\it Ab-initio},DFT) feasible for small systems
	\item Less accurate (semi-empirical)
	\item Very Approximate (Molecular Mechanics) large systems
	}
      \end{enumerate}
     \end{itemize}
  \end{block}
\end{frame}

\begin{frame}
 \frametitle{\small Introduction}
  \begin{columns}
    \column{11cm}
    \begin{exampleblock}{{\bf Theoretical Chemistry can be broadly divided into two main categories}}
      \begin{enumerate}
	\item Static Methods {\Large$\Rightarrow$} {\color{blue}Time-Independent Schr\"{o}dinger Equation}
	\begin{enumerate}
	  \footnotesize{
	  \item[]{\Large\hspace{3cm}${\color{DarkGreen}\hat{H}\Psi=E\Psi}$}
	  \item[] 
	  \item[$\vardiamond$] Quantum Chemical/\emph{Ab Initio} /Electronic Structure Methods
	  \item[$\vardiamond$] Molecular Mechanics
	  }
	\end{enumerate}
	\item Dynamical Methods {\Large$\Rightarrow$} {\color{blue}Time-Dependent Schr\"{o}dinger Equation}
	\begin{enumerate}
	  \footnotesize{
	  \item[]{\Large\hspace{3cm}${\color{DarkGreen}\imath\hbar\dfrac{\partial}{\partial t}\Psi = \hat{H}\Psi}$}
	  \item[] 
	  \item[$\vardiamond$] Classical Molecular Dynamics
	  \item[$\vardiamond$] Semi-classical and \textit{Ab-Initio} Molecular Dynamics
	  }
	\end{enumerate}
      \end{enumerate}
    \end{exampleblock}
  \end{columns}
\end{frame}

\section{Electronic Structure Methods}
\begin{frame}
  \frametitle{\small Electronic Structure Methods}
  \begin{block}{}
    \begin{itemize}
      \item \emph{Ab Initio}, meaning "from first principles", methods solve the Schr\"{o}dinger equation and does not rely on empirical or experimental data. 
      \item Begining with fundamental and physical properties, calculate how electrons and nuclei interact.
      \item The Schr\"{o}dinger equation can be solved exactly only for a few systems
      \begin{itemize}
	\footnotesize{
	\item[$\vardiamond$] Particle in a Box
	\item[$\vardiamond$] Rigid Rotor
	\item[$\vardiamond$] Harmonic Oscillator
	\item[$\vardiamond$] Hydrogen Atom
	}
      \end{itemize}
      \item For complex systems, \emph{Ab Initio} methods make assumptions to obtain approximate solutions to the  Schr\"{o}dinger equations and solve it numerically.
      \item "Computational Cost" of calculations increases with the accuracy of the calculation and size of the system.
    \end{itemize}
  \end{block}
\end{frame}

\begin{frame}
  \frametitle{\small Electronic Structure Methods}
  \fontsize{9}{10}{
  \begin{block}{What can we predict with Electronic Structure methods?}
    \begin{itemize}
      \item Molecular Geometry: Equilibrium and Transition State
      \item Dipole and Quadrupole Moments and polarizabilities
      \item Thermochemical data like Free Energy, Energy of reaction.
      \item Potential Energy surfaces, Barrier heights
      \item Reaction Rates and cross sections
      \item Ionization potentials (photoelectron and X-ray spectra) and Electron affinities
      \item Frank-Condon factors (transition probabilities, vibronic intensities)
      \item Vibrational Frequencies, IR and Raman Spectra and Intensities
      \item Rotational spectra
      \item NMR Spectra
      \item Electronic excitations and UV-VIS spectra
      \item Electron density maps and population analyses
      \item Thermodynamic quantities like partition function
    \end{itemize}
  \end{block}
  }
\end{frame}


\subsection{Wavefunction Methods}
\begin{frame}
  \frametitle{\small Ab Initio Wavefunction Methods}
  \begin{block}{Ab Initio Theory}
    \fontsize{8}{10}{
    \begin{itemize}
      \item {\color{blue}Born-Oppenheimer Approximation}: Nuclei are heavier than electrons and can be considered stationary with respect to electrons. Also know as "clamped nuclei" approximations and leads to idea of potential surface.
      \item[]{\hspace{3cm}$\Psi({\bf r},{\bf R}) = \psi_e({\bf r};{\bf R})\psi_n({\bf R})$}
      \item {\color{blue}Slater Determinants}: Expand the many electron wave function in terms of Slater determinants.
      \item[]{\hspace{3cm}$\displaystyle{\psi_e({\bf r};{\bf R}) = (N!)^{-1/2}\mathcal{A}\prod^{N}_{i=1}\phi_i({\bf r};{\bf R})}$}
      \item {\color{blue}Basis Sets}: Represent Slater determinants by molecular orbitals, which are linear combination of atomic-like-orbital functions i.e. basis sets
      \item[] {\hspace{3cm}$\displaystyle{\phi_i = \sum^N_{\mu=1}c_{\mu i}\chi_\mu}$}
      \item {\color{blue}Variational Theorem}: Expectation value of the energy of a trial wavefunction is always greater than or equal to the true energy
      \item[]{\hspace{3cm}$E_e = \langle\psi_e\mid\hat{H}_e\mid\psi_e\rangle \ge \varepsilon_0$}
    \end{itemize}
    }
  \end{block}
\end{frame}

\begin{frame}
  \frametitle{\small Ab Initio Wavefunction Methods}
  \begin{block}{\small The most popular classes of ab initio electronic structure methods:}
    \begin{itemize}
      \item Hartree-Fock methods
      \begin{itemize}
	\footnotesize{
	\item[$\vardiamond$] Hartree-Fock (HF)
	\begin{itemize}
	  \item[$\blacksquare$] Restricted Hartree-Fock (RHF): singlets
	  \item[$\blacksquare$] Unrestricted Hartree-Fock (UHF): higher multiplicities
	  \item[$\blacksquare$] Restricted open-shell Hartree-Fock (ROHF)
	\end{itemize}
	}
      \end{itemize}
      \item Post Hartree-Fock methods
      \begin{itemize}
	\footnotesize{
	\item[$\vardiamond$] M{\o}ller-Plesset perturbation theory (MPn)
	\item[$\vardiamond$] Configuration interaction (CI)
	\item[$\vardiamond$] Coupled cluster (CC)
	}
      \end{itemize}
      \item Multi-reference methods
      \begin{itemize}
	\footnotesize{
	\item[$\vardiamond$] Multi-configurational self-consistent field (MCSCF)
	\item[$\vardiamond$] Multi-reference configuration interaction (MRCI)
	\item[$\vardiamond$] n-electron valence state perturbation theory (NEVPT)
	\item[$\vardiamond$] Complete active space perturbation theory (CASPTn)
	}
      \end{itemize}
    \end{itemize}
  \end{block}
\end{frame}

\begin{frame}
  \frametitle{\small Hartree-Fock}
  \begin{enumerate}
    \item Wavefunction is written as a single determinant
    \begin{align*}
      \Psi = det(\phi_1,\phi_2,\cdots\phi_N)
    \end{align*}
    \item The electronic Hamiltonian can be written as
    \begin{align*}
      \hat{H} = \sum_ih(i) + \sum_{i>j}v(i,j)
    \end{align*}
    where $\displaystyle{h(i) = -\dfrac{1}{2}\nabla^2_i - \sum_{i,\alpha}\dfrac{Z_\alpha}{r_{i\alpha}}}$ and $v(i,j) =\dfrac{1}{r_{ij}}$
    \item The electronic energy of the system is given by:
    \begin{align*}
      E = \langle\Psi|\hat{H}|\Psi\rangle
    \end{align*}
    \item The resulting HF equations from minimization of energy by applying of variational theorem:
    \begin{align*}
      \hat{f}(x_1)\phi_i(x_1)= \varepsilon_i\phi_i(x_1)
    \end{align*}
    \item[] where $\varepsilon_i$ is the energy of orbital $\chi_i$ and the Fock operator $f$, is defined as
    \begin{align*}
      \hat{f}(x_1) = \hat{h}(x_1) + \sum_j\left[\hat{J}_j(x_1)-\hat{K}_j(x_1)\right]
    \end{align*}
  \end{enumerate}
\end{frame}

\begin{frame}
  \frametitle{\small Hartree-Fock}
  \begin{enumerate}
    \item $\hat{J}_j\Rightarrow$ Coulomb operator $\Rightarrow$ average potential at $x$ due to charge distribution from electron in orbital $\phi_i$ defined as
    \begin{align*}
      \hat{J}_j(x_1)\phi_i(x_1) = \left[\int\dfrac{\phi^\ast_j(x_2)\phi_j(x_2)}{r_{12}}dx_2\right]\phi_i(x_1)
    \end{align*}
    \item $\hat{K}_j\Rightarrow$ Exchange operator $\Rightarrow$ Energy associated with exchange of electrons $\Rightarrow$ No classical interpretation for this term.
    \begin{align*}
      \hat{K}_j(x_1)\phi_i(x_1) = \left[\int\dfrac{\phi^\ast_j(x_2)\phi_i(x_2)}{r_{12}}dx_2\right]\phi_j(x_1)
    \end{align*}
    \item The Hartree-Fock equation are solved numerically or in a space spanned by a set of basis functions (Hartree-Fock-Roothan equations)
    \begin{columns}
      \column{4cm}
      \vspace{-0.5cm}
      \begin{align*}
	\phi_i &= \sum^K_{\mu=1}C_{\mu i}\tilde{\phi}_\mu\\
	\sum_\nu F_{\mu\nu}C_{\nu i} &= \varepsilon_i\sum_\nu S_{\mu\nu}C_{\nu i}
      \end{align*}
      \column{4cm}
      \vspace{-0.35cm}
      \begin{align*}
	S_{\mu\nu} &= \int dx_1\tilde{\phi}^\ast_\mu(x_1)\tilde{\phi}_\nu(x_1)\\
	F_{\mu\nu} &= \int dx_1\tilde{\phi}^\ast_\mu(x_1)\hat{f}(x_1)\tilde{\phi}_\nu(x_1)
      \end{align*}
    \end{columns}
    \begin{align*}
      \hspace{-1cm}{\bf FC}&={\bf SC}{\boldsymbol\varepsilon}
    \end{align*}
  \end{enumerate}
\end{frame}

\begin{frame}
  \frametitle{\small Hartree-Fock}
  \fontsize{8}{10}{
  \begin{columns}
    \column{12cm}
    \begin{enumerate}
      \item The Hartree-Fock-Roothan equation is a pseudo-eigenvalue equation
      \item ${\bf C}$'s are the expansion coefficients for each orbital expressed as a linear combination of the basis function.
      \item Note: ${\bf C}$ depends on ${\bf F}$ which depends on ${\bf C}\Rightarrow$ need to solve self-consistently.
      \item Starting with an initial guess orbitals, the HF equations are solved iteratively or self consistently (Hence HF procedure is also known as self-consistent field or SCF approach) obtaining the best possible orbitals that minimize the energy.
    \end{enumerate}
    \vspace{-0.3cm}
    \begin{block}{\footnotesize SCF procedure}
      \begin{enumerate}
	\item Specify molecule, basis functions and electronic state of interest
	\item Form overlap matrix ${\bf S}$
	\item Guess initial MO coefficients ${\bf C}$
	\item Form Fock Matrix ${\bf F}$
	\item Solve ${\bf FC}={\bf SC}{\boldsymbol\varepsilon}$
	\item Use new MO coefficients ${\bf C}$ to build new Fock Matrix ${\bf F}$
	\item Repeat steps 5 and 6 until ${\bf C}$ no longer changes from one iteration to the next.
      \end{enumerate}
    \end{block}
  \end{columns}
  }
\end{frame}

\begin{frame}
  \frametitle{\small SCF Flow Chart}
  \scriptsize{
  \tikzstyle{decision} = [ellipse, draw, fill=red!50!yellow,text width=8em, text badly centered, node distance=2.5cm, inner sep=0pt, minimum height=5em]
  \tikzstyle{block} = [rectangle, draw, fill=green!50!black,text width=10em, text centered, rounded corners, minimum height=4em]
  \tikzstyle{line} = [draw, -latex']
  \tikzstyle{cloud} = [draw, ellipse, fill=green!20,node distance=4cm, minimum height=2em]
  \begin{tikzpicture}[node distance = 2cm, auto]
    % Place nodes
    \node [block,fill=gray] (overlap){Form overlap matrix\\{\bf S}};
    \node [block, right of=overlap, node distance=4cm,fill=gray] (input) {Input Coordinates,\\ Basis sets etc};
    \node [block, below of=overlap,fill=gray] (guess) {Guess Initial \\MO Coefficients\\{\bf C}};
    \node [block, right of=guess, node distance=4cm] (fock) {Form Fock Matrix\\{\bf F}};
    \node [block, below of=fock](solve){Solve\\{\bf FC$^\prime$}={\bf SC$^\prime$}$\boldsymbol\varepsilon$};
    \node [decision, below of=solve,node distance=2cm](decide){SCF Converged?\\|{\bf C}-{\bf C}$^\prime$|$\le\epsilon_{tol}$};
    \node [block, right of=solve, node distance=4cm, fill=red!50!yellow!60!green](update){Update {\bf C}\\{\bf C}={\bf C}$^\prime$};
    \node [block, left of=decide, node distance=4cm,fill=red!80!black] (stop) {Calculate Properties\\END};
    % Draw edges
    \path [line] (input) -- (overlap);
    \path [line] (overlap) -- (guess);
    \path [line] (guess) -- (fock);
    \path [line] (fock) -- (solve);
    \path [line] (solve) -- (decide);
    \path [line] (decide) -- node {yes}(stop);
    \path [line] (decide) -| node[near start]{no} (update);
    \path [line] (update) |- (fock);
  \end{tikzpicture}
  }
\end{frame}

\begin{frame}
  \frametitle{\small Post Hartree-Fock Methods}
  \fontsize{8}{10}{
  \begin{columns}
    \column{12cm}
    \vspace{-0.75cm}
    \begin{block}{}
      \begin{itemize}
	\item[$\vardiamond$]Methods that improve the Hartree-Fock results by accounting for the correlation energy are known as {\bf Post Hartree-Fock methods}
	\item[$\vardiamond$]The starting point for most Post HF methods is the Slater Determinant obtain from Hartree-Fock Methods.
	\item[$\vardiamond$]{\bf Configuration Interaction (CI) methods}: Express the wavefunction as a linear combination of Slater Determinants with the coeffcients obtained variationally
	\item[]{\hspace{4cm}$|\Psi\rangle = \sum_ic_i|\Psi_i\rangle$}
	\item[$\vardiamond$]{\bf Many Body Perturbation Theory}: Treat the HF determinant as the zeroth order solution with the correlation energy as a pertubation to the HF equation.
%	\begin{align*}
	  \item[] {\hspace{4.5cm}$\hat{H} = \hat{H}_0 + \lambda\hat{H}^\prime$}
	  \item[] {\hspace{3.5cm}$\varepsilon_i = E^{(0)}_i + \lambda E^{(1)}_i + \lambda^2E^{(2)}_i + \cdots$}
 	  \item[] {\hspace{3cm}$|\Psi_i\rangle = |\Psi^{(0)}_i\rangle + \lambda|\Psi^{(1)}_i\rangle + \lambda^2|\Psi^{(2)}_i\rangle\cdots$}
%	\end{align*}
	\item[$\vardiamond$]{\bf Coupled Cluster Theory}:The wavefunction is written as an exponential ansatz
	\item[]{\hspace{4cm}$|\Psi\rangle = e^{\hat{T}}|\Psi_0\rangle$}
	\item[]where $|\Psi_0\rangle$ is a Slater determinant obtained from HF calculations and $\hat{T}$ is an excitation operator which when acting on $|\Psi_0\rangle$ produces a linear combination of excited Slater determinants.
      \end{itemize}
    \end{block}
  \end{columns}
  }
\end{frame}

\begin{frame}[bg=lightgray]
  \frametitle{\small Scaling}
  \begin{center}
    \begin{tikzpicture}
      \node (tbl) {
      \begin{tabularx}{.65\textwidth}{cc}
	\arrayrulecolor{tigersblue}
	\textbf{\color{tigersgold}Scaling Behavior} & \textbf{\color{tigersgold}Method(s)} \\
	$N^3$ & DFT\rule{0pt}{2.5ex} \\
	\midrule
	$N^4$ & HF \\
	\midrule
	$N^5$ & MP2 \\
	\midrule
	$N^6$ & MP3,CISD,CCSD,QCISD \\
	\midrule
	$N^7$ & MP4,CCSD(T),QCISD(T) \\
	\midrule
	$N^8$ & MP5,CISDT,CCSDT \\
	\midrule
	$N^9$ & MP6 \\
	\midrule
	$N^{10}$ & MP7,CISDTQ,CCSDTQ \\
	\midrule
	$N!$ & Full CI \\[0.5ex]
      \end{tabularx}};
      \begin{pgfonlayer}{background}
	\draw[rounded corners,top color=tigerspurple,bottom color=tigerspurple,draw=tigerspurple] ($(tbl.north west)+(0.14,0)$) rectangle ($(tbl.north east)-(0.13,0.9)$);
	\draw[rounded corners,top color=tigersgold,bottom color=tigerspurple, middle color=tigerspurple,draw=tigerspurple] ($(tbl.south west) +(0.12,0.5)$) rectangle ($(tbl.south east)-(0.12,0)$);
	\draw[top color=tigersgold,bottom color=tigersgold,draw=tigersgold] ($(tbl.north east)-(0.13,0.6)$) rectangle ($(tbl.south west)+(0.13,0.2)$);
      \end{pgfonlayer}
    \end{tikzpicture}
    \small
    \\
    {N = Number of Basis Functions}
  \end{center}
\end{frame}


\subsection{Density Functional Methods}
\begin{frame}
  \frametitle{\small Density Functional Theory}
%   \begin{block}{}
    \begin{itemize}
      \item Density Functional Theory (DFT) is an alternative to wavefunction based electronic structure methods of many-body systems such as Hartree-Fock and Post Hartree-Fock.
      \item In DFT, the ground state energy is expressed in terms of the total electron density.
      \item[]{\hspace{4cm}$\rho_0(r) = \langle\Psi_0|\hat{\rho}|\Psi_0\rangle$}
%       \item We again start with Born-Oppenheimer approximation and write the electronic Hamiltonian as
%       \item[]{\hspace{4cm}$\hat{H} = \hat{F} + \hat{V}_{ext}$}
%       \item[]where $\hat{F}$ is the sum of the kinetic energy of electrons and the electron-electron interaction and $\hat{V}_{ext}$ is some external potential.
%       \item Modern DFT methods result from the Hohenberg-Kohn theorem
      \begin{block}{Hohenberg-Kohn theorem}
      \begin{enumerate}
	{\footnotesize
	\item The external potential $V_{ext}$, and hence total energy is a unique functional of the electron density $\rho(r)$
	\item[]\hspace{3cm}${\mathrm {Energy}} = \dfrac{\langle\Psi\mid\hat{H}\mid\Psi\rangle}{\langle\Psi\mid\Psi\rangle} \equiv E[\rho]$
	\item The ground state energy can be obtained variationally, the density that minimizes the total energy is the exact ground state density
	\item[]\hspace{3cm}$E[\rho] > E[\rho_0], {\mathrm if }\rho \ne \rho_0$
	}
      \end{enumerate}
      \end{block}
    \end{itemize}
%   \end{block}
\end{frame}


\begin{frame}
  \frametitle{\small DFT}
%   \vspace{-0.4cm}
%   \begin{block}{}
    \fontsize{8}{10}{
    \begin{itemize}
      \item If density is known, then the total energy is:
      \item[]\hspace{3cm}$E[\rho]=T[\rho]+V_{ne}[\rho] + J[\rho] + E_{nn} + E_{xc}[\rho]$
      \item[]where 
      \item[]\hspace{3cm}$\displaystyle{E_{nn}[\rho] = \sum_{A>B}\dfrac{Z_AZ_B}{R_{AB}}}$
      \item[]\hspace{3cm}$\displaystyle{V_{ne}[\rho] = \int\rho(r)V_{ext}(r)dr}$
      \item[]\hspace{3cm}$\displaystyle{J[\rho] = \dfrac{1}{2}\int\dfrac{\rho(r_1)\rho(r_2)}{r_{12}}dr_1dr_2}$
      \item[] and two unknowns, the kinetic energy functional $T[\rho]$ and the exchange-correlation functional $E_{xc}[\rho]$
%     \end{itemize}
%     }
%   \end{block}
%   \vspace{-0.1cm}
%   \begin{block}{}
%     \fontsize{8}{10}{
%     \begin{itemize}
      \item To calculate $T[\rho]$, Kohn and Sham introduced the concept of Kohn-Sham orbitals which are eigenvectors of the Kohn-Sham equation 
      \item[]\hspace{3cm}$\left(-\dfrac{1}{2}\nabla^2+v_{\rm eff}(r)\right)\phi_{i}(r)=\varepsilon_{i}\phi_{i}(r)$
      \item[]Here, $\varepsilon_i$ is the orbital energy of the corresponding Kohn-Sham orbital, $\phi_i$, and the density for an ''N''-particle system is
      \item[]\hspace{3cm}$\displaystyle{\rho(r)=\sum_i^N |\phi_{i}(r)|^2}$
    \end{itemize}
    }
%   \end{block}
\end{frame}


\begin{frame}
  \frametitle{\small DFT}
%   \begin{block}{}
    \fontsize{8}{10}{
    \begin{itemize}
      \item The total energy of a system is 
      \item[]\hspace{3cm}$\displaystyle{E[\rho]  = T_s[\rho] + \int dr\ v_{\rm ext}(r)\rho(r) + V_{H}[\rho] + E_{\rm xc}[\rho]}$
      \item $T_s$ is the Kohn-Sham kinetic energy
      \item[]\hspace{3cm}$\displaystyle{T_s[\rho]=\sum_{i=1}^N\int dr\ \phi_i^*(r)\left(-\frac{1}{2}\nabla^2\right)\phi_i(r)}$
      \item $v_{\rm ext}$ is the external potential acting on the interacting system (at minimum, for a molecular system, the electron-nuclei interaction), $V_H$ is the Hartree (or Coulomb) energy,
      \item[]\hspace{3cm}$\displaystyle{V_{H}=\dfrac{1}{2}\int drdr^\prime\  \dfrac{\rho(r)\rho(r^\prime)}{|r-r^\prime|}}$
      \item[]and $E_{xc}$ is the exchange-correlation energy. 
      \item The Kohn-Sham equations are found by varying the total energy expression with respect to a set of orbitals to yield the Kohn-Sham potential as
      \item[]\hspace{3cm}$\displaystyle{v_{\rm eff}(r) = v_{\rm ext}(r) + \int \dfrac{\rho(r^\prime)}{|r-r^\prime|}dr^\prime + \dfrac{\delta E_{\rm xc}[\rho]}{\delta\rho(r)}}$
      \item[]where the last term $\displaystyle{v_{\rm xc}(r)\equiv\dfrac{\delta E_{\rm xc}[\rho]}{\delta\rho(r)}}$ is the exchange-correlation potential.
    \end{itemize}
    }
%   \end{block}
\end{frame}

\begin{frame}
  \frametitle{\small DFT}
%   \begin{block}{}
    \fontsize{8}{10}{
    \begin{itemize}
      \item The exchange-correlation potential, and the corresponding energy expression, are the only unknowns in the Kohn-Sham approach to density functional theory. 
      \item There are many ways to approximate this functional $E_{\rm xc}$, generally divided into two separate terms
      \item[]\hspace{3cm}$E_{\rm xc}[\rho] = E_{\rm x}[\rho] + E_{\rm c}[\rho]$
      \item[]where the first term is the exchange functional while the second term is the correlation functional.
      \item Quite a few research groups have developed the exchange and correlation functionals which are fit to empirical data or data from explicity correlated methods.
      \item Popular DFT functionals (according to a recent poll)
      \begin{enumerate}
	\item[$\vardiamond$] {\scriptsize PBE0 (PBEPBE), B3LYP, PBE, BP86, M06-2X, B2PLYP, B3PW91, B97-D, M06-L, CAM-B3LYP}
	\item[$\blacksquare$]  {\scriptsize \url{http://www.marcelswart.eu/dft-poll/index.html}}
	\item[$\blacksquare$]  {\scriptsize \url{http://www.ccl.net/cgi-bin/ccl/message-new?2011+02+16+009}}
      \end{enumerate}
    \end{itemize}
    }
%   \end{block}
\end{frame}


\begin{frame}
  \frametitle{\small DFT Flow Chart}
  \scriptsize{
  \tikzstyle{decision} = [ellipse, draw, fill=red!50!yellow,text width=10em, text badly centered, node distance=2.5cm, inner sep=0pt, minimum height=5em]
  \tikzstyle{block} = [rectangle, draw, fill=green!50!black,text width=12em, text centered, rounded corners, minimum height=4em]
  \tikzstyle{line} = [draw, -latex']
  \tikzstyle{cloud} = [draw, ellipse, fill=green!20,node distance=4cm, minimum height=2em]
  \begin{tikzpicture}[node distance = 2cm, auto]
    \node [block,fill=gray](initial){Select initial\\$\displaystyle{\boldsymbol{\rho^{(n)}(r)=\sum_i^N |\phi^{(n)}_{i}(r)|^2}}$};
    \node [block, right of=initial,node distance=4cm](KSoperator){Contruct Kohn-Sham Operator\\$\boldsymbol{ \hat{h}^{(n)}_{KS}=-\dfrac{1}{2}\nabla^2+v^{(n)}_{\rm eff}(r)}$};
    \node [block, below of=KSoperator, text width=15.5em,node distance=3cm](solve){Solve\\$\boldsymbol{\hat{h}^{(n)}_{KS}\phi^{(n+1)}_{i}(r)=\varepsilon^{(n+1)}_{i}\phi^{(n+1)}_{i}(r)}$\\$\displaystyle{\boldsymbol{\rho^{(n+1)}(r)=\sum_i^N |\phi^{(n+1)}_{i}(r)|^2}}$};
%    \node [block, below of=solve, text width=14em](density){Get new density\\$\rho^{(n+1)}(r)=\sum_i^N |\phi^{(n+1)}_{i}(r)|^2$};
    \node [decision, below of=solve,node distance=3cm](decide){Density Converged?\\$\boldsymbol{|\rho^{(n+1)}-\rho^{(n)}|\le\epsilon_{tol}}$};
    \node [block, right of=solve, node distance=4cm, text width=6em, fill=red!50!yellow!60!green](update){Set\\$\boldsymbol{\rho^{(n)}\rightarrow\rho^{(n)}}$\\$\boldsymbol{n\rightarrow n+1}$};
    \node [block, left of=decide, node distance=4cm, text width=5em,fill=red!80!black] (stop) {Calculate Properties\\END};

    \path [line](initial) -- (KSoperator);
    \path [line](KSoperator) -- (solve);
    \path [line](solve) -- (decide);
%    \path [line](density) -- (decide);
    \path [line](decide) -- node[near start]{yes}(stop);
    \path [line](decide) -| node[near start]{no}(update);
    \path [line](update) |- (KSoperator);
  \end{tikzpicture}
  }
\end{frame}


\subsection{Basis Sets}
\begin{frame}
  \frametitle{\small Basis Sets}
  \begin{itemize}
    \item %Most electronic structure codes use either a 
      Slater type orbital (STO) or Gaussian type orbital (GTO) to describe the AO's
    \begin{align*}
      \phi^{\mathrm{STO}}(r) &= x^ly^mz^ne^{-\zeta r}\\
      \phi^{\mathrm{GTO}}(r) &= x^ly^mz^ne^{-\xi r^2}
    \end{align*}
    where $L=l+m+n$ is the total angular momentun and $\zeta,\xi$ are orbital exponents. 
  \end{itemize}
  \begin{center}
    \includegraphics[width=5cm,keepaspectratio=true,clip=true]{STO-GTO}
  \end{center}
\end{frame}

\begin{frame}
  \frametitle{\small STO or GTO? Which One?}
  \begin{columns}
    \column{5.5cm}
    \vspace{-0.2cm}
    \begin{block}{Why STO}
      \begin{itemize}
	\item Correct cups at $r\rightarrow0$
	\item Desired decay at $r\rightarrow\infty$ 
	\item Correctly mimics H orbitals
	\item Natural Choice for orbitals
	\item Computationally expensive to compute integrals and derivatives.
      \end{itemize}
    \end{block}
    \column{5.5cm}
    \vspace{-0.2cm}
    \begin{block}{Why GTO}
      \begin{itemize}
	\item Wrong behavior at $r\rightarrow0$ and $r\rightarrow\infty$
	\item Gaussian $\times$ Gaussian = Gaussian
	\item Analytical solutions for most integrals and derivatives.
	\item Computationally less expensive than STO's
      \end{itemize}
    \end{block}
  \end{columns}
\end{frame}

\begin{frame}
  \begin{block}{Pople family basis set}
    \begin{enumerate}
      \item Minimal Basis: STO-nG
      \begin{enumerate}
	\footnotesize{
	\item[$\vardiamond$]Each atom optimized STO is fit with n GTO's
	\item[$\vardiamond$]Minimum number of AO's needed
	}
      \end{enumerate}
      \item Split Valence Basis: 3-21G,4-31G, 6-31G
      \begin{enumerate}
	\footnotesize{
	\item[$\vardiamond$]Contracted GTO's optimized per atom.
	\item[$\vardiamond$]Valence AO's represented by 2 contracted GTO's
	}
      \end{enumerate}
      \item Polarization: Add AO's with higher angular momentum (L)
      \begin{enumerate}
	\footnotesize{
	\item[$\vardiamond$]3-21G* or 3-21G(d),6-31G* or 6-31G(d),6-31G** or 6-31G(d,p)
	}
      \end{enumerate}
      \item Diffuse function: Add AO with very small exponents for systems with diffuse electron densities
      \begin{enumerate}
	\footnotesize{
	\item[$\vardiamond$]6-31+G*, 6-311++G(d,p)
	}
      \end{enumerate}
    \end{enumerate}
  \end{block}
\end{frame}

\begin{frame}
  \begin{block}{Correlation consistent basis set}
    \begin{enumerate}
      \item[$\vardiamond$]Family of basis sets of increasing sizes.
      \item[$\vardiamond$]Can be used to extrapolate basis set limit.
      \item[$\vardiamond$]cc-pVDZ: Double Zeta(DZ) with d's on heavy atoms, p's on H
      \item[$\vardiamond$]cc-pVTZ: triple split valence with 2 sets of d's and 1 set of f's on heavy atom, 2 sets of p's and 1 set of d's on H
      \item[$\vardiamond$]cc-pVQZ, cc-pV5Z, cc-pV6Z
      \item[$\vardiamond$]can be augmented with diffuse functions: aug-cc-pVXZ (X=D,T,Q,5,6)
    \end{enumerate}
  \end{block}
\end{frame}

\begin{frame}
  \begin{block}{Pseudopotentials or Effective Core Potentials}
    \begin{enumerate}
      \item[$\vardiamond$]All Electron calculations are prohibitively expensive.
      \item[$\vardiamond$]Only valence electrons take part in bonding interaction leaving core electrons unaffected.\item[$\vardiamond$]Effective Core Potentials (ECP) a.k.a Pseudopotentials describe interactions between the core and valence electrons.
      \item[$\vardiamond$]Only valence electrons explicitly described using basis sets.
      \item[$\vardiamond$] Pseudopotentials commonly used
      \begin{enumerate}
	\footnotesize{
	\item[$\blacksquare$]Los Alamos National Laboratory: LanL1MB and LanL2DZ
	\item[$\blacksquare$]Stuttgard Dresden Pseudopotentials: SDDAll can be used.
	\item[$\blacksquare$]Stevens/Basch/Krauss ECP's: CEP-4G,CEP-31G,CEP-121G
	}
      \end{enumerate}
      \item[$\vardiamond$]Pseudopotential basis are "ALWAYS" read in pairs
      \begin{enumerate}
	\footnotesize{
	\item[$\blacksquare$]Basis set for valence electrons
	\item[$\blacksquare$]Parameters for core electrons
	}
      \end{enumerate}
    \end{enumerate}
  \end{block}
\end{frame}

\section{Molecular Dynamics}
\begin{frame}
  \frametitle{\small Molecular Dynamics}
  \begin{block}{}
    \begin{itemize}
      \item Molecular Dynamics is a computer simulation of physical movements of atoms and molecules.
      \item The atoms and molecules are allowed to interact and their trajectories are determined by numerically solving the Newton's equation of motion.
      \item[] {\hspace{3.5cm}${\color{DarkGreen}{\bf F}_i = -{\boldsymbol \nabla}_iV = m_i{\bf a}_i}$}
      \item The fundamental principle of Molecular Dynamics is the {\color{Blue}\textbf{Ergodic Hypothesis}:} \textit{the time average of a process is equal to an average over the statistical ensemble}.
      \item {\color{Blue}Ab-Initio Molecular Dynamics:} Atoms and Molecules move in an average electronic potential obtained from ab-initio methods.
      \item {\color{Blue}Classical Molecular Dynamics:} Atoms and Molecules move in an empirical potential or one obtained from Molecular Mechanics force fields.
    \end{itemize}
  \end{block}
\end{frame}

\begin{frame}
  \frametitle{\small General Schematic for MD Program}
  \vspace{-0.5cm}
  \begin{columns}
    \column{4.5cm}
    \begin{colorblock}{white}{blue!30!black}{black}{blue!15!white}{}
      \begin{enumerate}
	\item Setup: Read input parameters
	\item Initialize: Obtain initial positions and velocities
	\item Evaluate: Potential Energy and Forces on nuclei
	\item Propagate nuclei using an appropriate time integration algorithm
	\item Check if Dynamics is complete. 
	\item If incomplete update variables and goto Step 3.
	\item If complete end dynamics or carry out all required analysis.
      \end{enumerate}
    \end{colorblock}
    \column{7cm}
    \begin{colorblock}{white}{blue!30!black}{black}{blue!15!white}{}
      \scriptsize{
      \tikzstyle{decision} = [ellipse, draw, fill=red!50!yellow,text width=6em, text badly centered, node distance=1.5cm, inner sep=0pt, minimum height=4.5em]
      \tikzstyle{block} = [rectangle, draw, fill=green!50!black, text width=10em, text centered, rounded corners, minimum height=3em]
      \tikzstyle{line} = [draw, -latex']
      \tikzstyle{cloud} = [draw, ellipse, fill=green!20,node distance=1.5cm, minimum height=2em]
      \begin{tikzpicture}[node distance = 1.35cm, auto]
	\node [block,fill=gray](setup){{\bf 1. Setup}\\Read Input Parameters};
	\node [block, below of=setup, fill=gray](initial){{\bf 2. Initialize}\\Positions \& velocities};
	\node [block, below of=initial](solve){{\bf 3. Evaluate}\\Potential \& Forces};
	\node [block, below of=solve](integrate){{\bf 4. Propagate}\\Find new \\Potential \& Forces};
	\node [decision ,left of=integrate,node distance=2.65cm, fill=red!50!yellow!60!green,text width=4.75em, minimum height=2.5em](update){\bf 6. Update};
	\node [decision, below of=integrate,node distance=1.5cm](decide){{\bf 5. Dynamics Complete?}};
	\node [decision, right of=decide, text width=4em, node distance=2.65cm,minimum height=2em, fill=red!80!black] (stop) {\bf 7. END};

	\path [line](setup) -- (initial);
	\path [line](initial) -- (solve);
	\path [line](solve) -- (integrate);
	\path [line](integrate) -- (decide);
	\path [line](decide) -- node[near end]{yes}(stop);
	\path [line](decide) -| node[near start]{no}(update);
	\path [line](update) |- (solve);
      \end{tikzpicture}
      }
    \end{colorblock}
  \end{columns}
\end{frame}

\subsection{Ab-Initio Molecular Dynamics}
\begin{frame}
 \frametitle{\small Ab-Initio Molecular Dynamics}
  \begin{columns}
    \column{11.5cm}
    \vspace{-0.5cm}
    \begin{block}{}
      \begin{itemize}
	\item {\color{blue}Born-Oppenheimer Molecular Dynamics (BOMD):} Electronic potential and nuclear forces are obtain on-the-fly during the dynamics.
	\item[] \hspace{3cm}$\displaystyle{M_I\ddot{\bf R}_I = -{\boldsymbol \nabla}_I \min_{\Phi_0}\langle\Phi|H_e|\Psi\rangle}$
	\item {\color{blue}Extended Lagrangian Molecular Dynamics (ELMD):} Extend the Lagrangian by adding kinetic energy of fictitious particles and obtain their equation of motions from Euler-Lagrange equations.
	\item[]$\hspace{2cm}\mathcal{L} = \hat{T} - \hat{V}\hspace{2cm}\dfrac{d}{dt}\dfrac{\partial\mathcal{L}}{\partial\dot{\bf R}_I} = \dfrac{\partial\mathcal{L}}{\partial{\bf R}_I}$
	\begin{enumerate}
	  {\scriptsize
	  \item {\color{tigerspurple}Car-Parrinello Molecular Dynamics (CPMD)}
	  \item[]$\displaystyle{\mathcal{L}_\mathrm{CPMD} = \sum_I\dfrac{1}{2}M_I\dot{\bf R}^2_I + \sum_i\dfrac{1}{2}\mu_i\langle\dot{\phi_i}|\dot{\phi_i}\rangle - \langle\Phi_0|H_e|\Phi_0\rangle + \sum_{i,j}{\boldsymbol \Lambda}_{i,j}\left(\langle\phi_i|\phi_j\rangle -\delta_{i,j}\right)}$
	  \item {\color{tigerspurple}Atom-centered Density Matrix Propagation (ADMP)}
	  \item[] $\displaystyle{\mathcal{L}_\mathrm{ADMP} = \dfrac{1}{2}\mathrm{Tr}({\bf V}^T{\bf M}{\bf V}) + \dfrac{1}{2}\mu\mathrm{Tr}(\dot{\bf P}\dot{\bf P}) - E({\bf R},{\bf P}) - \mathrm{Tr}[{\boldsymbol\Lambda}({\bf PP}-{\bf P})]}$
	  \item {\color{tigerspurple}Curvy-steps ELMD (cs-ELMD)}
	  \item[] $\displaystyle{\mathcal{L}_\mathrm{csELMD} = \sum_I\dfrac{1}{2}M_I\dot{\bf R}^2_I + \dfrac{1}{2}\mu\sum_{i<j}\dot{\Delta}_{ij} - E({\bf R},{\bf P}) ;\hspace{0.2cm} {\bf P}(\lambda) = e^{\lambda\Delta}{\bf P}(0)e^{-\lambda\Delta}}$
	  }
	\end{enumerate}
      \end{itemize}
    \end{block}
  \end{columns}
\end{frame}

\subsection{Classical Molecular Dynamics}
\begin{frame}
  \frametitle{\small Classical Molecular Dynamics}
  \begin{block}{}
    \begin{itemize}
%       \item Nuclei move in an empirical potential or one obtained from Molecular Mechanics.
      \item The potential energy of all systems in molecular mechanics is calculated using force fields. 
      \item Molecular mechanics can be used to study small molecules as well as large biological systems or material assemblies with many thousands to millions of atoms.
      \item All-atomistic molecular mechanics methods have the following properties:
      \begin{itemize}
	\footnotesize{
	\item[$\vardiamond$] Each atom is simulated as a single particle
	\item[$\vardiamond$] Each particle is assigned a radius (typically the van der Waals radius), polarizability, and a constant net charge (generally derived from quantum calculations and/or experiment)
	\item[$\vardiamond$] Bonded interactions are treated as "springs" with an equilibrium distance equal to the experimental or calculated bond length
	}
      \end{itemize}
      \item The exact functional form of the potential function, or force field, depends on the particular simulation program being used. 
    \end{itemize}
  \end{block}
\end{frame}

\begin{frame}
  \frametitle{\small General form of Molecular Mechanics equations}
  {\scriptsize    
  \begin{columns}
    \column{6.5cm}
    \begin{colorblock}{white}{blue!30!black}{black}{blue!8!white}{General form of Molecular Mechanics equations}
      \begin{align*}
	E&={\color{red}E_{\mathrm{bond}}}+{\color{blue}E_{\mathrm{angle}}} + {\color{tigerspurple}E_{\mathrm{torsion}}} + {\color{DarkGreen}E_{\mathrm{vdW}}} + {\color{indigo}E_{\mathrm{elec}}}\\
	&={\color{red}\frac{1}{2}\sum_{\mathrm{bonds}}K_b(b-b_0)^2 \hspace{1.6cm}\mathrm{Bond}}\\
	&+{\color{Blue}\frac{1}{2}\sum_{\mathrm{angles}}K_\theta(\theta-\theta_0)^2 \hspace{1.6cm}\mathrm{Angle}}\\
	&+{\color{tigerspurple}\frac{1}{2}\sum_{\mathrm{dihedrals}}K_\phi\left[1+\cos(n\phi)\right]^2 \hspace{0.55cm}\mathrm{Torsion}}\\
	&+\sum_{\mathrm{non bonds}}\left\{ \begin{array}{l}
	{\color{DarkGreen}\left[\left(\dfrac{\sigma}{r}\right)^{12}-\left(\dfrac{\sigma}{r}\right)^{6}\right] \mathrm{van\,der\,Waals}}\\
	+ {\color{indigo}\dfrac{q_1q_2}{Dr} \hspace{1.6cm}\mathrm{Electrostatics}}
       \end{array} \right.
      \end{align*}
    \end{colorblock}
    \column{5.2cm}
     \vspace{0.38cm}
    \begin{colorblock}{white}{blue!30!black}{black}{blue!8!white}{}
      \includegraphics[width=5cm,keepaspectratio=true,clip=true]{MM_PEF}
      \let\thefootnote\relax\footnotetext{\tiny Picture taken from }
      \let\thefootnote\relax\footnotetext{\tiny http://en.wikipedia.org/wiki/Molecular\_mechanics}
    \end{colorblock}
  \end{columns}
  }
\end{frame}

\section{Computational Chemistry Packages}
\begin{frame}
  \frametitle{\small Software Installed on LONI Systems}
  \scriptsize{
  \begin{center}
    \begin{tikzpicture}
      \node (tbl) {
      \begin{tabularx}{0.85\textwidth}{ccccccc}
	\arrayrulecolor{tigersgold}
	\textcolor{white}{\textbf{Software} }& \textcolor{white}{\textbf{QB}} &\textcolor{white}{\textbf {Eric}} & \textcolor{white}{\textbf{Louie}} & \textcolor{white}{\textbf{Oliver}} & \textcolor{white}{\textbf{Painter}} & \textcolor{white}{\textbf{Poseidon}}\\
	Amber\rule{0pt}{3.5ex} & \checkmark & \checkmark & \checkmark & \checkmark & \checkmark & \checkmark \\ 
	Desmond & \checkmark &  &  &  &  &\\
	DL\_Poly & \checkmark & \checkmark & \checkmark & \checkmark & \checkmark & \checkmark \\
	Gromacs & \checkmark & \checkmark & \checkmark & \checkmark & \checkmark & \checkmark \\
	LAMMPS & \checkmark & \checkmark & \checkmark & \checkmark & \checkmark & \checkmark \\
	NAMD & \checkmark & \checkmark & \checkmark & \checkmark & \checkmark & \checkmark \\
	OpenEye & \checkmark & \checkmark & \checkmark & \checkmark & \checkmark & \checkmark \\
	CPMD & \checkmark & \checkmark & \checkmark & \checkmark & \checkmark & \checkmark \\
	GAMESS & \checkmark & \checkmark & \checkmark & \checkmark & \checkmark & \checkmark \\
	Gaussian &  & \checkmark & \checkmark & \checkmark & \checkmark &  \\
	NWCHEM & \checkmark & \checkmark & \checkmark & \checkmark & \checkmark & \checkmark \\
	Piny\_MD & \checkmark & \checkmark & \checkmark & \checkmark & \checkmark & \checkmark \\
	[1.0ex]
      \end{tabularx}};
      \begin{pgfonlayer}{background}
	\draw[rounded corners,top color=blue!30!black,bottom color=blue!10!white,draw=tigerspurple!30] ($(tbl.north west)+(0.14,0)$) rectangle ($(tbl.north east)-(0.13,0.9)$);
	\draw[rounded corners,top color=blue!15,bottom color=blue!15,draw=blue!15] ($(tbl.north east)-(0.13,0.6)$) rectangle ($(tbl.south west)+(0.13,0.2)$);
      \end{pgfonlayer}
    \end{tikzpicture}
  \end{center}
  }
\end{frame}

\subsection{Gaussian}
\begin{frame}
  \frametitle{\small Gaussian}
  \begin{block}{}
    \begin{itemize}
      \item One of the most popular packages in Computational Chemistry.
      \item Wavefunction based modeling of electronic structure of chemical systems.
      \item Model Chemistry: Molecular Mechanics (AMBER, Dreiding, UFF force fields), HF (RHF,UHF,ROHF), DFT, MP(2-5), CIS, CISD, CCSD, CCSD(T), G1,G2,CBS, CASSCF,GVB
      \item Basis Sets: Large Library of basis sets, can also include basis set in input file
      \item Capability: Energy, Gradients, Geometry Optimization, Transition State search, Frequency analysis, Solvation methods, ONIOM, IRC for reaction path, ADMP and BOMD for Molecular Dynamics.
    \end{itemize}
  \end{block}
\end{frame}

\begin{frame}
  \frametitle{\small Using Gaussian on LONI/LSU HPC Systems}
  \begin{exampleblock}{}
    \begin{itemize}
      \item Site specific license
      \begin{enumerate}
	\item {\color{tigersblue}Gaussian 03 and 09}
	\begin{itemize}
	  \item {\color{tigersblue}LSU Users}: Eric, {\color{green!30!black}Pandora, Pelican, Philip, Tezpur}
	  \item {\color{tigersblue}Latech Users}: Painter, Bluedawg
	\end{itemize}
	\item {\color{tigerspurple}Gaussian 03}
	\begin{itemize}
	  \item {\color{tigerspurple}ULL Users}: Oliver, \sout{Zeke}
	  \item {\color{tigerspurple}Tulane Users}: Louie, \sout{Ducky}
	  \item {\color{tigerspurple}Southern Users}: \sout{Lacumba}
	\end{itemize}
	\item UNO Users: No License
      \end{enumerate}
      \item Add {\texttt +gaussian-03/+gaussian-09} to your .soft file and resoft
      \item \alert{If your institution has license to both G03 and G09, have only one active at a given time.}
    \end{itemize}
  \end{exampleblock}
\end{frame}

\begin{frame}
  \begin{columns}
    \column{8.5cm}
    \vspace{-0.5cm}
    \begin{exampleblock}{Gaussian Input}
      \begin{tabular}{lcr}
	\%chk=h2o-opt-freq.chk   & & {\color{red}checkpoint file} \\
	\%mem=512mb              & & {\color{red}amount of memory} \\
	\%NProcShared=4          & & {\color{red}number of smp processors} \\
				 & & {\color{red}blank line} \\
	\#p b3lyp/6-31G opt freq & & {\color{red}Job description} \\
				 & & {\color{red}blank line} \\
	H2O OPT FREQ B3LYP       & & {\color{red}Job Title} \\
				 & & {\color{red}blank line} \\
	0 1                      & & {\color{red}Charge \& Multiplicity} \\
	O                        & & {\color{red}Molecule Description} \\
	H 1 r1                   & & {\color{red}in Z-matrix format} \\
	H 1 r1 2 a1              & & {\color{red}with variables} \\
				 & & {\color{red}blank line} \\
	r1 1.05                  & & {\color{red}variable value} \\
	a1 104.5                 & & {\color{red}} \\
				 & & {\color{red}blank line} \\
      \end{tabular}
    \end{exampleblock}
  \end{columns}
\end{frame}


\subsection{GAMESS}
\begin{frame}
  \frametitle{\small GAMESS}
  \begin{block}{}
    \begin{itemize}
      \item Wavefunction based modeling of electronic structure of chemical systems.
      \item Model Chemistry: HF,DFT,MP2,CI,CCSD(T),AM1,PM3
      \item Basis Sets: Most not all are available, others can be read from input file.
      \item Capability: Geometry optimization, Transition State search, Frequency calculations, MEP for reaction, Direct Dynamics.
    \end{itemize}
  \end{block}
\end{frame}

\begin{frame}
  \begin{columns}
    \column{11cm}
    \begin{exampleblock}{GAMESS Input}
      {\scriptsize
      \begin{tabular}{lr}
	\$CONTRL SCFTYP=RHF RUNTYP=OPTIMIZE       & {\color{red}Job Control Data} \\
	\quad COORD=ZMT NZVAR=0 \$END             & {\color{red}} \\
	\$STATPT OPTTOL=1.0E-5 HSSEND=.T. \$END   & {\color{red}Geometry Search Control} \\
	\$BASIS GBASIS=N31 NGAUSS=6               & {\color{red}Basis Set} \\
	\quad NDFUNC=1 NPFUNC=1 \$END             & {\color{red}} \\
	\$DATA                                    & {\color{red}Molecular Data Control} \\
	H2O OPT                                   & {\color{red}Job Title} \\
	Cnv 2                                     & {\color{red}Molecule Symmetry group and axis} \\
						  & {\color{red}} \\
	O                                         & {\color{red}Molecule Description} \\
	H 1 rOH                                   & {\color{red}in Z-Matrix} \\
	H 1 rOH 2 aHOH                            & {\color{red}} \\
						  & {\color{red}} \\
	rOH=1.05                                  & {\color{red}Variables} \\
	aHOH=104.5                                & {\color{red}} \\
	\$END                                     & {\color{red}End Molecular Data Control} \\
      \end{tabular}
      }
    \end{exampleblock}
  \end{columns}
\end{frame}

\subsection{NWChem}
\begin{frame}
  \frametitle{\small NWChem}
  \begin{block}{}
    \begin{itemize}
      \item NWChem provides many methods for computing the properties of molecular and periodic systems using standard quantum mechanical descriptions of the electronic wavefunction or density.
      \item Classical molecular dynamics capabilities provide for the simulation of macromolecules and solutions, including the computation of free energies using a variety of force fields.
      \item Model Chemistry: Hartree-Fock (RHF,UHF,ROHF), DFT, MP2, CASSCF, CCSD,CCSDT,CCSDTQ etc
      \item Methods: Single Point Energies, Geometry Optimization, Molecular Dynamics, numerical first and second derivatives, Vibrational Analysis, ONIOM, COSMO, Electron Transfer, vibrational SCF and DFT, Pseudopotential Plane-Wave Electronic Structure, Molecular Dynamics.
      \item Basis Sets: Vast Library at \url{https://bse.pnl.gov/bse/portal}
    \end{itemize}
  \end{block}
\end{frame}


\begin{frame}
  \fontsize{8}{8}{
  \begin{columns}
    \column{7cm}
    \vspace{-0.5cm}
    \begin{exampleblock}{NWCHEM Input}
      \begin{tabular}{lr}
	title "H2O"            & {\color{red}Job title} \\
	echo                   & {\color{red}echo contents of input file} \\
	charge 0               & {\color{red}charge of molecule} \\
	geometry               & {\color{red}geometry description in} \\
	\quad zmatrix          & {\color{red}z-matrix format} \\
	\quad\quad O           & {\color{red}} \\
	\quad\quad H 1 r1      & {\color{red}} \\
	\quad\quad H 1 r1 2 a1 & {\color{red}} \\
	\quad variables        & {\color{red}variables used with values} \\
	\quad\quad r1 1.05     & {\color{red}} \\
	\quad\quad a1 104.5    & {\color{red}} \\
	\quad end              & {\color{red}end z-matrix block} \\
	end                    & {\color{red}end geometry block} \\
	basis noprint          & {\color{red}basis description} \\
	\quad * library 6-31G  & {\color{red}} \\
	end                    & {\color{red}} \\
	dft                    & {\color{red}dft calculation options} \\
	\quad XC b3lyp         & {\color{red}} \\
	\quad mult 1           & {\color{red}} \\
	end                    & {\color{red}} \\
	task dft optimize      & {\color{red}job type: geometry optimization} \\
	task dft energy        & {\color{red}job type: energy calculation} \\
	task dft freq          & {\color{red}job type: frequency calculation} \\
      \end{tabular}
    \end{exampleblock}
  \end{columns}
  }
\end{frame}

\begin{frame}
  \frametitle{\small Job Types and Keywords}
  \vspace{-0.5cm}
  \begin{columns}
    \column{12cm}
    \begin{block}{}
      \begin{tabular}{|c|c|c|c|}
	\hline
	Job Type & Gaussian & GAMESS & NWCHEM\\
	\cline{2-4}
	& \# keyword & runtyp= & task \\       
	\hline
	Energy & sp & energy & energy \\
	Force & force & gradient & gradient \\
	Geometry optimization & opt & optimize & optimize \\
	Transition State & opt=ts & sadpoint & saddle \\
	Frequency & freq & hessian & frequencies, freq \\
	Potential Energy Scan & scan & surface & \checkmark \\
	Excited State & \checkmark & \checkmark & \checkmark \\
	Reaction path following & irc & irc & \checkmark \\
	Molecular Dynamics & admp, bomd & drc & dynamics, Car-Parrinello \\
	Population Analysis & pop & pop & \checkmark \\
	Electrostatic Properties & prop & \checkmark & \checkmark \\
	Molecular Mechanics & \checkmark & \checkmark & \checkmark \\
	Solvation Models & \checkmark & \checkmark & \checkmark \\
	QM/MM & oniom & \checkmark & qmmm \\
	\hline
      \end{tabular}
    \end{block}
  \end{columns}
\end{frame}

\subsection{CPMD}
\begin{frame}
  \frametitle{\small CPMD}
  \begin{block}{}
    \begin{itemize}
      \item Car Parrinello Molecular Dynamics
      \item Typical problems CPMD is used for
      \begin{itemize}
	\footnotesize{
	\item Liquid Structures
	\item Polarization effects
	\item Bond breaking/formations
	\begin{itemize}
	  \item Proton transfer
	\end{itemize}
	\item Dynamic/thermal properties (e.g. diffusion)
	\item Metadynamics
	\item QM/MM
	\item Path Integrals
	\item TDDFT
	}
      \end{itemize}
    \end{itemize}
  \end{block}
\end{frame}

\begin{frame}
  \frametitle{\small CPMD Input}
  \scriptsize{
  \begin{block}{}
    \begin{itemize}
      \item Divided into sections
      \item Only sections pertaining to your simulated model at hand need to be present.
      \begin{itemize}
	\scriptsize{
	\item Common sections: INFO, CPMD, SYSTEM, ATOMS, DFT
	\item Special sections: PIMD, PATH, RESP, LINRES, TDDFT,  PROP, HARDNESS, CLASSIC, BASIS, VDW, QMMM
	}
      \end{itemize}
      \item Sections are defined \texttt{\&Section\_Name} and followed by \texttt{\&END}, for example,
      \item[]\texttt{\&ATOMS}  
      \item[]\texttt{\quad(information)}
      \item[]\texttt{\&END}
      \item Each section has their own keywords
      \item Lines that do not match known keywords are ignored
      \item KEYWORDS HAVE TO BE IN ALL CAPS
      \begin{itemize}
	\scriptsize{
	\item kEYWORDS starting with a lower case character are ignored
	\item Useful feature to re-use the input file
	}
      \end{itemize}
      \item Order of keywords is arbitrary
    \end{itemize}
  \end{block}
  }
\end{frame}

\begin{frame}
  \frametitle{\small }
  \fontsize{7}{9}{
  \vspace{-0.25cm}
  \begin{exampleblock}{CPMD Input}
    \begin{tabular}{lr}
      \&CPMD & \\
      \quad OPTIMIZE GEOMETRY XYZ & \\
      \quad CONVERGENCE ORBITALS & \\
      \quad\quad 1.0d-7 & \\
      \quad CONVERGENCE GEOMETRY \\
      \quad\quad 5.0d-4 & \\
      \&SYSTEM & \\
      \quad SYMMETRY & \\
      \quad \quad SIMPLE CUBIC & \\
      \quad CELL & \\
      \quad \quad 16.00 1.0 1.0  0.0  0.0  0.0 & \\
      \quad CUTOFF & \\
      \quad \quad 60.0 & \\
      \&END & \\
      \&ATOMS & \\
      \quad *H\_MT\_LDA.psp & \\
      \quad \quad LMAX=S & \\
      \quad \quad \quad 2 & \\
      \quad \quad \quad 8.800   8.000   8.000 & \\
      \quad \quad \quad 7.200   8.000   8.000 & \\
      \&END & \\
    \end{tabular}
  \end{exampleblock}
  }
\end{frame}


\subsection{Amber}
\begin{frame}
  \frametitle{\small AMBER}
  \begin{block}{What is AMBER?}
    \begin{itemize}
      \item AMBER, Assisted Model Building with Energy Refinement refers to two things
      \begin{enumerate}
	{\scriptsize
	\item A collective name for a suite of programs that allow users to carry out molecular dynamics simulation
	\item A set of molecular mechanical force fields for the simulation of biomolecules
	}
      \end{enumerate}
    \end{itemize}
  \end{block}
  \begin{block}{Capabilities}
    {\scriptsize
    \begin{columns}
      \column{0.4\textwidth}
      \begin{itemize}
	\item Classical MD (NVT,NPT,etc)
	\item Force Fields
	\item QM/MM
	\item Free Energy Calculations
	\item Structural and Trajectory analysis
      \end{itemize}
      \column{0.4\textwidth}
      \begin{itemize}
	\item Parallelize dynamic codes
	\item Enhanced sampling (replica exchange MD)
	\item Explicit Solvent Models with PME
	\item Implicit Solvent Models with PB or GB approach
      \end{itemize}
    \end{columns}
    }
  \end{block}
\end{frame}

\begin{frame}
  \frametitle{\small Basic Information Flow in AMBER}
  \begin{block}{}
    \begin{enumerate}
      \item obtain and edit initial structure
      \item prepare input parameter and topology file
      \item run simulations and save trajectory
      \item analyze output and trajectory files
    \end{enumerate}
    \scriptsize{
    \tikzstyle{block} = [rectangle, draw, fill=yellow!75!red,text width=6em, text centered, rounded corners, minimum height=4em]
    \tikzstyle{line} = [draw, -latex']
    \begin{tikzpicture}[node distance = 2cm, auto]
      \node [block](step1){\textbf{PDB} file};
      \node [block, right of=step1, node distance=3cm](step2){\textbf{Preparatory Programs}\\LEaP\\antechamber};
      \node [block, right of=step2, node distance=3cm](step3){\textbf{Simulation Programs}\\Sander\\PMEMD\\NMode};
      \node [block, right of=step3, node distance=3cm](step4){\textbf{Analysis Programs}\\Ptraj\\MM/PBSA};
      \path [line](step1) -- (step2);
      \path [line](step2) -- (step3);
      \path [line](step3) -- (step4);
    \end{tikzpicture}
    }
  \end{block}
\end{frame}

\begin{frame}
  \frametitle{\small Topology and Parameter File}
  \begin{block}{Topology information includes}
    \begin{itemize}
      \item atom types are assigned to identify different elements and different molecular orbital environments
      \item charges are assigned to each atom
      \item connectivities between atoms are established
    \end{itemize}
  \end{block}
  \begin{block}{Parameter information includes}
    \begin{itemize}
      \item force constants necessary to describe the bond energy, angle energy, torsion energy, nonbonded interactions (van der Waals and electrostatics)
      \item other parameters for setting up the energy calculations (GB radii, FEP parameter sets)
    \end{itemize}
  \end{block}
\end{frame}

\begin{frame}
  \frametitle{\small Preparatory Programs}
  \begin{block}{LEaP}
    \begin{itemize}
      \item Includes a tex-based interface - tleap and a graphical user interface - xleap
      \item Capabilities
      \begin{enumerate}
	{\scriptsize
	\item[$\vardiamond$] Read AMBER force field information
	\item[$\vardiamond$] Read and write files containing structural information (i.e.PDB files)
	\item[$\vardiamond$] Construct new residues and molecules
	\item[$\vardiamond$] Link together residues and create nonbonded complexes of molecules
	\item[$\vardiamond$] Place counterions around a molecule; Solvate molecules; Mdoify internal coordinates within a molecule
	\item[$\vardiamond$] Generate files that contain topology and parameters for AMBER
	\item[$\vardiamond$] Set atomic charges, identify the position of disulphide bridges, delete bonds, addition of atoms, ions, etc ...
	}
      \end{enumerate}
    \end{itemize}
  \end{block}
\end{frame}

\begin{frame}
  \frametitle{\small Using tleap/xleap}
  \begin{block}{\small loading PDB file, adding the disulfide cross links and saving files}
    \begin{itemize}
      \item[]source leaprc.ff03\let\thefootnote\relax\footnotetext{\tiny Amber 11 Manual Page 17}
      \item[]bpti = loadPdb 6pti.mod.pdb
      \item[]bond bpti.5.SG bpti.55.SG
      \item[]bond bpti.14.SG bpti.38.SG
      \item[]bond bpti.30.SG bpti.51.SG
      \item[]saveAmberParm bpti prmtop prmcrd
      \item[]quit
    \end{itemize}
  \end{block}
\end{frame}



\begin{frame}
  \frametitle{\small Simulation Programs}
  \begin{block}{Sander}
    \begin{itemize}
      \item Sander: Simulated Annealing with NMR-Derived Energy Restraints
      \begin{enumerate}
	{\scriptsize
	\item[$\vardiamond$] Energy minimization, molecular dynamics and NMR refinements
	\item[$\vardiamond$] Free energy calculations (Umbrella Sampling; SMD; etc)
	\item[$\vardiamond$] QM/MM implementation (EVB; semi-empirical/AMBER)
	\item[$\vardiamond$] Polarizable force field (AMOEBA)
	\item[$\vardiamond$] Enhanced Sampling (REMD; LES, etc)
	}
      \end{enumerate}
      \item Usage: \fontsize{7.5}{10}{\color{blue}\texttt{sander -i mdin -o mdout -p prmtop -c inpcrd -r restrt}}
    \end{itemize}
%     \begin{itemize}
      {\fontsize{6}{2}{
      \qquad\qquad{\texttt{mdin: input control data for minimization/MD run.}}\\
      \qquad\qquad{\texttt{mdout: output file for user readable state info and diagnostics}}\\
      \qquad\qquad{\texttt{prmtop: molecular topology, force fields etc}}\\
      \qquad\qquad{\texttt{inpcrd: initial coordinates and velocities}}\\
      \qquad\qquad{\texttt{restrt: restart filename}}
      }}
%     \end{itemize}
  \end{block}
\end{frame}

\begin{frame}
  \frametitle{\small Perform Minimization}
  \begin{block}{\small 200 steps of minimization, generalized Born solvent model}
    \begin{itemize}
      \item[]\&cntrl\let\thefootnote\relax\footnotetext{\tiny Amber 11 Manual Page 17}
      \item[]maxcycle=200, 
      \item[]imin=1, 
      \item[]cut=12.0, 
      \item[]igb=1, 
      \item[]ntb=0, 
      \item[]ntpr=10,
      \item[]/
    \end{itemize}
  \end{block}
  \begin{exampleblock}{Run Sander}
    sander -i min.in -o 6pti.min1 -c prmcrd -r 6pti.min1.xyz
  \end{exampleblock}
\end{frame}


\subsection{LAMMPS}
\begin{frame}
  \frametitle{\small LAMMPS}
  \begin{itemize}
    \item LAMMPS stands for Large-scale Atomic/Molecular Parallel Simulator.
    \item LAMMPS is a classical molecular dynamics code that models an ensemble of particles in a liquid, solid, or gaseous state designed to run efficiently on parallel computers.
    \item It can model atomic, polymeric, biological, metallic, granular, and coarse-grained systems using a variety of force fields and boundary conditions.
    \item LAMMPS can model systems with only a few particles up to millions or billions.
    \item LAMMPS is designed to be easy to modify or extend with new capabilities, such as new force fields, atom types, boundary conditions, or diagnostics.
    \item LAMMPS runs efficiently on single-processor desktop or laptop machines, but is designed for parallel computers.
    \item It is an open-source code, distributed freely under the terms of the GNU Public License (GPL).
  \end{itemize}
\end{frame}

\begin{frame}
  \frametitle{\small Running LAMMPS}
  \begin{itemize}
    {\footnotesize
    \item LAMMPS doesn't
    \begin{enumerate}
      {\footnotesize
      \item Build molecular systems
      \item Assign force-dield coefficients auto-magically
      \item Compute lots of diagnostics on-the-fly
      \item Visualize your output
      }
    \end{enumerate}
    \item LAMMPS version "4 May 2011" is installed on all LONI Dell Linux Clusters.% and LSU HPC machines, Tezpur (Linux) and Pandora (AIX).
    \item Add the appropriate soft keys to your .soft file
    \begin{enumerate}
      {\footnotesize
      \item[$\vardiamond$]  +lammps-4May11-intel-11.1-mvapich-1.1
      }
    \end{enumerate}
    \item Command line options for running LAMMPS
    \begin{itemize}
      {\footnotesize
      \item[$\vardiamond$] -in inputfile: specify input file
      \item[$\vardiamond$] -log logfile: specify log file
      \item[$\vardiamond$] -partition MxN L: Run on (MxN)+L processors with M partitions on N processors each and 1 partition with L processors
      \item[$\vardiamond$] -screen file: Specify a file to write screen information
      }
    \end{itemize}
    }
  \end{itemize}
\end{frame}

\begin{frame}
  \frametitle{\footnotesize LAMMPS Input}
  \begin{itemize}
    {\scriptsize
    \item Reads an input script in ASCII format one line at a time.
    \item Input script consists of 4 parts
    \begin{enumerate}
      {\scriptsize
      \item Initialization: Set parameters that need to be defined before atoms are created or read-in from a file.
      \item[] \texttt{units, dimension, newton, processors, boundary, atom\_style, atom\_modify}
      \item Atom definition: \texttt{read\_data, read\_restart, lattice, region, create\_box, create\_atoms, replicate}
      \item Settings: Once atoms and molecular topology are defined, a variety of settings can be specified: force field coefficients, simulation parameters, output options, etc. 
      \item[] \texttt{pair\_coeff, bond\_coeff, angle\_coeff, dihedral\_coeff, improper\_coeff, kspace\_style, dielectric, special\_bonds, neighbor, neigh\_modify, group, timestep, reset\_timestep, run\_style, min\_style, min\_modify, fix, compute, compute\_modify, variable}
      \item Run a simulation: A MD is run using the run command. Energy minimization (molecular statics) is performed using the minimize command. A parallel tempering (replica-exchange) simulation can be run using the temper command.
      }
    \end{enumerate}
    \item \url{http://lammps.sandia.gov/doc/Section_commands.html}
    }
  \end{itemize}
\end{frame}

\begin{frame}
  \frametitle{}
  {\scriptsize
  \vspace{-0.3cm}
  \begin{exampleblock}{LAMMPS Input for 3d Lennard-Jones Melt}
    \begin{tabular}{lr}
    units   lj & {\color{red}style of units used for a simulation}\\
    atom\_style  atomic & {\color{red}what style of atoms to use in a simulatio}\\
     & \\
    lattice   fcc 0.8442 & {\color{red}Lattice structure}\\
    region    box block 0 20 0 20 0 20 & {\color{red}define simulation region}\\
    create\_box  1 box & {\color{red}create simulation box}\\
    create\_atoms  1 box & {\color{red}create atoms on lattice}\\
    mass    1 1.0 & {\color{red}}\\
    velocity  all create 3.0 87287 & {\color{red}}\\
     & \\
    pair\_style  lj/cut 2.5 & {\color{red}Formula for Pairwise Interaction}\\
    pair\_coeff  1 1 1.0 1.0 2.5 & {\color{red}pairwise force field coefficients}\\
     & \\
    neighbor  0.3 bin & {\color{red}pairwise neighbor lists}\\
    neigh\_modify  every 20 delay 0 check no & {\color{red}}\\
     & \\
    fix   1 all nve & {\color{red}constant NVE time integration}\\
     & \\
    dump    id all atom 10 dump.melt & {\color{red}For visualization}\\
    thermo    50 & {\color{red}output thermodynamics data}\\
    run   2500 & {\color{red}Simulation Steps}\\
    \end{tabular}
  \end{exampleblock}
  }
\end{frame}


\subsection{NAMD}
\begin{frame}
  \frametitle{\small NAMD}
  \begin{block}{}
    \begin{itemize}
      \item NAMD is a parallel molecular dynamics code designed for high-performance simulation of large biomolecular systems. 
      \item Based on Charm++ parallel objects, NAMD scales to hundreds of processors on high-end parallel platforms and tens of processors on commodity clusters using gigabit ethernet.
      \item NAMD uses the popular molecular graphics program VMD for simulation setup and trajectory analysis, but is also file-compatible with AMBER, CHARMM, and X-PLOR.
    \end{itemize}
  \end{block}
\end{frame}

\begin{frame}
  \frametitle{\small What is needed to run NAMD?}
  {\scriptsize
  \begin{itemize}
    \item Protein Data Bank (pdb) file which stores atomic coordinates and/or velocities for the system.
    \item Protein Structure File (psf) which stores structural information of the protein, such as various types of bonding interactions.
    \item A force field parameter file. A force field is a mathematical expression of the potential which atoms in the system experience. CHARMM, X-PLOR, AMBER, and GROMACS are four types of force fields, and NAMD is able to use all of them. The parameter file defines bond strengths, equilibrium lengths, etc. 
    \item A configuration file, in which the user specifies all the options that NAMD should adopt in running a simulation.
  \end{itemize}
  \begin{block}{NAMD configuration file}
    \begin{itemize}
      \item A NAMD configuration file contains a set of options and values which determine the exact behavior of NAMD, what features are active or inactive, how long the simulation should continue, etc.
      \item The following parameters are \textit{required} for every NAMD simulation
      \item[] \texttt{numsteps, coordinates, structure, parameters, exclude, outputname} and one of the following three: \texttt{temperature, velocities, binvelocities}.
    \end{itemize}
  \end{block}
  }
\end{frame}

\begin{frame}
  \frametitle{\small }
  {\scriptsize
  \vspace{-0.35cm}
  \begin{exampleblock}{NAMD Configuration File for decalanin}
    \begin{tabular}{lcr}
%     # NAMD CONFIGURATION FILE FOR DECALANIN
      numsteps  10000  &  &  {\color{red}protocol params}\\
      & &  \\
      coordinates alanin.pdb & &  {\color{red}initial config} \\
      temperature 300K & &  \\
      seed    12345 & &  \\
      & &  \\
      outputname  ./alanin & &  {\color{red}output params} \\
      binaryoutput  no & &  \\
      & &  \\
      timestep  1.0 & &  {\color{red}integrator params} \\
      & &  \\
      structure alanin.psf & &  {\color{red}force field params}\\
      parameters  alanin.params & &  \\
      exclude   scaled1-4 & &  \\
      1-4scaling  1.0 & &  \\
      switching on & &  \\
      switchdist  8.0 & &  \\
      cutoff    12.0 & &  \\
      pairlistdist  13.5 & &  \\
      margin    0.0 & &  \\
      stepspercycle 20 & &  \\
    \end{tabular}
  \end{exampleblock}
  }
\end{frame}

\subsection{Gromacs}
\begin{frame}
  \frametitle{\small Gromacs}
  \begin{block}{}
    \begin{itemize}
      \item GROMACS (Groningen Machine for Chemical Simulations) is a versatile package which performs molecular dynamics, i.e. simulates the Newtonian equations of motion for systems with hundreds to millions of particles.
      \item It is primarily designed for biochemical molecules like proteins and lipids that have a lot of complicated bonded interactions, but since GROMACS is extremely fast at calculating the nonbonded interactions (that usually dominate simulations) many groups are also using it for research on non-biological systems, e.g. polymers. 
      \item GROMACS calling structure is quite complicated, and it is not recommended that users proceed without at least becoming familiar with the program.
      \item You will need a protein structure file (pdb file) and a file containing energy minimization data.
    \end{itemize}
  \end{block}
\end{frame}

\begin{frame}
  \frametitle{\small Using Gromacs: Simulation of Peptide in Water}
  \vspace{-0.2cm}
  \begin{block}{\small Files at /home/apacheco/CompChem/CMD/GROMACS}
    \begin{enumerate}
      {\tiny
      \item Generate topology file using pdb2gmx
      \item[] \texttt{{\color{magenta}pdb2gmx\_d} -f {\color{blue}cpeptide.pdb} -o {\color{DarkGreen}cpeptide.gro} -p {\color{DarkGreen}cpeptide.top}}
      \item Define the water box size
      \item[] \texttt{{\color{magenta}editconf\_d} -f {\color{blue}cpeptide.gro} -o {\color{DarkGreen}cpeptide.gro} -d 0.5}
      \item Solvate the peptide
      \item[] \texttt{{\color{magenta}genbox\_d} -cp {\color{blue}cpeptide.gro} -cs -o {\color{DarkGreen}cpeptide\_b4em.gro} -p {\color{DarkGreen}cpeptide.top}}
      \item Energy Minimization, step 1: preprocess the input files.
      \item[] \texttt{{\color{magenta}grompp\_d} -f {\color{red}em} -c {\color{blue}cpeptide\_b4em} -p {\color{blue}cpeptide} -o {\color{DarkGreen}cpeptide\_em}}
      \item Energy minimization, step 2
      \item[] \texttt{{\color{magenta}mdrun\_d} -s {\color{blue}cpeptide\_em} -o {\color{DarkGreen}cpeptide\_em} -c {\color{DarkGreen}cpeptide\_b4pr} -v }
      \item Position Restrained MD, step 1: generate the binary topology file
      \item[] \texttt{{\color{magenta}grompp\_d} -f {\color{blue}pr} -c {\color{blue}cpeptide\_b4pr} -r {\color{blue}cpeptide\_b4pr} -p {\color{blue}cpeptide} -o {\color{DarkGreen}cpeptide\_pr} }
      \item Position Restrained MD, step 2: run the simulation
      \item[] \texttt{{\color{magenta}mdrun\_d} -s {\color{blue}cpeptide\_pr} -o {\color{DarkGreen}cpeptide\_pr} -c {\color{DarkGreen}cpeptide\_b4md} -v}
      \item MD Simulation, step 1: generate the binary topolgy file
      \item[]\texttt{{\color{magenta}grompp\_d} -f {\color{blue}md} -c {\color{blue}cpeptide\_b4md}  -p {\color{blue}cpeptide} -o {\color{DarkGreen}cpeptide\_md} }
      \item MD simulation, step 2: run the simulation
      \item[] \texttt{{\color{magenta}mdrun\_d} -s {\color{blue}cpeptide\_md} -o {\color{blue}cpeptide\_md} -c {\color{DarkGreen}cpeptide\_after\_md} -v }

      }
    \end{enumerate}
  \end{block}
\end{frame}


\section{Exercises}
\begin{frame}
  \frametitle{\small Exercises}
  {\scriptsize
  \begin{exampleblock}{Goals}
    \begin{itemize}
      \item Create Gaussian (or GAMESS/NWChem) Input files for 
      \begin{enumerate}
	{\scriptsize
	\item Optimization
	\item Scan: relaxed and optimized
	\item Properties: Electrostatics, MOs, etc
	\item AIMD
	}
      \end{enumerate}
    \end{itemize}
  \end{exampleblock}
  \begin{block}{Assignment}
    \begin{itemize}
      \item Molecule: $[NH_3-H-NH_3]^+$
      \item Method/Basis: B3LYP/6-311++G(D,P)
      \item Job Type: Geometry Optimization + Freqeuncy
      \item Scan along the $N-H-N$ axis by moving the $H$, bonded to one $N$ to the other and analyze and discuss  the Potential Energy Curve.
      \item Population Analysis: Calculate and visualize MO's and electrostatic potential around the molecule.
      \item Optional: Run an AIMD simulation for at least 2ps and obtain a spectra. Compare with the harmonic spectra.
    \end{itemize}
  \end{block}
  }
\end{frame}
% \section{Tips for Quantum Chemical Calculations}

\begin{frame}
  \frametitle{\small Choosing Basis Sets}\vspace{1cm}
  \vspace{-0.5cm}
  \begin{colorblock}{white}{blue!30!black}{black}{white!80!black}{Choice of Basis Set}
    \begin{itemize}
      \item STO-3G is too small.
      \item 6-31G* and 6-31G** give reasonable results.
      \item For greater accuracy, use correlation consistent basis sets e.g. cc-pVTZ
      \item For anions and probably excited states, use basis sets with diffuse functions (aug, +). e.g. 6-31+G*, aug-cc-pVTZ
    \end{itemize}
  \end{colorblock}
% \vspace{0.5cm}
  \begin{colorblock}{white}{blue!30!black}{black}{white!80!black}{GAMESS Basis Sets}
    \begin{itemize}
      \item In GAMESS, you can create a file containing basis sets that you want to use
      \item Define \texttt{EXTBAS} variable which points to the basis set file
      \item See pseudo basis example
      \item In input line, if you name your basis set as \texttt{STTGRD}, then add
      \texttt{\$BASIS EXTFIL=.T. GBASIS=STTGRD \$END}
    \end{itemize}
  \end{colorblock}
\end{frame}

\begin{frame}
  \frametitle{\small Method and SCF Convergence}
  \vspace{-0.2cm}
  \begin{colorblock}{white}{blue!30!black}{black}{white!80!black}{Choice of Method}
    \begin{itemize}
      \item Always pick DFT over HF
      \item In general: HF < DFT $\sim$ MP2 < CCSD < CCSD(T)
      \item Pay attention to scaling behavior
    \end{itemize}
  \end{colorblock}
% \vspace{0.5cm}
  \begin{colorblock}{white}{blue!30!black}{black}{white!80!black}{SCF Convergence Issues}
    \begin{itemize}
      \item Has SCF (HF and DFT) really converged? Important if you use iop(5/13) in Gaussian route card.
      \item If SCF doesn't converge:
      \begin{enumerate}
	{\footnotesize
	\item Increase maximum number of SCF iterations.
	\begin{itemize}
	  {\footnotesize
	  \item GAMESS: max 200 SCF iterations cannot be increased further.}
	\end{itemize}
	\item Use smaller basis set as an initial guess.
	\item Try level shifting
	\item Use forced convergence method:
	\begin{itemize}
	  {\footnotesize
	  \item Gaussian: SCF=QC, XQC or DM and item 1 above
	  \item GAMESS: SOSCF}
	\end{itemize}
	}
      \end{enumerate}
    \end{itemize}
  \end{colorblock}
\end{frame}

\begin{frame}
  \frametitle{\small Geometry Optimizations}
  \begin{colorblock}{white}{blue!30!black}{black}{white!80!black}{Geometry Optimizations}
    \begin{itemize}
      \item Many problems in computational chemistry are optimization problems: i.e., finding the "stationary points" where a multidimensional function has vanishing gradients.
      \item The energy as a function of nuclear coordinates. Minima, transition states may be of interest.
      \item Make sure that the geometry optimization actually converges.
      \item Run a frequency calculation to check whether the geometry is a local minima (zero imaginary frequencies) or a transition state (only one imaginary frequency)
      \item Tighten convergence criterion to remove unwanted imaginary frequencies.
      \item Having more than 3N-6 (3N-5 for linear) frequencies implies that you are not at a minimum. Double check and tighten convergence if necessary.
    \end{itemize}
  \end{colorblock}
\end{frame}


\end{document}
